\section{WORKBOOK LESSON 5
AYURVEDIC CONSTITUTION
BODY AND MIND}
 

Vedic Astrology provides us the keys to determine the basic health strength of a person, regardless of what their Ayurvedic type may happen to be. Such an examination is probably the first step and foundation for any real medical astrology, whatever system of medicine it may relate to. It also helps us understand the vitality and longevity of a person, which are likely to be reduced if the constitution is not strong. In the following section we will provide a summary of this data. Such material reflects a more advanced level of study and is presented for the sake of those interested in such a deeper examination.

 

This Lesson follows Part II. Lesson 5 of the course that provides ten sample constitutional charts according to the Doshas of Vata, Pitta and Kapha. Here we add more detailed points of examination.

 

To determine the health and wellness potential of a person we must examine the general strength of the chart relative to the factors that represent the physical body.

 
\begin{enumerate}
\item First in importance is the strength of the Ascendant and its lord, which represents the body and life of the person.
\item Second is the strength of the sixth house and its lord, which represents the immunity of a person as well as their disease potential.
\item Third is the eighth house and its lord, which represents the longevity of a person as well as their potential for catastrophic illness or injury. The third house and its lord, as eighth from the eighth can also be examined but in a secondary way.
\end{enumerate}
 

The same positions in the Navamsha, which covers all the domains of life, should be examined. Drekkana positions, indicating vitality, are usually important as well. Sometimes Dwadashamsha and Trimshamsha positions can be looked into for further clarification. Dwadashamsha helps indicate the heredity of the person. Trimshamsa is another indication of difficulties or calamities. The books of  K. S. Charak, Essentials of Medical Astrology and Subtleties of Medical Astrology, can be examined for more detail in regard to most of these issues.



\subsection{The Ascendant and Basic Birth Chart}
 

In determining the basic health potential of a person the basic factors of chart strength come into play. Benefics do well located in angles or trines (1, 4, 5, 7, 9, 10). Malefics in upachaya houses (3, 6, 11) are good. Benefics in Duhsthanas (6, 8, 12) and malefics in angles and trines (1, 4, 5, 7, 9, 10) are bad. The second and the seventh house can be bad as potential marakas. This basic benefic-malefic chart orientation shows how our vitality is strengthened or weakened.

 

Natural benefics located in the angles give strong health, provided they are not retrograde or afflicted.
 

Retrograde status weakens these benefics or makes them unpredictable. Affliction weakens them but they still provide some benefit. The Ascendant is the most important angle for health because it relates to the body. Several benefics in angles greatly fortifies the health and destiny of a person.

 

Jupiter is the best benefic for health and Venus is second. Mercury is third in strength and may not provide much help if it is not associated with another benefic. However, its association with the Sun, unless very close, does not weaken the health.

 

The Moon may not be very helpful for health, if it is weak in brightness. The Moon can prove unhelpful if afflicted in an angle, particularly if it also rules a bad house from the Ascendant like the sixth, eighth or the twelfth. An afflicted Moon in the seventh (a maraka house) can be bad for health, particularly in childhood.

 

Jupiter is very good for health when located in trines, from which it aspects the Ascendant, more so if it also aspects the Ascendant lord.
 

A good Jupiter in angles or trines protects the health and cancels many afflictions. Benefics in the fifth house can be good for health because the fifth is a house of positive health (in contrast to the sixth that is a house of disease). The fifth is the best trine for health. The ninth is not so specific for health but a good ninth house often gives help should a person fall ill.

 

Natural malefics located in angles in the birthchart damage the health, particularly if they are found in the Ascendant or aspecting the Ascendant and its lord.
 

Natural malefics in the seventh can become potent marakas (death or harm causing planets), owing to the maraka nature of that house. Saturn weakens vitality and Mars causes toxins and infections. Retrograde malefics cause yet more harm. Rahu-Ketu in the first and seventh houses weakens the health. Several malefics in angles, with no benefic influence, can cause severe health problems.

 

Natural benefics in houses 2 and 12 fortify the Ascendant by surrounding it with good influences. The Ascendant suffers if hemmed in by malefics in 2 and 12. The same is true of the Moon. Benefics in the second and twelfth from it aid in health, malefics in these positions promote disease.

 

A vargottama Ascendant (same sign in both Rashi and Navamsha) also fortifies the health of the person, but is not a major influence.

 

\subsubsection{Ascendant Lord}

 

The Ascendant lord does best if associated with or aspected by natural benefics, particularly Jupiter.
 

It suffers from aspect or association by natural malefics, particularly Saturn. It also suffers when combust, particularly within a degree of the Sun, and is weakened in regard to health if retrograde. Debilitated it also can harm the health or lower resistance to disease, while exalted it improves health.

 

Exchanges between the Ascendant lord and lords of the sixth and the eighth can cause major health problems, particularly if this exchange involves the Ascendant or occurs in difficult houses (6, 8, 12) or in maraka houses (2, 7).

 

The Ascendant lord can suffer from location with or aspect by temporal malefics (lords of the sixth and eighth mainly), particularly if this occurs in bad houses (6, 8, 12) or if the Ascendant lord is otherwise weak or afflicted. Even Jupiter as a malefic lord may not help an otherwise afflicted Ascendant lord. A vargottama Ascendant lord also strengthens the health.

 

\subsubsection{House Position}

 

The Ascendant lord occupying or aspecting the Ascendant helps our health because it strengthens its own house. If Ascendant lord occupies the Ascendant and is afflicted, however, this is doubly bad, because the both the Ascendant and its lord become harmed.

 

The Ascendant lord strengthens whatever house it tenants, even if that is a difficult house.
 

An unafflicted Ascendant lord located in the sixth house is good for health and improves resistance to disease. An unafflicted Ascendant lord located in the eighth house is good for longevity. However if the Ascendant lord is afflicted in these houses, retrograde, debilitated, or associated with Rahu and Ketu, it can cause great harm, particularly if it is aspected by the lords of these malefic houses.

 

A naturally malefic Ascendant lord (Mars or Saturn) is particularly good for health if located in the sixth house in the birthchart because malefics do well in the sixth.
 

If malefic influences do not involve the Ascendant or its lord, they may not damage our health, though they may harm other aspects of our lives. If they do not affect the sixth or eighth houses either, our lives will be free of significant health problems.

 

\subsection{Sixth and Eighth Houses}
 

Houses six and eight are the two disease causing houses. Benefics in the sixth and eighth are weak and cause disease, unless they are both in their own or exalted signs, unafflicted and not retrograde.
 

Benefics with Ketu in these houses can be an exception. Jupiter can be all right in the sixth or eighth even if with Ketu (not Rahu) unless under bad aspect. Mercury with Ketu in these houses can be good for health. But in both instances the patient must engage in a pure life and perhaps spiritual practices as well. Such positions are good for mantra and practice of mantra will help improve the health if weak. However if Jupiter or Mercury are the Ascendant lord in these houses the results for the health can be less helpful.

 

Malefics in the sixth house give good health, sports ability and resistance to disease (they give this in the third and eleventh houses also). But it is better if only one malefic is there, not several that can be complicating.
 

They give a competitive nature. If diseases occur they give the resistance to overcome them. More than one malefic in the sixth house, however, can cause harm. Malefics in the sixth, however, can cause accident. We note that people with good health and athletes may push themselves too hard and cause some injury. This shows the good and the bad of sixth house malefics.

 

Mars in the eighth house usually shows injury or disease, and can make a person accident prone. Saturn in the eighth house aids in longevity unless it is afflicted, retrograde, or the chart is otherwise weak, in which case it can show severe or chronic diseases.

 

\subsubsection{Sixth and Eighth Lords}

 

The sixth lord in its own house strengthens health, immunity and resistance to disease, unless is with or afflicts the Ascendant lord.
 

However, if afflicted it causes health problems. An exalted sixth lord improves health, while debilitated it damages it. A strong sixth lord improves our health. A weak sixth lord damages it. The sixth lord indicates our immune system. If strong we have a good immune system, if weak our immune system suffers.

 

The sixth or eighth lords located in the Ascendant can damage the health, even if they are natural benefics, particularly if they are retrograde or otherwise afflicted. If they aspect the Ascendant lord as well, the results can be severe. This is because they bring disease causing factors into the physical body (the Ascendant).

 

The sixth lord located in the eighth house gives many, chronic or debilitating diseases. So does the eighth lord in the sixth house. Such disease will cause some permanent health damage, like scars, damaged limbs or organs, or permanently weakened vitality. If the Ascendant lord is involved in this combination the results may be worse and greatly harm the health of a person.

 

The eighth lord located in the eighth house gives good longevity and also spirituality, if unafflicted. A strong eighth lord aids in longevity, though it may harm the other factors relating to the houses that it affects.
 

The eighth lord located in the eleventh house can give abundant long life, particularly if a natural malefic, especially Saturn.

 

The third house as eighth from the eighth, and the sixth house as sixth from the sixth, can be examined as secondary factors for health and longevity, particularly the latter. A strong third house and its lord gives vitality, creativity, curiosity and a good will to live.

 

Relative to such Duhsthana rulers the issue of Vipreet Raja Yoga comes up. It is generally held by this rule that the lords of six, eight and twelve, if located in other Duhsthanas can give good results. However, relative to health this is not always the case, particularly if they are associated with the Ascendant lord or with the Moon, in which case such positions can be very bad for the health.

 

\subsection{The Moon}
 

The Moon indicates more the emotional than the physical nature, but the two are closely related. If we are emotionally depressed, our health vitality suffers. If we are emotionally happy, our vitality improves. The Moon also indicates our ability to find nourishment and to receive right care in life that impacts our health.

 

\subsubsection{House Position of the Moon}

The Moon causes health problems if located in difficult houses 6, 8 and 12, particularly if weak or afflicted.
 

Afflicted in these houses it can cause severe health problems in childhood (Balarishta) and creates emotional affliction as well (particularly if the fourth house or its lord are also afflicted). In this regard the eighth house position is the worst. Sometimes the seventh house is added to these.

 

If the Moon is located in these difficult houses but not afflicted it can be good for health or more likely make one into a healer. This is provided that it is a night birth during the bright half of the month (waxing moon or shukla paksha), or a day birth during the dark half of the month (waning moon or krishna paksha).
 

If the Moon is afflicted in these houses even such positive birth time factors cannot help it. This is easy to determine from the chart by noting the positions of the Moon and the Sun.

 

If the Sun is below the horizon (in houses 1-6), it is a night birth. If it is above the horizon (in houses 7-12) it is a day birth. If the Sun is in either house 1 or 7, note its degree position. It will have to be less than that of the Ascendant for a day birth, and less than that of the seventh house cusp for a night birth.
If the Moon is in houses 1-6 from the Sun, the person is born during the bright half of the Moon. If it is in houses 7-12 from the Sun, the birth is in the dark half of the Moon. Again if the Moon is in the same sign as the Sun, it will have to be more in degrees to be in the bright half. If it is in the seventh from the Sun, it will have to be more in degrees to be in the dark half.
 

Again this cancellation of affliction may not work if the Moon is under difficult aspects as well as being in one of these difficult houses. Must be examined carefully.

 

\subsection{Signs and Disease Potential}
 

Some rising signs are better than others for giving health. Relative to the three qualities:

 

Cardinal signs are usually best for health, giving vitality and the capacity for action, but can suffer from acute problems and from injuries.
Mutable signs give psychological sensitivity and can be bad for health, particularly for chronic nervous or endocrine diseases, and can give a weak constitution. However the person may not have acute or severe diseases.
Fixed signs when unafflicted give good health, when afflicted can cause long standing diseases.
 

 

\subsubsection{Elements of Signs}

 

Fire signs usually provide good resistance to disease because they give strong immune systems, allow us to burn up toxins and give good digestion. Leo is the best in this regard, Aries is second, and Sagittarius is third. Ayurveda also tells us that a good Agni or digestive fire is the key to health.
Water signs cause health problems through obesity. A watery body allows bacteria and other pathogens to flourish. Water signs can also show the growth of tumors. Pisces is the worst in this regard, Cancer the second, and Scorpio third. However if the person controls his weight, water signs can be good for health. Ayurveda tells us that Kapha (watery) constitution is the best for health, if they are not overweight.
Air signs are better for mental development and are not always specific to physical health. Yet they can show weak health because of their airy and ungrounded nature. Generally Libra is the best for health, Gemini is second and Aquarius is third. According to Ayurveda the Vata or air constitution is generally weakest for health.
Earth signs connect us with the physical body and so when afflicted can show significant health problems. Generally Taurus is the best for health, Capricorn second and Virgo third. We have already examined the relationship of Virgo with health problems.
 

\subsubsection{Planetary Rulership of Signs}

 

Venus ruled Ascendants are not always good for health. Venus, the lagna lord, rules also either the sixth house of disease in the case of Taurus, or the eighth house of death in the case of Libra. The natives can suffer from over enjoyment or sexual indulgence. Afflictions to the Ascendant and its lord can go to a deep level (that of Venus, the reproductive tissue). Generally Taurus is worse than Libra, being an earth (body) sign.

 

Mars ruled Ascendants make us physically active and are better for health but can cause injury, infections and acute diseases. Mars rules the sixth for Scorpio Ascendant, a house of accident, and the eighth for Aries, a house of calamity. Generally Scorpio is worse than Aries, being itself the eighth sign in the zodiac.

 

Mercury ruled signs are prone to mental and nervous system disorders that lead to chronic diseases. Not only are Mercury ruled signs, Gemini and Virgo, dual or mutable in nature, making them unstable, so is their ruler Mercury. Virgo is worse than Gemini in this respect, being an earth (body) sign.

 

Jupiter ruled signs are usually better for health but Sagittarius is much better than Pisces. Saturn ruled signs are prone to difficulties in life, which can extend into the field of health.

 

\subsubsection{Sensitive Signs}

 

Virgo is difficult for health, being the sixth sign and general health-disease indicator. Pisces as mutable water is also difficult and can give emotional sensitivity that leads to disease.

 

The Virgo-Pisces axis is sensitive to health both physically and psychologically, but can also make people good healers.
 

Planets along this axis should be examined for health implications, particularly if they are malefics or if this axis involves the first and seventh houses, the second and eight houses, or the sixth and twelve houses. Rahu and Ketu along the Virgo-Pisces axis can damage the health of the person, regardless of the houses involved.

 

Gemini is also a sensitive sign for health, particularly for the mind and the nervous system. It has a strong Vata dosha influence by its mutable airy nature. Yet may be a healer as well.

 

Sign indications however should not be weighted too heavily unless reinforced by other factors. Planetary positions are more important than the nature of signs in determining health. Any Ascendant can have good or bad health depending upon favorable or unfavorable planetary positions.

 

\subsection{Dashas and Transits}
 

The health of a person is indicated by the dasha, bhukti and antardasha (major, minor and subminor planetary periods) that occur at birth. Please note this down whenever you look at a chart.
 

This period if weak causes poor health, particularly if the chart is difficult in this regard. If strong it gives good health. In this regard both the temporal and natural status of the planets must be considered. Birth in the Dasha of the lord of the sixth or eighth can be bad for health, even if that planet is a natural benefic. Dashas always reflect the temporal status of planets.

 

Health problems manifest in related dashas of disease-causing malefics or weak benefics.

 

Dashas of natural malefics, particularly Saturn and Rahu, can cause disease regardless of the Ascendant. Saturn weakens the vitality and Rahu disturbs the immune system and natural equilibrium.
 

Dashas of the sixth and eighth lords can give disease because these are disease causing houses. Dashas of the twelfth lord can show time in hospitals, convalescence or periods of low energy.

 

Dashas of the Ascendant lord, if afflicted, give disease and can be very difficult, showing the disease affecting the body and vitality of the person. Dashas of Maraka planets (planets ruling or located in houses 2 and 7) give disease, if the chart is weak in regard to health or if the person is elderly.

 

Dashas of debilitated or retrograde planets can cause health problems during their periods. Combined dashas are generally worse, like a bhukti of the eighth lord in the dasha of the sixth lord. Difficult annual charts (varshaphal) and transits can cause yet more problems.

 

Similarly the development of Dashas shows either recovery or worsening of disease. If the Dashas get worse from the time of the onset of disease recovery will not be good. If the Dashas get better recovery will likely occur. If a disease occurs at the beginning of a bad Dasha, like that of Saturn or the sixth lord, recovery will be difficult.

 

Difficult transits affect the heath, like Saturns transit of the Moon or the Ascendant, which can weaken the health. Rahu-Ketu transiting the Ascendant are similarly difficult for physical and psychological health.
 

The transits of the sixth or eighth lords to the Ascendant can also be difficult. So can transits of the Ascendant lord to the sixth or eighth houses. The Ascendant lord transited by malefics (natural and temporal) can also be difficult. Conversely transits of benefics like Jupiter to these same factors can give good health or nullify the effects of malefic transits. Again look at the yearly chart or solar return (varshaphal) for transit indications for the year.

 

\subsection{Special Factors}
 

\subsubsection{Retrograde Planets}

Retrograde is often  a source of weakness in regard to health, whether it occurs in the case of benefics or malefics. Retrograde benefics in angles will not be much help. Retrograde malefics will cause more harm. Retrograde Ascendant lord can cause health problems.

 

\subsubsection{Rahu-Ketu Axis}

The Rahu-Ketu axis can be a source of weak health implications for the planets located along it, particularly lords of the first, sixth and eighth houses or if occurring in these houses. Association with Rahu is a greater factor of harm than association with Ketu because Rahu weakens the immune system. Jupiter with Ketu can be good for health, even in the sixth house, and can make one a doctor.

 

\subsection{Nature of Health Problems}
 

The nature of health problems  will be largely determined by the planets and houses involved in health afflictions. For example, if the fifth house, its lord and the Sun are involved, then heart problems are likely. Afflictions to the Ascendant show a weak constitution that can manifest in any number of health problems. The health problems will vary by Dashas but be connected to an overly health weakness. A strong Ascendant but afflictions to a particular house will cause problems to that part of the body only.

 

For this factor we look at the usual areas governed by each house:

 
\begin{enumerate}
\item First house – body as a whole, head, brain
\item Second house – face, neck, throat
\item Third house – arms, lungs
\item Fourth house – heart, stomach, emotions
\item Fifth house – heart, small intestine, reproductive system, brain, intellect
\item Sixth house – immune system
\item Seventh house – kidneys, reproductive system
\item Eight house – reproductive system, immune system
\item Ninth house – thighs
\item Tenth house- knees
\item Eleventh house – lower legs
\item Twelfth house – feet, plasma, sugar metabolism
\end{enumerate}

If the same numbers house and sign from the Ascendant and Moon is afflicted disease in that part of the body is likely to occur. For example, if Saturn aspects the fifth house and fifth lord from the Ascendant and Moon, a woman may not be able to have children.

 

Generally the afflicting planets indicate the Dosha or type of the disease. Afflictions by Saturn, Mercury and Rahu cause Vata diseases. Afflictions by Mars, Ketu and the Sun cause Pitta diseases. Afflictions by or weakness of Venus, Jupiter and the Moon cause Kapha diseases. This is particularly true if the planet rules bad houses in the chart like six and eight. Meanwhile, the afflicted planets and houses indicate the type or site of the disease.

 

\subsection{Psychological Afflictions}
 

Any affliction to the Ascendant and its lord will also affect the personality and psychology of the person. The person may have character problems, egoism, instability or dishonesty. If planetary afflictions affect the fourth and fifth houses and their lords, psychological problems are likely. If planetary afflictions extend from the sixth and eighth to the fourth and fifth houses, then health and psychological problems will be related.

 

People with afflicted Moons and fourth houses will imagine health problems even if they do not have any. Their emotional nature will be troubled and their imagination distorted. If they do have health problems they will overreact to them, make them worse or get morbid about them. People with afflicted fifth houses will make poor judgments that bring them health problems even when not necessary. They are also more likely to choose wrong forms of treatment. Afflictions to Mercury will impair the mind and senses, even if they do not bring about any physical problems.

 

Strong malefics in angles will distort the psychology of a person because of their affliction of the Ascendant. Weak benefics in angles will make the person psychologically unreliable. Afflictions of the seventh house will harm the mind and heart of a person through relationship problems. One must remember that the seventh is the fourth house from the fourth and has a similar association with the heart. Afflictions to the eighth house will weaken the character of a person, which will also impair the psychology. Hence many different things in the chart have their psychological counterparts because everything that happens to us outwardly has some inner correlation.

 

\subsection{Ayurvedic Constitutional Strength}
 

A good general rule is:

If the chart shows a good constitutional strength, the dominant planet will show the Ayurvedic constitution.
 

A strong Mars as Ascendant or sixth lord generally creates a Pitta constitution, for example. These factors we have already discussed both in the course and relative to the Astrology of the Seers, so they will not be repeated here.

 

Generally a Jupiter dominant chart is best for health and gives a strong Kapha constitution. A Venus dominant chart is good for health and gives a moderate Kapha or Kapha-Vata constitution but a tendency to self-indulgence. A Moon dominant chart is mixed for health and can give a weak Kapha constitution and emotional fluctuations.

 

A Sun dominant chart is good for health and gives a strong Pitta constitution. A Mars dominant chart is mixed for health and can give Pitta disorders or accidents. A Ketu dominant chart can be difficult for health showing a Pitta-Vata constitution.

 

A Mercury dominant chart is usually good for health but can give nervous and mental sensitivity. A Saturn afflicted chart can be bad for health and show Vata disorders and chronic diseases. On the other hand, an unafflicted and strong Saturn can give good health and strong resistance to disease. A Rahu dominant chart can show Vata or mental problems.

 

If the chart shows a weak constitution, the Ayurvedic aspect of constitution can be revealed by the nature of the afflictions, that is by the planetary influences relative to the sixth house and, to some extent, to the eighth house.

 

Sun, Mars and Ketu predominant afflictions cause Pitta problems and a weak Pitta constitution. Saturn, Rahu and Mercury predominant afflictions cause Vata problems and a weak Vata constitution. Weak Jupiter, Venus and Moon cause Kapha problems and a weak Kapha constitution.

 

\subsection{Health Complications}
 

Some people are healthy throughout their lives but may come down with a serious disease quickly or die suddenly, sometimes at an early age. If this occurs at an early age there will be a major chart affliction. If it occurs at a later age the chart affliction will be severe for the Dasha operative.

 

Some people have good health but suffer from injuries or accidents. The indications for this are not much different than those for disease. Mars should be involved and the sixth and eighth lords. For vehicular injuries the fourth house and lord should be involved, like the fourth lord located in the sixth or eighth houses.

 

Poor health combinations are likely in any chart because we all get sick and die in the natural course of time. These must be severe for a severe disease to occur or for an early onset in life. Never look at the factors that weaken the health only. Always consider the factors that strengthen it as well.

 

See if there are benefics in angles to fortify the health. Note if there are malefics in upachayas to fortify the health. See if there are benefic aspects to the Ascendant and its lord.  See if the person is born during a good Dasha and Bhukti. If we are looking only for disease-causing combinations we will find them in any chart. We must look to health-giving combinations as well so that we keep a balanced judgment.

 

\subsection{Psychological Aspects of Vedic Astrology}


The following section will examine the psychology of the chart in greater detail. The entire chart has ramifications relative to the psychology of the person, which after all, is how we feel about ourselves and our lives. The chart as a whole is like a picture of the subtle body, which is mainly the mind. The following section provides a summary of that information

\begin{center}
\textbf{Audio Overview of Vedic Astrology and Psychology: Connections with Ayurveda, Vedanta, Yoga and Vedic Counseling
By Dr. David Frawley\\
Audio Player}
\end{center}


\subsubsection{Key Factors of Psychology}


\begin{enumerate}
\item The Ascendant and its lord for the basic self-expression and presentation of the person. We can classify people Ascendant types and Ascendant lord types (by the house in which the Ascendant lord is located). The first is the house of the ego (ahamkara) and the I am the body idea or self-image upon which the ego is based.
\item The Moon for the basic emotionality of the person, including Moon sign, aspects to the Moon and, especially, the Moon Nakshatra and the relationship between the Moon and its Nakshatra lord.
Mercury for basic mentality, communication and intellect.
\item The fourth house along with the Moon and Mercury for the emotionality of the person and personal happiness. The fourth house is the house of the home, mother, heart and emotional nature (manas).
\item The fifth house for Buddhi, along with Jupiter and Mercury for judgment, discrimination and intelligence, including creativity.
\item The ninth house, along with Jupiter and the Sun, for the Self or Soul (Atman).
\end{enumerate} 

\subsubsection{Additional Considerations}

\begin{enumerate}
\item The seventh house as a measure of relationship fulfillment, as part of the one-seven axis or self-definition in life, and as the fourth house from the fourth as another house of the mind.
\item The second house as speech and communication.
Planets in duhsthanas generally, houses six, eight and twelve as causing difficulty and unhappiness.
\item The same Navamsha positions both in themselves and relative to the Rashi. The Moon Navamsha. Atma Karaka in Navamsha.
\end{enumerate} 
 
\subsubsection{Psychological Type Considerations}
 

There are a number of psychological typologies possible in Vedic astrology. In fact astrology is where the psychology is best revealed. A good astrologer can explain these in detail.

\begin{enumerate}
\item Guna type: The gunas of person as Sattva, Rajas and Tamas.

\begin{center}
\begin{tabular}{ |c|c| } 
 \hline
Sattva & Sun, Moon, Jupiter \\ 
Rajas & Mercury, Venus \\ 
Tamas & Mars, Saturn, Rahu, Ketu \\ 
 \hline
\end{tabular}
\end{center}
 

Influences of trine lords, benefics in kendras and trines, malefics in upachayas

 

\item Doshic type: The doshas of a person as Vata, Pitta, and Kapha.

\begin{center}
\begin{tabular}{ |c|c| } 
 \hline
Vata & Mercury, Saturn, Rahu \\ 
Pitta & Sun, Mars, Ketu \\ 
Kapha & Moon, Venus, Jupiter \\ 
 \hline
\end{tabular}
\end{center}


Influences of these planets to Ascendant and its lord, nature of Rashis, Navamsha and Nakshatra


 

\item Planetary type (9): The predominant planetary type based upon the dominant planet in the chart.

Most important astrologically speaking.

 

\item Ascendant type by Ascendant Sign (12).

Most important for outer expression in life.

 

\item Moon sign type by Moon sign (12).

Most important for emotional nature.

 

\item Nakshatra type by Moon Nakshatra (27).

Most important for spiritual nature.

 

\item Sun type by Sun sign (12).
    Most important for character and vitality.

\end{enumerate}

\subsubsection{Factors of Disturbed Psychology}
 

The following information summarizes the main points of psychological disturbances which can arise from a number of sources.

\begin{enumerate}

\item Malefic influences to the Moon, particularly if more than one: Mars, Saturn, Rahu, Ketu. Calculated by position, aspect and yoga. The Sun also comes into play here particularly with the waning or new Moon.

 

Each planetary influence on the Moon  has its own nature but typically the result will be anger (Mars), depression (Saturn), psychic vulnerability (Rahu), psychic sensitivity Ketu).

 

Combined Mars-Saturn influences to the Moon destroy emotional happiness and can make a person hard hearted but can lead to spirituality. Moon-Rahu conjunctions cause severe psychic disturbances, particular if aspected by Saturn as well. Yet can give psychic and healing abilities.

 

\item Malefic aspects to Mercury, particularly if more than one: Mars, Saturn, Rahu, Ketu.

 

These will result in wrong judgment and poor communication, as well as stress to the nervous system. While not as difficult as aspects to the Moon they have their results.

 

Combined Mars-Saturn influences for Mercury can disturb mental happiness, as it destroys emotional happiness for the Moon. Mercury-Rahu causes wrong communication leading to deception.

 

\item Malefic influences on the first house.

Sensitive Ascendants for the mind: Gemini, Virgo, and Pisces. Disturbed one-seven axis, including afflictions to the first house and/or the seventh house.

 

\item Malefic influences on the fourth house, particularly if more than one, similar in effect as malefic aspects on the Moon.

 

\item Malefic influences on the fifth house, particularly if more than one.

Mars – recklessness, bad judgment, argumentative nature.

Saturn – dullness, depression.

Rahu – confusion, imagination, Ketu – myopic judgment, fear.

\end{enumerate} 

\subsubsection{Timing of Psychological Issues Via the Dashas – Sensitive Dashas for the Mind}

 

The following Dashas are more like to cause psychological and emotional disturbances.

\begin{enumerate}

\item Moon Dasha and malefic subperiods

Particularly Moon-Rahu, Moon-Saturn, Moon-Ketu, Moon-Mars, Moon-Mercury.

 

\item Mercury Dasha

Particularly Mercury-Rahu.

 

\item Combined Moon-Mercury periods, particularly when afflicted.

 

\item Rahu Dasha

Particularly Rahu-Moon, Rahu-Saturn, Rahu-Mars, Rahu-Mercury, Rahu-fourth lord, Rahu-fifth lord

 

\item Saturn Dasha

Generally depressing. Saturn-Moon, Saturn-Rahu.

 

\item Period of the lord of the fourth house, if afflicted or in bad subperiods, for emotional disturbances.

 

\item Lord of the fifth, if afflicted or in bad subperiods, for wrong judgment.

 

\item Lord of the Ascendant, if weak or afflicted, for overall and deep seated health problems.

\item Any combination of two malefics influencing the fourth or fifth houses will be particularly bad.

 

\item  Dashas of Lords of Duhsthanas or planets in Duhsthanas.

 
\end{enumerate}
 

\subsubsection{Difficult Transits}

 

The following transits will increase psychological problems.
 
\begin{enumerate}

\item Malefic transits to natal Moon

By Saturn, Rahu, Ketu, Mars

 

\item Malefic transits to natal Mercury

By Saturn, Rahu, Ketu, Mars

 

\item Malefic transits or transit aspects to the first, fourth and fifth houses:

Ascendant – stress on life-expression, ego, self-image

Fourth – emotional stress and upset

Fifth – mental stress, worry and anxiety

\end{enumerate}
