\section{THE QUESTION OF AYANAMSHA}
 

Here we will examine a few technical background issues of Vedic Astrology mainly the Ayanamsha issue, more from an astronomical angle. This is purely reference material for those interested in these important details.

 

\subsection{The Name Vedic Astrology}

 

There are some individuals today who do not like the term Vedic Astrology because they do not think that this system of astrology can be found entirely in the ancient VEDAS, which are, after all, a series of symbolic texts on various religious subjects. They would prefer to call the system “Hindu Astrology,” “Indian astrology,” or “Jyotish.” Such views are misinformed for the following reasons.

 

All classical books on Vedic Astrology call the system “Vedanga Jyotish,” Jyotish or astrology/astronomy that is the limb, anga, of the VEDAS. Even a text like YAVANA JATAKA, thought to have a Greek influence, traces its origin back to the VEDAS. Therefore the term Vedic Astrology is the name that this tradition has chosen to give itself, the classical name of the system. Whether the entire system derives from the VEDAS, or whether it is part of a tradition that began with the VEDAS, but incorporated information from other sources as well, is a matter of debate.

 

To call the system “Indian Astrology” has no basis in the tradition, which never defined itself in geographical terms. To call it simply “Jyotish” is also inappropriate because Jyotish simply means astrology/astronomy in Sanskrit and was always referred to as Vedanga Jyotish. To call the system “Hindu Astrology” is not wrong in that Hinduism generally bases itself on the VEDAS. But this system of astrology can be used by non-Hindus and does not require a religious belief to follow, though it does require a spiritual orientation to fully benefit from. Therefore the use of the original name as Vedanga Jyotish or Vedic Astrology appears most appropriate.

 

\subsection{The Question of Ayanamsha}
 



 

Several different Ayanamshas exist, which can vary from each other by a few degrees. This at first appearances is a small distinction but it can make a difference. Yet though the Ayanamsha issue is crucial, few Vedic astrologers know the basis for these different Ayanamshas. The different Ayanamshas proposed reflect astronomical considerations as well as modifications based upon the experience of different astrologers. Here we will examine these issues to show the main points of consideration, so that readers can have an informed opinion about all these issues. Our emphasis will be the primary Ayanamshas used in India in recent times.

 

\subsubsection{Traditional Calculation of the Ayanamsha: The Revati Zodiac}

 

The calculation of Ayanamsha has its basis in astronomy. The Vedic zodiac traditionally begins with the first point of the constellation Aries, which is said to coincide with the first point of the Nakshatra Ashwini. A star also called Revati traditionally marks the beginning of the Nakshatra Ashwini, which is the end of the Nakshatra Revati.

 

The primary astronomical Ayanamsha used in Vedic Astrology therefore is the “Revati Ayanamsha” or the “Revati Paksha Ayanamsha” as it is called. This calculates the beginning of the zodiac from the star Revati and marks the divergence of the point of the vernal equinox from it. Revati Ayanamsha is only Ayanamsha found in all traditional texts of Vedic astronomy, including the SURYA SIDDHANTA, the most important of these. Revati Ayanamsha therefore is the original Vedic Ayanamsha.

 

SURYA SIDDHANTA makes the star Revati mark 359° 50 or 10 minutes west of the beginning of the zodiac. Nearly all the other major Vedic astronomical texts mark it at 360° 00 or right at the exact beginning of the zodiac. It is also placed exactly on the ecliptic at 00° 00 (note page 321 of the E. Burgess and W.D. Whitney translation of the SURYA SIDDHANTA). So calculation of the Ayanamshas should be a simple matter of subtracting the difference between the current place of the vernal equinox and the position of the star Revati. Unfortunately the matter is not so easy.

 

The identity of the star Revati is not definite. The region of the sky at end of Pisces and beginning of Aries contains no bright stars. There is a dispute as to which of these dim stars corresponds to the star Revati or the beginning of Aries. Generally Zeta Aries was used, which is a dim fifth magnitude star. However other stars in the region have also been proposed for this role (Burgess translation page 158).

 

The result is that there are two different types of simple Revati Ayanamshas. Those that use Zeta Aries have an Ayanamsha for 1950 of around 19° 30, like Usha-Shashi. This can be called the “lesser Revati Ayanamsha,” because the Ayanamsha is smaller than any others used. It was very popular in South India up to a few decades ago and is still used by many astrologers there. It appears to have been the main Ayanamsha used in South India in the nineteenth century, if not previous centuries.

 

A second type of Revati Ayanamsha exists, like those used by B.V. Raman and Sri Yukteswar, an Ayanamsha of about 21°45 for 1950, or about 2°15 more than the Zeta Aries Ayanamsha. These we could call a “greater Revati Ayanamsha” because it uses a larger Ayanamsha, or it could be called a “reformed Revati Ayanamsha.” These do not use Zeta Aries, but some other star or a modification based upon experience.

 

In any case, any Ayanamsha, however it is calculated, must always ultimately be a Revati Ayanamsha because the Ayanamsha is always referring to the difference between the beginning of Aries (Revati) and the point of the vernal equinox. However some Ayanamshas calculate this difference not with reference to the star Revati but relative to some other star in the zodiac.

 

\subsection{Chitra Paksha Ayanamsha}

 

Since the identity of the star Revati is in doubt there has been an effort to mark the Vedic zodiac from the standpoint of a star that is brighter and whose identity is clear. In this regard traditional books on Vedic astronomy list the positions of the main stars in all the Nakshatras. Of these different stars the most common one used to mark the zodiac instead of Revati is the star Chitra. Chitra, Alpha Virgo, is a first magnitude star, so there is no doubt as to its location and it is easy to calculate.

 

The second type of astronomical Ayanamsha commonly used in India is therefore the “Chitra Paksha.” This calculates the beginning of the zodiac as 180 degrees opposite the star Chitra, which is regarded as corresponding to the beginning of the sign Libra and occurs in the middle of the Nakshatra Chitra that extends from 23 20 Virgo to 06 40 Libra. Votaries of this Ayanamsha consider that the beginning of Ashwini (and therefore the star Revati as well) is exactly 180 degrees away from the star Chitra. This is the basis of Lahiri, Krishnamurti and other Chitra Paksha Ayanamshas. It comes to an Ayanamsha of about 23° 09 for 1950.

 

In this regard the position of Chitra is given as exactly 180 degrees of polar longitude in SURYA SIDDHANTA, providing a traditional basis for this calculation. However, it should be noted that the great majority of other Vedic astronomical texts give Chitra a longitude of 183 degrees, not placing it exactly opposite the beginning of the zodiac (note Burgess translation pages 321 and 334). These calculations probably refer to Zeta Aries as the star Revati, which is located about 183 degrees from Chitra.

 

The SURYA SIDDHANTA position is the traditional basis of the Chitra Paksha Ayanamsha. The Lahiri Ayanamsha is based upon a calculation of the beginning of the zodiac as 180 degrees opposite Chitra. This then would appear to solve the matter of the Ayanamsha simple and easily. Chitra is exactly opposite Revati so that if we subtract exactly 180 degrees from its position, we can easily find the correct Ayanamsha.

 

However another problem arises at this point. Traditional Vedic astronomy does not calculate zodiacal positions in the same way as modern astronomy. Traditional Vedic astronomy used “polar longitudes,” those based on a line between any particular star, the zodiac and the point of the pole (note Burgess translation 319-320 for more information on this method). This different calculation can make for anything from a slight or significant difference in star positions relative to whether the star is closer to the ecliptic (zodiacal belt) or farther away from it. The greater a star is positioned in terms of polar latitude, the greater will be the deviation of its polar longitude position from that of modern astronomical calculations. Hence we cannot simply substitute Vedic star positions for those used modern astronomy. For example, the star Svati, which Vedic astronomy gives a polar latitude of 37 degrees, a position very far north, has a polar longitude of 198-199 degrees, but in terms of modern astronomy its position is 183-184 degrees (Burgess translation 335), a whopping difference of 15 degrees in position!

 

The star Revati was chosen because it was directly on the ecliptic and has a longitude of nearly 360 00. Hence there will be no deviation in its position however it is calculated. The star Chitra is close to the ecliptic but not exactly on it. It has a polar latitude of 2° 00. This causes some difference in its position if it is calculated relative to polar longitude. The position of Chitra as 180 00 in polar longitude translates into a position 180° 48 in terms of modern astronomy (Burgess translation 334), or a difference of 48.

 

In other words, if we use polar longitude for Chitra as in Vedic texts, the Chitra Ayanamsha would be about 48 minutes less than what is used today under the Lahiri Ayanamsha. This is a point that  B.V. Raman has recently brought out. However no one so far has brought out the idea of polar latitude Chitra Ayanamsha. This would be around 22 20 for 1950 or about half way between Lahiri and Raman Ayanamshas.

 

Since no classical Vedic Astrology texts define the zodiac relative to Chitra, but always relative to Revati, clearly Chitra Ayanamsha is a development from the Revati Ayanamsha, aligning it to a brighter star on the other side of the zodiac. We could call it therefore an “indirect” Revati Ayanamsha.

 

Some Vedic Astrology charts of earlier this century listed both the Revati and Chitra Ayanamshas, showing that the employment of both Ayanamshas has long been an issue and these two are the most important Ayanamshas. So essentially there are two main types of astronomical Ayanamshas in traditional Vedic Astrology, the Revati and Chitra Pakshas, with the Revati Ayanamsha itself being of main two types. All the other Indian Ayanamshas are more or less variations on these, different ways of calculating them, or modifications of them according to experience.

 

\subsection{Western Ayanamshas}

 

There are additional Ayanamshas that have come from the West. These are generally greater than the Lahiri Ayanamsha. Some reflect the Western idea that we are close to the point of 0 Pisces for the vernal equinox. These Ayanamshas are generally based on other grounds than either Revati Paksha or Chitra Paksha. I am not certain of their possible astronomical basis. Those who think that we are already in the age of Aquarius would make the Ayanamsha already more than 30 degrees. However there is one of them, the Fagan-Bradley Ayanamsha, which is another Chitra Ayanamsha but calculates the position of Chitra as somewhat more than Lahiri, making its position correspond to about 29 06 of Virgo, not 00 00 of Libra.

 

\subsection{Controversies of Ayanamsha}
 

The main dispute between Ayanamshas is which Ayanamsha works best in actual experience. However we do find that astrologers using different Ayanamshas have both their successes and failures. Generally more Vedic astrologers think that the Lahiri Ayanamsha works better. It is also the official Ayanamsha that the government of India decided to use. Yet Chitra Paksha Ayanamsha does have its astronomical controversy. Why is Chitra Paksha Ayanamsha not mentioned in traditional Vedic astronomical texts? And if we calculate it according to traditional methods we should use polar longitude, which would make it 48 minutes less than the current figure. Of course, one could argue that the traditional method of calculation was in error or that it was only an approximation to be modified by experience. In this regard there are many methods of calculation in Vedic astronomical texts that are only approximations to be modified by experience. One need not take their statements as final or absolute.

 

\subsection{The Rate of Precession}

 

Not only is the point at which the zodiac is supposed to begin a matter of controversy, so is the rate of the precession. Modern astronomical observations reveal that it is about 50.3 seconds per year, but it appears to fluctuate over a long period of time and appears to be increasing in its rate. However modern observations of the precession are at best only a few centuries old. What it was thousands of years ago could have been different. SURYA SIDDHANTA uses 54 seconds (Burgess translation page 244), a slightly larger figure, which would put the precessional cycle at closer to 24,000 years. This is used by Sri Yukteswar and some other traditional Vedic astrologers, but most other Vedic astrologers subscribe to the scientifically observed rate.

 

\subsection{Are There True and Mean Rates of Precessional Motion?}

 

In astronomy there are variations between various true and mean rates of motion. For example, the mean rate of motion for the Moon is about 13 degrees per day but it can be as little as 11 or as much as 15. Whether there are true and mean rates for the Ayanamsha is another possible complication. In fact all celestial motions are subject to fluctuations. While these are generally minor even a small amount can make a difference over time or in subtle calculations.

 

There also are average (ideal) and actual cycles. For example, there is the ideal year of 360 days, the solar year of 365 days, and the lunar year of 354 days (12 lunar months). If one asks which is the correct length of the year, it depends upon how one is defining the year. So it is possible that there may be true and mean Ayanamshas that may vary slightly in their positions, like the true and mean nodes of the Moon. If this is true it may be possible for more than one Ayanamsha to be correct, though the variations are bound to be slight.

 

The Sri Yukteswar Ayanamsha has a strong spiritual basis in the work of the great guru. His yugic time cycles work very good historically. His work also most reflects the statements of Vedic astronomy like Surya Siddhanta. However his Ayanamsha rate does not agree with the perceived rate.

 

The Lahiri Ayanamsha appears to give better results relative to the divisional charts, which is why more astrologers use it. However I am not yet certain whether the actual Ayanamsha lies somewhere between the two. In addition the Yukteswar Ayanamsha may reflect a more long term or ideal rate, with Lahiri reflecting a more short term but actual rate.

 

\subsection{Conclusion}

 

Lahiri Ayanamsha is currently most popular in India, but this is an occurrence of only the last few decades. This is mainly according to its ability to give better results according to divisional charts as its proponents say. Yet other Ayanamshas continue to be used. There are examples of great Vedic astrologers and yogis who have used Revati Paksha Ayanamshas, which in previous eras were more popular. And of course a good astrologer with an imperfect Ayanamsha may give a better reading than a bad astrologer with the best Ayanamsha. As there are many complexities and variations in Vedic Astrology, that there are differences among Ayanamshas should not be surprising.

 

 

\subsection{A Few Popular Ayanamshas}

\begin{tabular}{ l l l l}
Calculated For 1950          \\
 \end{tabular}

 

\paragraph{Western Ayanamshas}

\begin{tabular}{ l l l l}
30 00 Age of Aquarius.          \\
27 07 De Luce          \\
  \end{tabular}


\paragraph{Chitra Paksha Ayanamshas}          \\

\begin{tabular}{ l l l l}
24 02 Fagan-Bradley          \\
23 09 Lahiri          \\
23 04 Krishnamurti          \\
22 21 Chitra Paksha Ayanamsha according to Polar Longitude          \\
  \end{tabular}


\paragraph{Greater Revati Paksha Ayanamshas}

\begin{tabular}{ l l l l}
22 12 Sundara Rajan          \\
22 02 J.N. Bhasin          \\
21 46 Yukteswar          \\
21 43 B.V. Raman          \\
  \end{tabular}


\paragraph{Lesser Revati Paksha Ayanamshas}

\begin{tabular}{ l l l l}
19 52 Shill Pond          \\
19 27 Usha-Shashi          \\
 \end{tabular}
