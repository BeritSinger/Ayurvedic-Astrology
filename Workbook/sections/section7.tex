\section{CHART RECTIFICATION}
 

Chart rectification is an important consideration for all charts. Most charts will require some at least slight changes in the birth time, which will likely be evident more for charts we have examined several times already. This means that all astrologers should be familiar with the principles of rectification. We will discuss these here.

 

Rectification first of all consists of methods to find out what the appropriate time of birth for a client may be when the actual time of birth is unknown. It some instances only the day of birth is known. This requires considerable study to determine the right time of birth. In other instances the birth may be known as morning or evening, or early morning or early afternoon, but no specific time indicated. Sometimes we may be relying upon the recollection of a parent or relative that may not always be correct either.

 

Often the birth time is known only to the nearest hour or half-hour. This also requires the proper adjustment. Other times the birth time may be down to the minute but the chart shows the first or last degree of a sign rising. As a difference of a minute or two could change the Ascendant, rectification is necessary even in these instances.

 



 

\subsection{NEED TO ASCERTAIN CORRECTNESS OF BIRTH DATA}

 

Most of the time we must rely on the birth time sent to us by the client as being accurate. Rectification – methods of correcting the birth time – usually only becomes an issue if the birth time is in doubt. Yet much birth data that we receive may be of questionable accuracy. Even birth certificate data, which appears very specific, may not be totally accurate.
 

I have found in several instances that when an astrological reading for a client did not seem accurate that the birth time given was wrong. In some cases this mistake has been verified. For example, I had a client with a Gemini Ascendant that did not appear to suit them at all. Upon further research she discovered that her correct birth time was much later (she had relied on her mothers memory and then checked the hospital where she was born and got a different time). This gave her a Scorpio Ascendant, which fit her quite well.

 

Hence we can never be totally certain with birth data. The more we know a chart the more likely we are to modify the birth time of a person by at least a few minutes.  Generally if it is birth certificate data and is specific in nature, down to the minute, we can figure that it is probably accurate but that it still may be fifteen minutes off. If it is rounded off to the nearest hour, it could be up to half an hour wrong. And in some exceptional instances, the actual birth time may be even further off than this. In instances of difficult birth or caesarian birth the times given are more questionable.

 

Mothers memories are not often that good, because the labor process takes time and the sense of time is impaired by the difficulty of the birth process. The memory of the father or a family or friend present at the time of birth is generally more reliable. There is also the issue of difficult births. These are less likely to be recorded accurately.

 

This means that we cannot automatically assume from a birth certificate that the birth time is accurate enough to guarantee the correct navamsha ascendant. We may need to use the rashi or birth chart ascendent in the navamsha until we are correct. The same is true of the other divisional charts except perhaps the drekkana. It also means that we may be some time off if we go to third or fourth levels of dashes.

 

\subsection{CHART VERIFICATION}

 

For this reason it is important always to try to verify a chart, even if we do not try to rectify it. Verification consists of seeing whether the chart agrees with the person. This is accomplished by asking a few questions of the person based upon what appears in the proposed birthchart: seeing if difficult Dashas and Bhuktis had difficult results, good ones good results and so on.

 

The principles of rectification are equally useful in chart verification. It can be a good idea to list ten or more major events or personality traits based upon the birthchart in order to verify it. Many Vedic astrologers in India do this. This can also be good method to get into the chart. Unfortunately many Americans do not think this additional work is necessary and often want the astrologer to immediately attempt to answer their questions. We must therefore educate our clients as to the importance of the correct birth time and an initial chart verification before starting the usual reading. For some clients this may only take a minute, for some it may take an hour. They must understand the importance of the process in order to appreciate it.

 

Attempting to rectify controversial charts is a good way to learn Vedic Astrology. One has to learn to justify the persons life in terms of the chart. This causes the astrologer to look deeply into planetary positions and combinations.

 

\subsection{METHODS OF RECTIFICATION}
 

Rectification can be one of the most labor-intensive parts of astrological work, as well as one of the most speculative. Its value is reflected in how much better readings become based upon the rectified chart. Yet there may be no unanimity in rectification and different astrologers may have their different opinions about the same chart. However each astrologer should do at least a little work in this area. It makes us much more sensitive to charts and to how changes in planetary positions, particularly Ascendants, can have such strong effects. In short it can be a very good learning tool for astrological study.

 

There are several methods of rectification used by different astrologers. There is, however, no foolproof method of rectification that I know of. Even very well known or famous astrologers rectify charts differently. Hence the whole field of rectification requires caution and is not something that a beginner in astrology needs to worry about. We should generally regard whatever rectification we do as tentative, and the more work we do with a client, the more we will be able to be certain of their right moment of birth.

 

Generally speaking, Vedic Astrology is more helpful for rectification than is Western Astrology. Its planetary periods and planetary yogas give us additional means of rectification. This is particularly true for famous or exceptional people. For example, political leaders will usually have Raja Yogas in their chart. We should examine what Ascendant during the time in question has the best Raja Yogas, and we will probably have their correct Ascendant. This often makes rectification easy up to the nearest hour of birth, but getting it down to the minute is often another matter.

 

\subsubsection{1) MECHANICAL METHODS OF RECTIFICATION}
 

Methods of rectification are of two types. The first are “mechanical methods.” These methods rely on certain birth factors, like times of day favorable for male or female births. In Western Astrology, one method of this type is to use various types of progressed aspects. The advantage of such methods is that they are relatively easy to do. Like any mechanical interpretative methods in astrology, however, they do have their limits. While it can be very helpful to consider them, it is good if we can cross-reference or verify them in some way.

 

Mechanical methods of rectification have certain parameters that must be understood as well. Some are only useful if the birth time is known to the nearest hour. There is no simple mechanical method that I know of for rectifying the birth time of a chart, if only the day is known.

 

There are a number of such methods used in Vedic Astrology but they are too complex a subject to try to deal with here. Most relate to the possibilities of male or female births at different times of the day.

 

\subsubsection{2) INTERPRETIVE METHODS OF RECTIFICATION}
 

The second types are what I call “interpretative methods,” which can be very complex, and it is on these that we will focus in this section. This division is similar to the mechanical and interpretative approaches I have outlined for examining marriage or partnership compatibility in charts. Such methods involve examining the planetary positions at different times during the day – or for whatever time period that is in doubt – and seeing which different pattern makes most sense relative to the life-experience of the person. In short, these methods require an examination of the entire chart of a person, or at least those factors that are most strong at any given time during the day of birth. Sometimes this can be easy if the planetary positions are very powerful in one way or another, but other times it can be very complex.

 

The interpretative method consists of gradually narrowing down the parameters until we have a birth time that we can have confidence in. Usually this begins with determining the Ascendant, then the Decanate within that, and then the Navamsha within that, if this is possible. Planetary Periods are useful in this regard as well.

 

\subsection{INFORMATION NECESSARY FOR RECTIFICATION}

 

For rectification, it is helpful to have any important dates for a person (particularly for cross-referencing relative to Planetary Periods). These include dates of marriage, divorce, birth of children, major moves, major career changes, diseases, accidents, dates of loss of parents or relatives, or other important life-events.
General information on character, health (including Ayurvedic constitution if possible), relationship, career, spirituality and so on is helpful – anything that may indicate a very specific thing to look for, a typical pattern of ease or difficulty in these areas.
 

For example, much difficulty in relationship, like a person who never married, can be helpful in rectification for it would cause us to look for afflictions in that part of the chart.

 

We should note that it is much easier to rectify a chart for someone we know. It is not possible to rectify the chart for someone we do not know except by mechanical methods unless they provide us much data about their lives. To rectify a chart therefore requires that we know much more about a person than if their birth time is accurately known. In general we want to try to cross reference as much relevant information as possible.

 

It is easier to rectify the charts of older people because there are more life-events to consider. For a child or young person who has had very few major events in life and whose pattern of character and destiny has not emerged, rectification becomes more difficult but can still be done.

 

Generally it is not too difficult to rectify a chart if the hour of birth is known. This involves mainly establishing the correct Ascendant. This allows enough information  for an accurate initial astrological reading. If the birth time is known to a few hours, rectification is more difficult but it is still usually quite possible to at least determine the correct Ascendant. However, it can be very difficult to rectify a chart if only the day is known. Some astrologers, therefore, will not attempt such rectifications. It is always best to test out a trial rectification and put it forth as absolutely accurate.

 

\subsection{INTERPRETATIVE METHODS OF RECTIFICATION CONSIDERATIONS}
 

\subsubsection{PARAMETERS}

 

First it is necessary to establish the time period parameters in which one wishes to look. If the birth is known only by day, one must then look at the whole day. If the birth is known to be in the morning or in the evening, then we can look at that half of the day. If the birth is known to be late morning or late afternoon, we can look into that quarter of the day. If the birth time is known to the nearest hour, we can look to that one-hour period. If it is known to the nearest half-hour, we can look to that half-hour period etc.

 

Naturally it is important to narrow down these parameters as much as possible. Even to reduce them from the whole day, to one half of the day, helps a great deal. So even if the time of birth is not known, if we can find out whether it was morning or evening, that will be of much benefit.

 

\subsubsection{ASCENDANT}

 

The most important thing is to determine which Ascendant is most characteristic for the person. Once the Ascendant is known we can give a very helpful reading, even if the birth time is not specifically determined.

 

There are two ways to determine what a persons Ascendant may be. The first is simply the “examination of Ascendants,” which is to see which Ascendant best represents the qualities and characteristics of a person. The second way, which is more significant, is to examine the weight of planetary influences from each Ascendant and use this rather than the nature of the Ascendant itself to determine which is correct.

 

Prior to either method, however, we should note the possible Ascendants we have to deal with. If the birth time is only known by day, we may have all twelve as possibilities. If the birth time is known to within an hour, we may two or only one (if the Ascendant is already known, we can proceed with more specific rectification factors).

 

\subsubsubsection{Examination of Ascendants and their Characteristics}

 

As the Ascendant represents our basic way of expressing ourselves in life, the most evident expression of the person is the most important thing to judge. We can first narrow the Ascendant down by sign qualities as expressing earth, water, fire or air signs, or as cardinal, fixed or mutable, and then looking to see if we can identify the particular Ascendant.

 

We can look for the drama, warmth and aggressive nature of fiery signs; the emotionality, sensitivity and personability of water signs; the nervousness, expressiveness or intellectuality of air signs; and the practicality, groundedness or informative nature of earth signs.

 

We can look for the active and achievement orientation of cardinal signs; the stable, steady and obstinate nature of fixed signs; or the sensitive, doubting and changeable nature of mutable signs.

 

Then we can examine the qualities of the Ascendant more specifically. For example an Aries Ascendant will be very expressive, headstrong, self-centered or self-focused, and usually something of a leader. We can examine the characteristics of each Ascendant to become more familiar with these.

 

\subsubsubsection{Examination of Planetary Influences}

 

This aims not at the general qualities of the signs (whether as Ascendants or according to other houses) but at the specific qualities shown by them on any particular day relative to the planets located in or aspecting them.

 

Ascendants may be strong or weak, or dominated by other planets. An afflicted Leo Ascendant, as with Saturn and Rahu rising and the Sun in Libra, may not exhibit much typical Leo qualities. Or a Cancer Ascendant with Mars in the tenth house in Aries, forming Ruchaka Yoga, will have a strong Mars energy, which will also color the nature of the Cancer Ascendant.

 

In this regard we must not only look at the Ascendant but also at the planetary positions within the Ascendant, including the various house positions, yogas and aspects. This is not to try to see what Ascendant in general is typical for the person, but what Ascendant on the specific day shows the particular factors of life-experience found in the persons life. We call this “specific examination of Ascendants.” This depends on the individual. If the person has poor health since childhood, we may look for an afflicted Ascendant that exhibits such a potential. In this regard we may not so much consider the general qualities of Ascendants.

 

We should look at the planetary positions in operation during the given time period. This is the real key to rectification. Naturally each day is different, and some days have some specific combinations that can make rectification much easier. For example, if there is a Mars-Saturn conjunction in Cancer, we can examine what houses this conjunction might occur in. If we have a chart wherein it might occur in the fifth house for a woman, we could examine her ability to have children (which would obviously be impaired). The following rule thus arises:

 

Do not look just at the Ascendant but at the characteristic life-patterns that emerge according to the house positions of the planets that would occur relative to it for the given day.
 

We should go over the planetary positions from each Ascendant and their effect upon the houses and see whether this agrees with the life and temperament of a person. We should see if there are special combinations from the Ascendant like Raja Yogas, Dhana Yogas, or other special features and use these to clinch our position.

 

\subsubsubsection{Ascendant and Moon Sign}

 

In this way we may come up with one Ascendant that appears very typical for a person. However, this is not always as easy as it sounds, because the Ascendant is only one factor in a chart. If the Moon is stronger, the Moon sign characteristics may be stronger. The Moon sign, however, is usually the same for the entire day. Hence we must bear its qualities in mind. For example, if the Moon is in Aquarius, we will expect much Aquarian influence on the nature, even if the Ascendant is elsewhere.

 

\subsubsection{PLANETARY PERIODS (DASHAS AND BHUKTIS)}

 

The planetary periods are relative to the lunar Nakshatra, and can be used for rectification as a factor within themselves. The Moon stays in each Nakshatra for about one day, which would keep the major period or Dasha in effect at birth the same. It is in regard to the Bhuktis that changes can occur, often within the period of one hour of the day.

 

For rectification purposes relative to the planetary periods, it is best to pick three or four most important life events for the person involved and examine how well they fit into the scheme of planetary periods relative. As Bhuktis can last several years, unless we have several events to cross reference, we may not be that accurate. In this regard we should look to the relevant factors. If, for example, a person lost their father at an early age, we should see if the Dasha and Bhukti involved showed afflicted to the factors that represent the father (the ninth house, its lord and the Sun).

 

Generally I prefer to try to determine the Ascendant first before trying to determine the planetary periods. But the planetary periods can help us determine the Ascendant. The planetary periods do not change much with the change of Ascendant but the lordship of planets does. Jupiters period as the lord of the fifth house, for example, will bring very different results than its period as lord of the sixth. Once the Ascendant is known, the planetary periods are very useful to examine for more fine-tuning. The first thing is to see how the transition between the major planetary periods or Dashas agrees with the life-experience of a person. Then we can examine the Bhuktis.

 

However, we must not forget transits and events as well. Sometimes we may have to examine Saturn or Rahu transits for difficult events in life, for example.

 

\subsubsection{JAIMINI DASHAS}

 

The Dashas of Jaimini, which are introduced in the course, are helpful for rectification because they are sign based and change with the change of Ascendant. For those who know this system they should be used for rectification of all charts.

 

\subsubsection{DECANATE AND NAVAMSHA CHARTS}

 

Once the Ascendant is known, we can also examine the Decanate or third of the Ascendant and see which makes most sense relative to the birth experience of a person.
 

For example, if we have determined the Ascendant to be Libra, we should then look at which Decanate of Libra – Libra, Aquarius or Gemini. In the Gemini Decanate of Libra its communication skills will be stronger, along with its ability to gain knowledge, but its social projection will be weaker, with some degree of instability. In the Aquarius Decanate, its compassion, humanitarianism, or selfless idealism (or fanaticism) will be stronger. In the Libra Decanate, the more cardinal, active, artistic, and socially successful, side will prevail.

 

However, we should also look at the Decanate chart and note the planetary positions within it. If Libra is the Ascendant and Decanate but Saturn and the Moon go into the Gemini Decanate, we would expect some degree of detachment and knowledge orientation relative to the persons work and friendship associations (qualities of the Decanate chart).

 

Once the Decanate is known, we can then look to which of the three Navamsha Ascendants within it makes most sense relative to the person both in terms of its basic qualities and in terms of the house positions from it.
 

Generally, I would put more weight on planetary periods for this purpose, but the periods can be cross-referenced relative to these subtle charts.

 

\subsection{CONCLUSION


Once we have examined these factors, we should be able to set forth a tentative Ascendant and do a reading based on that. We can try to be more specific than this if we wish, but this may take some experience. Hopefully we will find some characteristic or event in the life of a person that can clinch our rectification, but something like this does not always stand out.

 

\subsection{OTHER METHODS OF RECTIFICATION}
 

There are some methods of chart rectification based on Horary astrology. There are also some mechanical methods based upon the timing of birth and likelihood of male or female births, as we have already noted. These are very complex, so that we are just mentioning them in passing as a topic for possible further study on the part of the student.

 

Some astrologers or other psychics use their intuition to determine the birth time of a person. Such methods may be accurate but they are not strictly part of astrology and hence we will not examine them here. Still we can consider them as a factor in rectification.

 

\subsection{ALTERNATIVES TO CHART RECTIFICATION – PRASHNA CHART}

 

We also note that a Horary chart (Prashna or question based reading) can be drawn up to answer questions astrologically for a client whose birth time is not known. This may be easier and more effective than rectification and should always be considered. See Part III of the course for information on this subject. It is particularly good for addressing specific concerns and current issues of clients.

 

Prashna can help us overcome the limitations of inaccurate or unknown birth charts but it remains limited to specific questions or areas of life and does not have the long term relevance of the birth chart.