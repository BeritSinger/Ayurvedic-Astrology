\section{THE ANTIQUITY OF VEDIC ASTROLOGY}
 

I have included in the course this article on the origins of Vedic Astrology because there is such little good information on the subject. Most of the ideas about the origins of astronomy and astrology come from people who have never studied Vedic Astrology or the ancient Vedic scriptures, and who may not even believe in astrology. The following is a Vedic perspective on this issue. This material is for reference purposes. There are no test or study questions based upon it.

 

Over the decades I have been involved in historical research about ancient India, which has extended to over half a dozen published books, a number of conferences and interviews. This material is part of that greater study. It is not all astronomical or astrological in nature but has important ramifications in these fields.

 

The question of the antiquity of astrology is intimately bound up with the history of astronomy and therefore the history of science itself. Up to recent years, and still strong, a Eurocentric view of history, science and therefore also of astrology has prevailed, giving importance to the contributions of Greece, and to a minor extent, Babylonia. This what could be called “Greek bias” largely dismisses the ancient sciences of non-European cultures with little acknowledgement, even though the Greeks themselves honored such eastern traditions as the Hindu and Persian. Hindu and Mayan calendars of 3100 BC have been rejected as imaginary inventions of a later era, as has Chinese calendar of 2600 BC. As we move out of the era of Eurocentric predominance and discover the limitations of materialistic views of humanity – and rediscover the authenticity and profundity of traditional cultures – such ideas of history must be challenged.

 

Those of us who follow astrology, which is an occult science, usually accept the validity of occult sciences which modern science as yet rejects. Occultism teaches us that there have been many civilizations in the ancient world prior to those our history books record and that much of the true and spiritual knowledge of humanity lies in the very cultures that modern scholars have regarded as primitive or unscientific. Therefore we cannot look to a tradition which does not respect occult knowledge for understanding the origins of astrology, perhaps the most important of occult sciences. Native traditions like the Hindu which have maintained a continual usage of astrology from the most ancient times up to the modern age are more likely to give us a correct view of the origin of a science, the continuity of which for them has never been broken.

 

The modern scientific view of astrology would make astrology an invention of late Babylonian and early Greek thought around 500 BC, which makes the ancient Egyptians, Babylonians, and Hindus, famous in antiquity for their occult lore going back tens of thousands of years, to be cultures without astrology. This is not acceptable to those of us who have studied these ancient spiritual and occult traditions and know their vast understanding of both the human psyche and the cosmic mind, which certainly would have brought them in contact with the great truths of astrology as well.

 

\subsection{TRADITIONAL VEDIC VIEW AND MODERN ACADEMICS}

 



 

There are two main views on the antiquity of Vedic Astrology. The traditional Hindu view gives it an antiquity of tens of thousands of years. This is in accord with its idea that there have been many cycles of human civilization on Earth going back hundreds of thousands of years, in a universe undergoing many cycles of creation and destruction in a multibillion year cosmic evolution of intelligent life. The modern view – which often derives from nineteenth century colonial scholars – generally makes Hindu astrology and astronomy a rather late phenomenon, deriving from Greek influences, which entered India starting with the invasion of Alexander around 300 BC. This same view has the Greeks taking over the basic elements of astrology, like the planets and signs, from the earlier Babylonians and Egyptians.

 

Generally it is acknowledged that the Nakshatra (lunar constellation) system of astrology, which is found in the earliest Vedic texts, is of a pre-Greek era in India, though some would still derive it from a previous Babylonian influence (though no clear Babylonian record of such a system exists). It is mainly the natal astrology and the use of the twelve signs that some would derive from the Greeks, as there is little reference to these in earlier Vedic literature.

 

Western occultists, like the Theosophists since H.P. Blavatsky, have accepted the antiquity of sign-based astrology in India. This is also the view of the great spiritual teachers of modern India including Paramahansa Yogananda and his teacher Sri Yukteswar, Sri Aurobindo, and Vivekananda. In fact no great teacher of modern India has argued to the contrary. Of course the astrologers of India like B.V. Raman are also of this view.

 

Unfortunately few ancient works on astrology have survived. This is not surprising as few ancient books of any type have survived. In addition in ancient times astrology, like other occult and spiritual knowledge, was often kept hidden or veiled in a symbolic language, which those not possessing the key (which requires initiation and intuition and not merely logical analysis) may not be able to discover. Much was preserved in oral traditions that were not accessible to outsiders and never written down.

 

The ancient Greeks and Romans were indeed good astrologers, like most ancient people. Unfortunately this knowledge was largely destroyed by the advent of Christianity and little of the tradition has remained. The Hindus honored the Greeks for their astrological knowledge, but that in itself is not enough to derive Hindu astrology from the Greeks. The Greeks similarly honored the Hindus for their occult and spiritual knowledge, and for the greater antiquity of their civilization over the Greeks.

 

Civilization in the ancient world had much communication and naturally ideas went back and forth, as today. There were probably less rigid boundaries between people of different countries, languages and religions than today, and a greater general intermixing of populations. Hence people today trained in ancient Greek thought may indeed find parallels in Hindu thought for which they may see a Greek influence. But those trained in Hindu thought, such as few scholars in the West are today, will find many parallels in Greek thought that appear to have a Hindu origin.

 

Greco-Roman culture is known for its eclectic nature. It borrowed much from the Egyptians, Babylonians, Persians and Hindus, as well as having its own traditions. In Rome we find Egyptian Gods like Osiris and Isis, Persian cults like Mithra, and a mystique for India, with the ultimate triumph of a religion from Palestine! Greece was not an original culture but more peripheral to the larger and older cultures of Mesopotamia and India. Therefore it is unlikely that it contains the origins of the key knowledge of the ancient world, like astrology.

 

There were many contacts of trade and culture between India, Babylonia and Greece in ancient times. If Vedic Astrology derived from a Western source it would not likely have been the Greeks. There is an archaeologically proved on-going trade between India and Mesopotamia going back to at least 2500 BC and the Sumerian-Akkadian eras, or before the time of Babylonia. The people of India would have been exposed to Babylonian science and astrology long before the Greeks and would not have to use the Greeks to bring it to them.

 

In fact the weights and measures used in the island of Bahrain in early ancient times, the main point of trade from Babylonia east, were those of the Indus Valley civilization of India. The same weights and measures were used in later India of the classical era. Hence the peoples of the Middle East must have also been influenced by Hindu thought and science.

 

\subsection{New Light on Ancient India}
 

Sites of the sophisticated urban Indus civilization of India (3100 – 1900 BC), also called the Harappan culture, are spread over a region from Turkestan in the northwest to the Ganges valley in the east. They occur from the border of modern Iran in the west to the Godavari river in south India, with even a site on the Arabian coast. This makes it a civilization so large in extent that both Egypt and Babylonian can be put into it and only cover a small portion of its domain.

 

While the Indus civilization was previously thought to be pre-Vedic, new evidence has caused it to become regarded as Vedic. This is now the view of the great majority of scholars working in the field (like S.R. Rao, S.P. Gupta, Jim Schaffer, Mark Kenoyer, Subhash Kak, Navaratna Rajaram, B.G. Sidharth etc.). The most decisive point is that Vedic culture bases itself upon a large river called the Saraswati that flowed to the ocean in western India west of the Yamuna river. Such a river has been rediscovered by satellite photography and confirmed by ground water studies as having existed in ancient times. As scientific studies show that the Saraswati dried up by 1900 BC, the Vedic literature which speaks of this river must date well before this time.

 

In fact the Indus valley civilization has the great majority of sites located along the dried banks of the Saraswati. Over five hundred located there compared to a mere three dozen on the Indus. Ecological changes that occurred along with the drying up of the Saraswati river brought the Harappan civilization to an end. Because of this the Indus valley civilization has been renamed the “Saraswati civilization.” Such finds confirm Vedic astronomical references to the time in which the vernal equinox was in the Pleiades (Krittika Nakshatra) or a period of around 2500 BC, the Indus era itself, which now appear as probably the most ancient authentic astronomical references in the world.

 

In addition there is no evidence of any Harappan site destroyed by invaders or any significant migrations into India during the period of the so-called Aryan invasion. The latest view is of an organic and indigenous development in civilization in India from the ancient Mehrgarh site of 7000 BC (near the Bolan Pass in Pakistan) through the Indus era, to the classical period of India, with a continuity of the same traditions of culture and science. The previous view, which looked to a Middle Eastern influence for the urban Indus culture, has been discredited. Hence ancient India with its urban civilization must have had its own system of astronomy and astrology as well. Such a civilization gives a good basis for the antiquity of Vedic Astrology.

 

\subsection{Vedic and Western Astrology}

 

Let us examine what Hindu astrology consists of and how much of it has counterparts in Western Astrology. What is surprising is that much of what is characteristic of Vedic Astrology has no real counterpart in Western or Greek astrology.

 

Planets, signs, houses and aspects are used in Vedic as in Western Astrology deriving from the Greeks. The planets and signs have a similar, though not exactly the same, meaning. The planets and signs are described according to traditional Hindu mythology. The houses are similar in their interpretation, but with significant differences both in terms of calculation and interpretation. Aspects are used in a very different way in Vedic than in Greek astrology, with a totally different set of aspects and a different way of calculating and interpreting them.

 

The main difference, however, which divides the two systems is that the Vedic zodiac is determined sidereally rather than tropically as is the astrology derived from the Greeks. This shows an entirely different view of the zodiac and very different calculations necessary to arrive at sign positions.

 

The question is what zodiac is the original, the sidereal (star oriented) or tropical (equinox oriented). We have no evidence that Vedic Astrology was originally tropical. Rather it appears that Greek astrology may have been originally sidereal, as an abstract tropical zodiac must come at a later date than the sidereal zodiac, which can be determined by mere observation of the stars. Hence while Vedic and Greek astrology share certain basic features, there are fundamental differences between them. And beyond these Vedic Astrology has many other factors that it considers.

 

The amount of calculations considered in Vedic Astrology is easily over ten times that of Western Astrology. All these factors can be found in works of writers like Varaha Mihira and Aryabhatta (who also mentions the earth revolving on its axis), who date around 500 AD by the most conservative estimates.

 

Below are some of these special factors of Vedic Astrology, which have little or no counterpart in classical Western Astrology.

 

Vedic Astrology employs a total of sixteen divisional charts, based upon dividing up each sign of thirty by numbers as small as sixty. No more than two or three of these are considered in the Western system.
\begin{enumerate}
\item[*] It has an entire system of planetary periods based upon the lunar constellation (Nakshatra) in which the Moon is located at birth. The Vedic astrologer-sage Parashara mentions fifty-four such Dasha systems, of which the Vimshottari system is the most common. It also has several systems of planetary periods and other calculations based upon various chakras or geometrical and mathematical designs like Kalachakra Dasha, Kurma Chakra etc.
\item[*] It has an elaborate system of determining planetary strengths and weaknesses called Shadbala that considers minute details of planetary positions in different divisional charts, times of the hour, day, month and year, direction, motional strength, etc.
\item[*] It has a complex system of determining favorable planetary rays via signs called Ashtakavarga, which can be used for transits or natal astrology.
\item[*] It does not rely so much upon aspects, as does Western Astrology, but upon various planetary combinations called yogas of which several hundreds exist defined by various factors of aspect, interchange, position and house rulership.
\item[*] It has another system of astrology called “Jaimini astrology,” which employs its own set of aspects and special Dasha systems according to the signs, relying on a unique system of planetary significators.
\item[*] It has its unique system of astrological remedial measures including gems, colors, mantras and rituals, which are rather different and more complex than those used in medieval Western Astrology.
\item[*] It has a very complex and unique system of Horary astrology.
\item[*] It has a complete system of astropalmistry, as well as various related forms of numerology.
 \end{enumerate}

We should note that most of the books of Hindu astrology have not been translated into English. Moreover, what is written in books in India is only a fraction of what exists in oral traditions in the country, which may have never been written down. Hence more factors than these may yet exist to be discovered.

 

\subsection{Tajika Astrology}

 

There are some methods of Western Astrology that were taken into Vedic Astrology, primarily in regard to progressions and solar returns, like those found in the Tajika system. Yet when this was done there was an acknowledgement of a foreign source. Such ideas were developed along Vedic lines using the sidereal zodiac, Nakshatras, and other more uniquely Hindu characteristics. They were additions to the main system and used in a secondary way, rather than forming any core part of it.

 

For example, Western aspects (like the trine or sextile) are spoken of and employed in the Tajika system only. Why were such things employed in a secondary way if the original system of Vedic Astrology came from the Greeks, in which these aspects are central? Why were they only introduced through the Tajika system, which has an apparent Arabian origin, if the Hindus were exposed to them previously via the Greeks?

 

\subsection{Nakshatras}

 

Besides these factors Hindu astrology has a system of lunar astrology based on the Nakshatras or twenty-seven constellations of the moon. Vedic Astrology judges planets not only by their sign but also by their Nakshatra positions, with the Nakshatras regarded as a deeper level of interpretation.

 

There is an entire science of Nakshatras and time that includes the lunar days or tithis (a thirty-fold division of the lunar month of twenty-nine and a half days), a half division of these called karanas, and Sun-Moon relationships also called yogas. There is similarly a thirtyfold division of the day called muhurtas. Such time calculations have no real counterpart in Western Astrology.

While there is some doubt as to the origin of the Nakshatra system, as similar systems can be found in Chinese and Arabic thought, there is no doubt that no group has used them to the extent that the Hindu has. We note that the earliest names of the Nakshatras are according to their Vedic deities. They reflect the oldest language of Hindu thought.

 

Earlier Vedic Astrology and ritual relies mainly on the Nakshatras or 27 lunar constellations. These are found in their entirety in ancient texts like the Yajur Veda (Taittiriya Samhita) and Atharva Veda (XIX.7), which are dated from 1000 BC by the most conservative estimates, and to 2500 BC by more liberal accounts. Nakshatras are also mentioned in the oldest Vedic text the Rig Veda, dated 1500 BC by conservative accounts back to 4000-6000 BC by liberal accounts.

 

\subsection{Nakshatras and Signs}

 

There is no doubt that Nakshatra astrology considerably antedates the proposed Greek influence, and probably the Babylonian as well. Because the signs are not mentioned clearly in such ancient Hindu literature, some thinkers hold that they are not Vedic and are a later invention. We do find references as early as the Rig Veda, the oldest Vedic text, there is reference to a wheel of 360 spokes divided into twelve parts (Rig Veda I.164.11).

 

Moreover a twelvefold division of the zodiac is inherent in the Nakshatra system itself.

 

The twenty-seven Nakshatras coincide  with the twelve signs. The points of 0 Aries, 0 Leo, and 0 Sagittarius divide the 27 Nakshatras into three equal parts, marking the beginning of the Nakshatras Ashvini, Magha and Mula. This can hardly be coincidental and shows whoever did the system had a similar idea as to the main divisional points of the zodiac as the twelve signs.

As early as the Rig Veda the path of the Heavens was called the “Path of Revati” as Revati is the star that traditionally marks the beginning of the Aries. The Ashwins, the deities ruling Ashvini, the first Nakshatra in Aries, were traditionally the first of the deities to be invoked on the Path of Light. Meanwhile Pushan, the deity of Revati, the last of the Nakshatras in Pisces, was always invoked as the guide of the path and the lord of death.

 

Vedic Astrology is sidereal and considers the precession. Today it places the vernal equinox in early Pisces about 6 degrees. In Varaha Mihiras time he places the summer solstice in the beginning of Cancer, a date of circa 400 AD, but notes that previously it was in the middle of Aslesha Nakshatra (22 30 Cancer) or a date of 1400 BC (Varaha Mihira, Brihat Samhita III. 1-2). This is the information found in the early astronomical work Vedanta Jyotish. Late Vedic texts, including some Upanishads, place the vernal equinox in the Pleiades (early Taurus) and the summer solstice in Leo, a period of around 2500 BC (for example, Atharva Veda XIX.7.2). Yet earlier texts place the winter solstice in the month of Phalguni, placing the summer solstice in the beginning of Virgo or c. 4000 BC (ie. Atharva Veda XIV.1.13). Vedic Astrology shows an ongoing tradition of equinoctial changes going back to the earliest historical period.

 

The Maitrayani Upanishad presents some interesting astronomical lore, including mentioning Saturn, Rahu and Ketu. It divides the year into two portions, relative to the northern and southern course of the Sun (from winter to summer solstice and vice versa), as is the Vedic and Hindu custom.

 

Time beginning with the moment to the twelvefold-natured year is the form of the Sun bears. Half of the year belongs to fire and half belongs to water. From the beginning of Magha (0 – 13 20 Leo) to the middle of Sravishta (23 20 Capricorn – 6 40 Aquarius) belongs to fire (Agni). Step by step from the end of Sarpa (16 40 – 30 00 Cancer) to the middle of Sravishta belongs to water.Then by subtlety are the single parts of time, the Navamshas together with the moving ones (the planets). By this measure time (the year) is measured.
Maitrayani Upanishad VI.14

 

Now these Nakshatra points are exactly equal to 0°Leo on one hand and 0° Aquarius on the other hand as summer and winter solstices (which occurred around 1800 BC). Why would such a point be chosen if the twelve signs were not also known?

 

To affirm this the Navamsha or ninefold division is mentioned. The Navamsha division occurs only relative to the signs, being a ninth division of them, and a quarter division of the Nakshatras. Hence the signs, Nakshatras and Navamshas were known by the time of the Maitrayani Upanishad. Here we have a calendar of 1800 BC that shows a knowledge of signs and Navamshas. If this reference occurred in a Greek work it would be hailed as a great scientific achievement. However, in a Hindu work it is simply ignored. Such are the prejudices of the Eurocentric view. Ignoring such references Western scholars date the Maitrayani Upanishad to the early centuries BC. Yet even this suggests that the signs and Navamshas were known long before this period and hence long before the advent of the Greeks in India.

 

Thus knowledge of the Nakshatras does not exclude that of the signs, but rather implies it, as a more precise twenty-sevenfold division of the zodiac, which reflects a different position for the moon every day, implies a more general twelvefold division of the zodiac based upon the months. Originally the moons Nakshatra at birth was noted as this is the necessary basis for Vedic rituals even today.

 

\subsection{Natal Charts}

 

The birthchart of Lord Rama, described purely in the terms of classical Hindu astrology, is given in Valmikis Ramayana, an ancient Hindu text, usually dated to the pre-Christian era. This is the oldest clear reference to the practice of natal astrology in Hindu or, probably, in any other literature. It says that Rama was born “In the twelfth month, in Chaitra, on the ninth tithi, with Punarvasu Nakshatra as the ascendant, with five planets exalted, with the Moon and Jupiter in the ascendant of Cancer” (Valmiki Ramayana I.18.8-9).

 

Not only is natal astrology indicated but in the full traditional Hindu style using both signs and Nakshatras, as well as tithis. It is also given relative to one of the greatest Hindu avatars. It must have been highly respected in order for this to occur, particularly in conservative India wherein foreign influences were shunned. Such a development must have taken some centuries before the time this reference was made.

 

Similar astrological information occurs in the Mahabharata, generally regarded as an older text than the Ramayana. Bhishma is said to be born with the Moon and Jupiter were conjunct in Leo (Adiparva 122.19). Saturn is placed in Bhagas Nakshatra, Purva Phalguni in Leo (13 20 – 26 40), and said to oppress the Nakshatra Rohini, in Taurus (10 00 – 23 20), showing the effect of Saturns full aspect on the sign tenth from it in the Vedic system (Bhishmaparva 2.32 & 3.14). This is interesting because if Saturn is throwing aspects it must be relative to sign.

 

Meanwhile Bhishma dies at the winter solstice during the bright half of the month of Magha, with three-quarters of the month left, or one week before the full moon. The month of Magha occurs when the full moon is in Magha, the center of which is 06 40 Leo (Mahabharata Anusasana Parva 168.26-28), at which time the Sun opposite the Moon would have been at 06 40 Aquarius. A week before the full moon the Sun would have been at 0 Aquarius, or about seven degrees before, the same information as the Maitrayani Upanishad just quoted.

 

We should note that the construction of the natal chart in Hindu astrology is very different from that in Western Astrology. Vedic Astrology uses various square or diagonal forms, quite different than the circular Western chart. Different chart styles can be found in north, south and east India.

 

\subsection{Greek Terms in Vedic Astrology}

 

There are a few apparent Greek terms that can be found in Vedic Astrology like kendra or hora. Yet the number is small, less than one percent of the terms used. This can easily be explained by the presence of Greek rulers in India from the second century BC to the first century AD. The use of a few such Greek terms in Vedic Astrology no more proves that Vedic Astrology had a Greek origin than does the finding of Roman coins in Indian ruins prove they were Roman colonies. In fact, if a Greek origin to Vedic Astrology is admitted the opposite problem of explaining why so few terms in Vedic Astrology have a Greek basis arises.

 

We have noted that when factors of Western Astrology are used in the Vedic system, like in the Tajika system, their foreign origin is acknowledged. Yet there is no mention of Vedic Astrology as having derived from the Greeks. Even Varaha Mihira who praises the Greeks for their astrological knowledge places Parashara, Vasishta and Asura Maya as founders of Vedic Astrology in previous eras long before the Greeks (Brihat Samhita II.32).

 

\subsection{Yavana Jataka}

 

There are some traditional texts which appear to have a foreign influence like Yavana Jataka. Yavana, thought to mean Ionian or Greek, is referred to as proof that Vedic Astrology has a Greek origin. One should note that the Sanskrit term Yavana can refer to any person of Western origin, including Afghans and Arabs, as well as Greeks, and was applied to people in the Puranas and Mahabharata for periods long before the Greeks ever came to India.

 

Yavana Jataka is a relatively minor work on Vedic Astrology of medium length. It does contain some Greek terms. However, if we examine the contents of Yavana Jataka as a whole we find a typical work on Vedic Astrology including Vedic aspects, yogas, special divisional charts (vargas) like saptamsha, apacaya and upacaya houses, planetary friendships and enmity, and Ashtakavarga not found in Greek astrology. It includes Nakshatra astrology with Navamshas (frequently mentioned throughout the work) and Dashas (commonly mentioned, though without reference as to how to calculate them), as well as the special Hindu calendar and time system of, tithis, muhurtas (48 minute divisions of the days) and yugas (time-cycles). In addition are Sanskrit letters and mantras for planets, signs and Navamshas, with the whole thing couched in all the terms of Hindu culture, spirituality and Ayurveda (Hindu medicine) quite unlike any Greek or Western astrological work.

 

The most interesting thing about the book is its ending. The author of Yavana Jataka claims that his knowledge came from the grace of Lord Vishnu and by transmission through Brahma (Prajapati), who received this knowledge from the Vedic Gods, the Ashvins, who are the traditional source of all esoteric Vedic lore including Ayurveda and much of the Upanishads! The book criticizes the Buddhists and other unorthodox ascetics. Hence the author was obviously a Hindu Vaishnava or follower of the Hindu God Vishnu, with a Vedic background. It is known that some of the Greeks in India did become Vaishnavas. The Greek community in India was famous for its astrological knowledge, as noted. But the astrology and culture presented in Yavana Jataka is a Hinduized Greek culture, long removed from Greece – speaking nothing of Greek thinkers like Plato or Aristotle, Greek astrologers like Ptolemy, place names in Greece, kings of Greece like Alexander, or Greek Gods.

 

How such a work could be a translation from the original Greek text from a place like Alexandria, as some Western scholars like Pingree would maintain, is absurd. Are we to assume that all of this, including not only the astrology but the Hindu culture, came from the Greeks, which the Greeks then forgot and, in regard to much of the knowledge, lost all trace of? Whenever the Yavana Jataka was written and by whomever it shows at that time the whole basic system of Hindu astrology, such as is practiced in India today – most of which as we have noted is either non-existent or unimportant in Greek astrology – had already been fully developed. All Yavana Jataka really proves is the antiquity of Vedic Astrology because the different elements of the system could not have been invented or integrated quickly, but must have taken many centuries to develop before the time of the book, which has been dated to the early centuries AD.

 

Hence to make conclusions based upon a few apparent Greek names or terms, or speculation about texts which have not been preserved, makes little sense. Most of the books on Vedic Astrology are named after Vedic sages like Parashara or Garga, and the system is always given a Vedic basis. Dozens of such ancient sages are mentioned in Vedic astrological literature as well as many ancient texts which have been lost in the course of time.

 

Another problem is that we do not have any Greek texts from the Hellenistic world that show these many additional factors of Vedic astrology like dashas, numerous divisional charts, and Nakshatras. Had such factors originated in Greece, then some record of them in that part of the world should remain.

 

\subsection{Greek and Hindu Mathematics}

 

Hindu methods of chart calculation using the time of sunrise and 24 minute divisions of the day (ghatis and vighatis) are quite different from the Greek. Hindu mathematics and astronomical calculation methods are different from the Greeks. For example, the Hindu method of determining the precession is different from the Greeks, as well as its method of determining planetary epicycles. Even its means of calculating the value of Pi or of deriving the Pythagorean theorem are different. How the astrology was transmitted from the Greeks but not the astronomy or mathematics is difficult to prove.

 

Hindu methods of calculating time, as we have noted, are different with a unique calendar employing a luni-solar year, divisions of the day by thirty and sixty and, above all, long cosmic cycles or yugas up to multibillion year universal cycles. This interest in large numbers goes back to the Vedas, where in the Yajur Veda (Shukla Yajur Veda XVII.2) special names are given for all numbers from one, to ten, a hundred, a thousand up to and ultimately up to one followed by 12 zeros or 1,000,000,000,000.

 

Besides the decimal system, the duodecimal system is common in Vedic texts including frequent numbers like 12, 24, 36, 108, 120, 180, 360 and 720. The prominence of 360 in Vedic texts relative to both the calendar and to the heavens suggests the 360 degree zodiac at an early date. In the Rigveda, Vishnu, often identified with the Sun or the pole star, has four times ninety or 360 names.

 

If the Hindus adapted Greek or Babylonian astrology whey did they not take their mathematics or calendar as well?

 

\subsection{Vedic Astrology and Other Vedic Systems}
 

Vedic Astrology is not an isolated system. As part of the Vedas it is interrelated with Ayurveda, Yoga and other systems. As these systems do not have a Greek origin by anyones account, how can their sister science of Jyotish, which employs their same language of the qualities (gunas), elements, and deities, be of a different source. Astrology (Jyotish) is said to be the very eye of the Veda. Hence it could not have been developed later than the rest of the Vedic system?

 

\subsection{Astrology and Karma}

 

Astrology only makes sense if we interpret it according to an understanding of karma and rebirth. If one does not accept these it becomes difficult to understand why a particular person is born with a particular chart. Hence astrology should likely arise only in a culture which accepts the validity of the law of karma. Hindu thought is based upon the law of karma and rebirth. While there is some indication of an acceptance of karma and rebirth among the ancient Greeks and Babylonians it is not clearly formulated by them or universally accepted and may derive from a Hindu influence. The philosophical basis for astrology is thus found more clearly in the Hindu tradition.

 

Yet even if one were to admit a Greek origin for Vedic Astrology insurmountable problems arise. Which part of Vedic Astrology came from the Greeks? The Nakshatra system obviously did not, nor did the Hindu view of the planets. Ashtakavarga, Jaimini, etc. have no real counterpart in the Greeks. Was the Greek addition merely the signs and houses, or the construction of the birthchart? Does this mean that prior to the Greeks the Hindus had no idea of a twelvefold division of the heavens, even though they had twelve months? And why did the Hindus not get this knowledge from the Babylonians at an earlier date? Was the original Greek astrology tropical or sidereal? If the Hindu system derived from the Greek why did it preserve a sidereal basis, which gives it an entirely different orientation? If the Hindus took over the signs and natal astrology from the Greeks why did they discard the aspects and other factors of Greek astrology?

 

If we ascribe a Greek origin for Vedic Astrology we have to admit either that Europe forgot the greater part of its original astrology, which was preserved somehow better in a foreign country that already had its own ancient tradition of Nakshatra astrology. Or we have to admit that on the basis of a few factors from Greek influence (like the signs and the houses) the Hindus immediately built an elaborate and complex system of astrology that went far beyond it. We have to make a smaller system the origin of a much larger one.

 

Whatever the connection between Vedic and Western Astrology may have been historically, it is clear that the two systems, although they share a number of common terms, have a significantly different foundation. One is tropical and the other is sidereal, and they are different in what they develop from this, with the greater part of Vedic Astrology having no real counterpart in the Greek.

 

To deal with this question conclusively there must first be a greater examination of Vedic Astrology, of which our knowledge today is still very scanty. Such study should include its different classical texts as well as its oral traditions, by people today with a background in astrology. Unfortunately very few of us in the West know Sanskrit or have access to traditional Hindu astrological, astronomical and mathematical knowledge.

 

This discussion underscores the need for a new scholarship of astrology at a time when the Eurocentric view of history and civilization is being discarded. As Europe has never been known for being the center of spiritual or occult knowledge for humanity, it may also be that the attribution of astrology to Greece, may just be a another facet of this Eurocentric view. While Babylonia or Egypt would be a more likely candidate than Greece, until we have adequately examined the Hindu system and new archeological finds on ancient India, such a conclusion may be premature.

 

I have dealt with the issues of ancient astronomy and astrology in my books Gods, Sages and Kings: Vedic Secrets of Ancient Civilization (Lotus Press new edition. 2012), and In Search of the Cradle of Civilization (with Feuerstein and Kak, Quest 1995). Those who are interested in more information on this subject can consult the books. I also have many papers on recent archeological finds in India and studies in Hindu astronomy, astrology and mathematics which confirm these views. Most of these are available on-line on the Vedic astrology section of our articles:
\url{https://www.vedanet.com/ayurvedic-astrology/}.

