\section{JAIMINI ASTROLOGY}
 

The Jaimini system is the alternative system to the main Vedic Astrology system, which is the Parashari or system of Parashara. Jaimini, while a subset of Parashara, has an identity of its own. It is a complex system in its own right. It is not commonly used or entirely understood but it is being revived by a number of astrologers in India today. Here we will go into its basic factors for reference purposes. The student will not be tested on the system but should at least have a general idea what it is.

 

Generally the student should first aim at mastering the basic Parashari system of the Bright Parashara Hora Shastra, which is the main focus of what we teach. However, the Jaimini Karakas are important. The Jaimini Chara Dasha is the best system of planetary periods after the Vimshottari. Jaimini astrology is an important study for advanced study and practice.

 

\subsection{Jaimini Karakas}
 

Jaimini Karakas are based upon the planets degrees in any particular sign.

 

\begin{enumerate}
\item[*] 1.    Atma Karaka AK, significator of self – the planet with the highest number of degrees, minutes and seconds in any particular sign.

\item[*] 2.    Amatya Karaka AmK, significator of any important friend or associate – the planet with the second highest number of degrees, minutes and seconds in any particular sign.

\item[*] 3.    Bhratri Karaka BK, significator of brothers and sisters and friends – the planet third in regard to number of degrees, minutes and seconds.

\item[*] 4.    Matri Karaka MK, significator of the mother – the planet fourth in order of degrees, minutes and seconds.

\item[*] 5.    Putra-Pitri Karaka PK, significator of children (and father), the planet fifth in this order.

\item[*] 6.    Jnati Karaka JK, significator of relations like cousins etc, planet sixth in order.

\item[*] 7.    Dara Karaka DK, significator of the marriage partner – the planet with the lowest amount of degrees minutes and seconds in any particular sign
\end{enumerate}
 

These Karakas follow the general meaning of the first seven houses.

 

Atma Karaka – first house, Amatya– second house, Bhratri – third house, Matri – fourth house, Putra – fifth house, Jnati – sixth house, and Dara – seventh house.
 

Yet Amatya Karaka has also a tenth house influence and Putra Karaka as also Pitri Karaka, a ninth house influence. Amatya karaka also refers to a good partner, aide or helper.

 

In the Jaimini system the planets are looked at not only in their natural and temporal statuses, but also by these special significators. Rahu and Ketu are usually not used in this scheme of significators.

 

These Jaimini Karakas are also useful in the regular Parashara astrology. Most important is their position in the Navamsha chart, especially that of the Atma Karaka, which indicates the inner Self or soul of a person. Most Vedic software calculates these Karakas for you.

 



 

\subsection{Jaimini Aspects}
 

Jaimini aspects are by sign only. In this scheme fixed signs aspect cardinal signs except the sign immediately adjacent. Cardinal signs similarly aspect fixed signs except the one adjacent. Mutable signs aspect each other. This gives makes each sign be aspected by three others. As aspects are by sign they include Rahu and Ketu.

 

Table of Jaimini Aspects

 
\begin{center}
\begin{tabular}{ l l l l}
Aries	& Leo, Scorpio, Aquarius                \\


Taurus	 &Cancer, Libra, Capricorn                \\
 

Gemini	 &Virgo, Sagittarius, Pisces                \\
 

Cancer	& Scorpio, Aquarius, Taurus                \\
 

Leo	& Libra, Capricorn, Aries                \\
 

Virgo	 &Sagittarius, Pisces, Gemini                \\
 

Libra	 &Aquarius, Taurus, Leo                \\
 

Scorpio	 &Capricorn, Aries, Cancer                \\
 

Sagittarius	& Pisces, Gemini, Virgo                \\
 

Capricorn	 &Taurus, Leo, Scorpio                \\
 

Aquarius	 &Aries, Cancer, Libra                \\
 

Pisces	 &Gemini, Virgo, Sagittarius                \\
 
 \end{tabular}
\end{center}
 

\subsection{Jaimini Chara Dash}
 

Jaimini has its own Dasha system. Most important is the sign periods (Chara Dashas), which most Vedic Software will calculate for you today. It also has sixteen Navamsha Dashas. 

 

\subsubsection{Rules of Interpretation for Jaimini Chara Dasha}

 

\begin{enumerate}
\item[*] Consider Chara Dasha Rashi, the sign of the Chara Dasha period, as the Ascendant and examine the planetary positions from it as if it were the Ascendant.
\item[*] Consider only Jaimini aspects (Planets in fixed signs aspect those in cardinals, planets in cardinal signs aspect those in fixed, planets in mutable signs aspect each other).
\item[*] Take into consideration the Jaimini Karakas (Atma, Amatya etc.) and see what Karakas are involved.
\item[*] Judge fortune, work and benefit from ninth, tenth and eleventh from the Rashi and their rulers.
 \end{enumerate}

\subsubsection{Jaimini Planetary Strength}
 

Determine the strength of planets in Jaimini using the following rules  Best to use Vedic software to do these calculations for you.

 

\paragraph{1.    Mulatrikona strength}

 
 
\begin{center}
\begin{tabular}{ l l l l}
70	&Exaltation	&60	&Mulatrikona             \\
50	&Own House	&40	&Friend             \\
30	&Neutral	&20	&Enemy             \\
10	&Debilitation	&&	             \\
  \end{tabular}
\end{center}

\paragraph{2. Amsha Strength}

 
 
\begin{center}
\begin{tabular}{ l l l l}
70	&Atma Karaka	&60	Amatya &Karaka             \\
50	&Bhratri Karaka	&40	&Matri Karaka             \\
30	&Putra Karaka	&20	J&nati Karaka             \\
10	&Dara Karaka		&&             \\
  \end{tabular}
\end{center}

 

\paragraph{3.    Kendra Strength}

 
 
\begin{center}
\begin{tabular}{ l l l l}
60	&Angular from Atma Karaka           \\
40	&Succedent from Atma Karaka           \\
20	&Cadent from Atma Karaka           \\
  \end{tabular}
\end{center}

Adding together these three factors the strength of planets can be known. Generally Atma Karaka will be the strongest planet.

 

\subsubsection{Sign Strength}

 

The strength of signs can be determined according to Jaimini also.

 

\paragraph{1.    Fixed or Inherent}

 
 
\begin{center}
\begin{tabular}{ l l l l}
60	&Mutable	&40	&Fixed	&20	&Cardinal           \\
  \end{tabular}
\end{center}

\paragraph{2. Association}
 
\begin{center}
\begin{tabular}{ l l l l}
20	&1 planet (including Rahu and Ketu)           \\
10	for each additional planet           \\
  \end{tabular}
\end{center}

\paragraph{3. Aspect of Mercury, Jupiter and Sign Lord}

60 points – Jaimini aspects only

 

\paragraph{4. Lords Strength according to Jaimini Planetary Strength}

Add together these four factors to determine the strength of the various signs.