\section{OVERVIEW OF VEDIC ASTROLOGY}
 

This lesson is based upon a special feature issue on Hindu/Vedic astrology that Vamadeva did for the magazine Hinduism Today some years ago. It addresses the practical questions about Vedic/Hindu astrology and its applications. It does not deal with the technicalities of reading a chart that we will examine later in the course. It addresses the background, role and usage of Vedic astrology overall, and how to explain its relevance to a general audience and their concerns about its nature, validity and usage. We have recently edited and updated this lesson.

 



\subsection{EDUCATIONAL INSIGHT}
Jyotisha, Hindu/Vedic Astrology

How the Science of Light Can Help You in Daily Life

By Pandit Vamadeva Shastri

 

In the Hindu view, the planets are not mere celestial bodies circling the Sun. They are also divine beings. Each is like a prism, conveying subtle energy from the far galaxies, thus impacting man’s affairs on Earth according to its unique nature and location in the sky. The ancient science of space and time that understands and maps this influence is called jyotisha (literally “science of light”) or Hindu astrology. 

 

It’s About Time

\subsection{AN INTRODUCTION BY HINDUISM TODAY}

Believing nothing, the skeptic is blind; believing everything, the naif is lame. Somewhere between the two lies the lauded land of viveka, discrimination, which neither doubts every inexplicable phenomenon nor swallows every unexamined statement. In this issue we explore the uncanny Vedic technology of jyotisha, that hoary knowledge, derived from secondary Vedic texts, which embraces both astronomy and astrology. It’s about time.

 

President Ronald Reagan confounded the White House staff and embarrassed aides by having his itinerary and major meetings scheduled in consultation with his wife’s astrologer in California. Scoffing staffers counted it pure silliness; others thought it merely impolitic of him, maybe because of the implication that he wasn’t totally in charge or that a Christian would so publicly propound such things.

 

Mr. Reagan is not a lone heretic. Queen Elizabeth I, a Virgo, consulted the stars. Galileo, the Italian mathematician and astronomer, cast charts on the side, as did the German celestial scientist Johannes Kepler. Britain’s Princess Diane followed the stars, and many Hollywood stars do the same. Ditto with Carl Jung and American millionaire J.P. Morgan. A 2013 Harris Poll concluded that 29 percent of Americans (and nearly half of 18- to 24-year-olds) believe in or follow astrology. By contrast, 92 percent of the Chinese public think horoscopes are nonsense.

 

Like so many other things, astrology in the West is about personal things—about me and mine, my spiritual progress, my love life and business success. These concerns are not absent in the East, but larger concerns dominate. Astrology in India is about auspiciousness, about connections, about sacred timing and being in a flow with the ebb and tide of divine forces.

 

Astrology is a part of Vedic self-understanding. We look to the stars to see ourselves better, to discover the mysteries that lie all about us and within us. In rita dharma, that heavenly cosmic orderliness, stars are more than massive conglomerates of molecules or fiery furnaces fleeting afar. They are entities, potent presences that affect us despite their distance. There are, of course, many Hindus today who pooh-pooh such notions. “Stuff and nonsense,” they will cry, “What thoughtful person can accept that stars, so remote, influence life on Earth?”

 

But what thoughtful person, asks the astrologer, would deny the powerful tides dragged across our planet by a faraway moon, or gainsay the not-so-subtle solar forces that are the very stuff of life here? “Ah, but go out another few thousand light years and tell us what petty influences persist,” our doubter might challenge. The jyotishi (Vedic astrologer), realizing the basic East/West difference in world views, attempts to help the skeptic understand the Hindu perspective. “In Eastern thought, particularly Hinduism, we conceive of all existence—including the stars and planets—not as being ‘out there,’ but rather ‘in here’—within the consciousness of each one of us. In other words, consciousness encompasses all of creation. The ‘outside’ and ‘inside’ are mirror images, and the essential nature of the cosmos is not that of multitudinous distinctions but rather the many-faceted expression of a one unified Reality. Thus we do not follow the mechanistic, externalized approach typical of Western thought.”

 

The astrologer is something of a tribal shaman. Ideally, he or she is the one among us with special insight, with a wider vision that lifts awareness beyond our little world, connecting us to the canopy above, expanding perception beyond the narrow sliver of time in which we live by bringing past lives and actions into the now. You could say that astrologers tell time with a bigger watch.

 

The genuine astrologer is, in a sense, a time navigator. He teaches that time is not all colorless and neutral, the same in all directions. Time has its eddies, its waxing and waning, its preferential ways—and in that sense is much like the oceans. No ship’s captain worth his hardtack would consider the sea a uniform body of water, everywhere equal and indifferent to his passage. No, the sea is alive with idle doldrums and treacherous tempests, and, yes, dangers worthy of anticipation.

 

To the astrologer, time is like that sea, with moods and forces, some propelling us swiftly forward, others opposing our well-plotted progress. How foolhardy the seaman who keeps his canvas unfurled in a storm or stows his sails when the good winds blow. Time is a kind of moral wind, blowing now this way, now that. As a ship’s captain heeds the chart reckoned by his navigator as to course, winds and tides, so our life’s journey benefits from periodically examining another chart, our astrologer’s appraisal of protean time’s patterned flow.

 

Those who still doubt are members of a hoary club. Yogaswami of Jaffna had the perfect prescription for them, one that sets aside all of the good versus bad, will versus fate: “All times are auspicious for the pure Siva bhakta.”

 

\subsection{Working with Our Karmic Code}

Philosophically, Hindu astrology reflects the law of karma, which includes both free will and an aspect of predetermination, or fate. Predetermination means our present condition is the result of our past actions from previous lives; free will means we shape our future by our present actions—how we respond to the challenges. The birth chart represents a person’s karmic code, the samskaras with which he or she is born, imprinted on the subtle or astral body. This code is analogous to the genetic code that outlines the main potentials of the physical body. The birth chart indicates the main potentials of our entire life.

 

From an astrologer’s point of view, the birth chart is the most important document we have in life. Yet, like the genetic code, it is written in a mathematical language that requires decoding by a trained expert, and it calls for careful examination over time to unfold its dynamic secrets. K.N. Rao observed, “A horoscope reflects the allotment of karmas of previous lives. We are all getting the results of our karma, but not all of our karma.”

 

According to the Vedas, when a soul takes birth, it descends through the heavens and the atmosphere before reaching Earth, taking on heavier sheaths of material density. It can only take birth in the physical plane at a time karmically in accord with its nature and destiny. The birth chart represents the seed pattern of its life; how it develops depends upon environment as well.

 

Satguru Sivaya Subramuniyaswami advised: “When unfavorable times arise which have to be lived through (as they all too frequently do), we do not carp or cringe, but look at these as most excellent periods for meditation and sadhana rather than worldly activities. Just the reverse for the positive periods. Spiritual progress can be made during both periods. Both negative and positive times are, in fact, positive when used wisely. A competent jyotisha shastri (Vedic astrologer) is of help in forecasting the future as to when times will come along when advancements can be made. A positive mental attitude should be held during all the ups and downs that are predicated to happen. Be as the traveler in a 747 jet, flying high over the cities, rather than a pedestrian wandering the streets below.”

 

\subsection{Cosmic Consciousness}

Astrology is the science of fathoming the influence of the sun, moon, planets and stars upon living creatures. In Sanskrit it is called jyotisha, which means the “science of light”—specifically, “Vedanga Jyotisha,” the astrological limb of the Vedas, said to be the very eye of the Vedas.

 

Jyotisha is a system of understanding how our lives and our karmas relate to the movements of the cosmos, which is cognized as a single greater organism. Under jyotisha is included astronomy, meteorology and forms of divination, including palmistry, the reading of omens, svara (reading the breath) and various oracles.

 

Like yoga, jyotisha is a super science that links us with the cosmic intelligence behind nature. Its first message is that we are one with the Universal Being. New discoveries in quantum physics demonstrate the interrelatedness of the universe, showing subtle levels of immediate interaction even at great distances of time and space. Jyotisha is an integral aspect of the traditional Vedic sciences, along with ayurveda, vastu and yoga, all of which are usually used together.

 

How can the stars and planets influence events on Earth? Obviously the Sun is the basis of all life. According to the Vedas, it also projects a force of intelligence and spirituality. The Moon is important to all creatures and governs the fertility cycles of animals. In the Vedic system it rules the emotional nature. It is well known that the large magnetic and gravitational fields of the planets affect the Earth physically. That they would have subtler influences as well is not illogical.

 

Astrology is common in one form or another in all cultures, though in India it has had the widest and freest development, from the most ancient period to the present day. Ancient Greece and Rome used astrology extensively, as did Europe to the eighteenth century, even though it was often banned by the church. We could say that the type of astrology used by a culture reflects its understanding of the universe, particularly the subtle and spiritual influences guiding our lives. Curiously, modern cultures continue to employ astrology even when its validity is questioned by the scientific community. The ever-popular sun signs in newspapers reveal this undying interest.

 

Jyotisha remains an important facet of Hindu spiritual, religious and social practice, not only in India but worldwide, throughout the Hindu diaspora. It is widely used by Hindus, from common villagers to the sophisticated urban elite. It is an important component of temple worship, pilgrimages and yoga practices. It is avidly used for guiding family life, business and career, physical health and psychological well being. Jyotisha is famously employed by politicians to aid them in winning elections. This science of Vedic astrology is now going worldwide and followed by all those who wish to understand the movement of karma and dharma in their lives.

 

Hindus follow a special sacred yearly astrological calendar, called panchangam, for the right timing of all actions. India has many notable astrological and planetary temples, and new ones are coming up as astrology grows once more in popularity. Astrological icons are found in Hindu temples of all types. In South Indian temples, an altar of astrological Deities, called the navagrahas (“nine planets”), is placed in the corner of the central courtyard. After doing the clockwise perambulation around the Deity sanctum, devotees perform a second walk around the planetary Deities’ shrine.

 

Many yogis and sages have been astrologers or written on astrology. This includes modern figures like Sri Aurobindo, Ganapati Muni, Paramahamsa Yogananda and his guru Sri Yukteswar, Sivananda Murty, Swami Dayananda (Arsha Vidya Gurukulam) and historic figures like Madhva, Bhishma, Vashishta, Parashara, Bhrigu and others.

 

Newborns are traditionally named based on their jyotisha charts which provide optional syllables, based on the nakshatra, to begin the child’s name. Astrological concepts are pervasive in the organization of the calendar and holidays, as well as in areas of life such as the timing of marriage, opening a new business or moving into a new home. Hindu priests and teachers are routinely trained in astrology, among other Vedic disciplines. Introduced as an elective study at the university level in India in 2003, Vedic astrology manages to retain a position among the sciences in modern India. There is a movement in progress to establish a national Vedic university to teach astrology together with the study of tantra, mantra and yoga. All this despite complaints by some scientists.

 

From Kerala in the South to the Himalayas in the North, there is an astounding variety of profound astrological approaches, systems and techniques, including different ways of designing the birth chart.

 

\subsection{Remedial Measures}

Jyotisha does not leave us helpless before the onslaughts of karma. It provides practical ways of dealing with them. Sa­dhana invariably helps neutralize the effects of a “bad chart.” Ultimately, in fact, there is no such thing. A chart that does not portend worldly benefits, such as wealth or marriage, is likely to be good spiritually. “Afflictions” to home, family, marriage and money are often necessary for a person to renounce the world and devote himself to spiritual practices. Afflictions in the area of health can benefit from spiritual practices like mantra japa. While one career may not be favorable for success, another may be. Many remedial measures can help with karmic obstacles, including penance, pilgrimage, bhakti, praying for divine intervention, mantras and yantras, performing rituals, seva and charity. Planetary effects can be softened through special disciplines such as feeding crows (Saturn) or planting trees (Jupiter). Remedial measures are routinely recommended in Vedic, yogic, tantric and ayurvedic texts.

 

The main remedies are ritual and mantra. Propitiating the planets is an integral part of all Hindu rites. Many temples, particularly in the South of India, have a shrine with murtis of all nine planets (navagraha). You can worship them and even employ temple priests to perform special planetary pujas for you.

 

Each planet also has a name mantra (e.g., Om Sum Suryaya Namah for the Sun) and a set of special names, 108 or 1,008, that are chanted to propitiate it. Each planet has a Vedic verse and a Puranic verse used in its worship. Chants to the planets can be done singly or in combination (depending upon the recommendation of one’s teacher) while meditating on a yantra and an image of the Deity or related Deities. Scriptural verses to the Deities can also be recited. For example, Vaishnavas prescribe the Santana Gopala Stotra, to Krishna, for couples whose charts are unfavorable for bearing children. The Mahamrityunjaya Mantra, to Lord Siva, is used to counter the influences of Mars and Saturn.

 

Hindus commonly wear gemstones to balance negative and promote positive influences. Some but not all astrologers prescribe gemstones. Mantras and rituals are preferable but require more time on the part of the person. Each planet has a particular gemstone: ruby for the Sun, pearl for the Moon, red coral for Mars, emerald for Mercury, etc. High quality gemstones can be expensive. Less costly substitutes, though less effective, are allowed. Gemstones should be chosen with care and preferably with a good astrologer’s approval. They should be properly energized with mantras and rituals to function in the best possible manner.

 

Having said all that, sometimes it is better to try to learn from difficult karmas rather than trying to avoid or change them through remedial measures. We cannot buy off the planets or our karma merely by putting on expensive gems or paying someone else to take care of our life. Humility and devotion should be the basis of all remedial measures, along with a willingness to work on ourselves. Some things just can’t be changed or avoided.

 

\subsection{A Mystical Science}
 

How did the ancient Hindu rishis and yogis arrive at the knowledge of astrology? By the same means that all the other Vedic and yogic systems of knowledge arose, and by which they are studied today. Those methods include meditation and samadhi, starting with dharana or samyama, on the Sun, Moon, planets and stars. Another means is communion with planetary Deities, who can speak to us and disclose their nature and influences. Another is reason-based thinking in which we draw connections between phenomena at cosmic and individual levels. Finally, centuries of experience, study and communication among astrologers have helped turn intuition into science.

 

Eighteen traditional systems (siddhantas) are mentioned in Vedic astrology, some bearing the names of the greatest sages of Hinduism. Unfortunately, none of their texts has survived intact. Five of the eighteen were, however, summarized by Varaha ­Mihira—perhaps the greatest astrologer of classical India—in his Pancha Siddhan­tika, namely, Pitamaha (or Bhishma), Vashishta, Paulisha, Romaka and Surya. Of these, only the Surya Siddhanta has survived, and that in a later form. In addition, we have the work of Rishi Parashara, which has endured in expanded form as the Brihat Parashara Hora Shastra. That is the main text of Vedic astrology used today, containing all the essential features of the system. Many South Indian astrologers, however, use the Brihat Jataka and Brihat Samhita of Varaha Mihira, which are similar to Parashara’s overall indications.

 

\subsection{Antiquity}
 

Evidence indicates that jyotisha goes back to ancient times. The Kali Yuga calendar, which begins in 3100bce, is well known. Greeks in the fourth century bce wrote of an Indian calendar relative to ancient king lists with a beginning date of 6700bce (mentioned by Megasthenes in his Indika). The nakshatras (asterisms) are mentioned in the Rig Veda and other Vedic texts, with a Nakshatra Sukta noted in the Taittiriya Brahmana (I.1.2). Nakshatra positions relative to equinox and solstice points aid in the dating of Vedic texts. The Atharva Veda (XIX.7) contains a full listing of the nakshatras, starting with Krittika as the point of the vernal equinox and the solstice in Magha nakshatra, or early Leo, providing a date of around 2000bce. There are references of equinoxes in Rohini (late Taurus, ca. 3000bce), Mrigashira (Orion/Gemini ca. 4000bce), and yet earlier.

 

The Rigveda (I.164.48) refers to a twelvefold wheel of heaven with 360 spokes, showing that a zodiac of 360 degrees was well known in Vedic times. In verse I.155.6, Lord Vishnu is said to have four times ninety, or 360, names, suggesting a divine name for each degree of the zodiac. The Satapatha Brahmana (X.5.4.5) refers to a 720-fold zodiac divided by upa-nakshatras, or sub-asterisms, showing a detailed mathematical observation of the heavens.

 

Rahu and Ketu, the lunar nodes that foreshadow eclipses, are also mentioned in Vedic texts. The planets are mentioned by group or individually. For example, in Aitareya Brahmana XIII.10, we find reference to the birth of Venus (Bhrigu) and Jupiter (Brihaspati), and their relation to the two main rishi families, the Bhrigus and Angirasas, showing a planetary connection with the sages.

 

\subsection{A Comparison with Western Astrology}
 

Like its Western (or Hellenistic) counterpart, jyotisha employs a system of planets, signs, houses and aspects. However, it relies on the sidereal zodiac for its calculations, which differs from the tropical zodiac used in Western astrology, in that an ayanamsa adjustment is made for the gradual precession of the vernal equinox. This puts Hindu astrological calculations in line with the fixed stars and removes it from the criticism of modern astronomy that astrological signs are no longer astronomically accurate. The main ayanamsa currently used is around 24 degrees less than positions in the tropical zodiac, causing most planetary positions to go back one sign from the Western to the Hindu chart. This naturally results in a very different reading. It can be confusing for those accustomed to their Western chart, particularly for the Sun sign, so emphasized in Western astrology. An Aries in Western astrology might be a Pisces according to jyotisha.

Choosing & Working with a Jyotisha Shastri
Go to Vedic astrologers known to have good reputations for their interpretations, predictions and spiritual insight, and who are recommended by people you know and respect, particularly in the Hindu and yoga communities. An astrologer should follow a strict ethical regimen in the pursuit of dharma. He should begin and end his work with mantra, meditation or worship and live and work in a sanctified environment. He must maintain a good sense of humor and humility and give counseling that is beneficial, not harmful to the client, and not fatalistic in nature.

 

Beware of those who claim to give quick, fantastic and infallible predictions, particularly without any detailed examination of your chart, or who declare that they can magically solve your problems through mantras done by them, gems they sell to you or rituals they perform for you, particularly if these are expensive and are done at a distance.

 

It is best to look upon an astrologer like a counselor, doctor or therapist. We don’t expect one session to be enough. An astrologer may need an hour or more to examine the birth chart before even seeing a client. Initial readings with the individual may take over an hour and require several follow-up sessions. Focusing on particular time periods or specific issues may require additional research and analysis. It is best to choose an astrologer you can interact with on a regular basis.

 

The competent astrologer is not a psychic with a crystal ball. Time, effort and examination of a number of factors are needed to reach conclusions as to what is likely to happen to you or what you should do in any given area. Astrological counseling must have an element of spirituality and should direct us to higher goals in life, not simply encourage or direct the fulfillment of worldly desires.

 

Once you have found a good astrologer, it is best to maintain an ongoing relationship with him, like a close friend or advisor. Like a loving mother, father, guru or wise friend, a good astrologer can help navigate life’s challenges. The right use of jyotisha alleviates what is perhaps the greatest fear for human beings—uncertainty and anxiety about the future. It helps us confidently navigate through the confusing waves of ­prarabdha karma, remaining aware of our outer destiny and our timeless inner Self as well.

 

Most Vedic astrologers, particularly in the West, charge for their work, which is the basis of their livelihood, and they deserve comparable compensation as for any professional consultant. Take care to compensate the astrologer appropriately. Without the proper dakshina or offering, advice given may not prove effective.

 

An additional 27-fold division of the zodiac by nakshatras is used in jyotisha. Personality traits are read more through the nakshatra of the Moon (birth star) than by the Sun sign. The birth star is used for naming a person, for determining optimum timing of rituals, and for astrological forecasting. Nakshatra positions of planets are examined in the birth chart as well.

 

Jyotisha rests upon a complex system of calculations that takes into consideration a massive amount of data about planetary and stellar influences, including the mathematical and geometrical relationships between heavenly bodies. A jyotishi must be able to produce the rationale behind his determinations; he cannot rely on speculation or intuition alone. Traditional Hindu astrology does not usually use the newly discovered outer planets (Uranus and Neptune) or Pluto; but it affords special importance to Rahu and Ketu, the lunar nodes, which reflect subtle influences.

 

Jyotisha includes nuanced sub-systems of interpretation and prediction, including numerous divisional charts, several systems of dashas, or planetary periods, and other factors like Ashtakavarga and Muhurta. It determines signs, houses and planetary aspects differently than Western astrology and has a sophisticated system of yogas, or planetary combinations.

 

The Indian system is well known for its understanding of longer cosmic cycles, or yugas. It begins with sixty-year cycles reflecting the movements of Jupiter and Saturn, extends to 3,600-year cycles, and ultimately dates the universe at billions of billions of years. As there are several levels of these cycles, there is still some debate on exactly where we stand in all of these presently.

 

\subsection{Vedic Astrology Today}
 

With the availability of computers to streamline calculations and the many new books coming out, jyotisha is enjoying a renaissance and expansion that is likely to continue for decades. Dr. BV Raman was the main architect of the revival of jyotisha in modern India in the twentieth century, bringing the ancient science into a modern English medium. He was instrumental in its development in the West as well, taking several important trips to the US and inspiring a new generation of jyotishis there. Dr. Raman was the founder of The Astrological Magazine and the Indian Council of Astrological Sciences. His son and daughter, Niranjan Bapu and Gayatri Vasudev, continue in his work.

 

In recent decades Vedic astrology has gone global, along with yoga, Vedanta, vastu and ayurveda. Many non-Hindus and Western Hindus are taking up the science and using it in a regular manner to improve their lives. Hindu-based groups that have promoted it include the TM movement, the Krishna movement (ISKCON), Sivananda, Self-Realization Fellowship (SRF), Arsha Vidya Gurukulam and many others. Jyotisha services are now common in yoga centers and ashrams throughout the world. Various Hindu/Vedic astrology organizations have arisen. Many Ayurvedic groups include it in their curriculum. It is one of the foundations of Vedic counseling and life-guidance.

 

Most traditional jyotisha texts were composed in a medieval Hindu society. Vocations and other aspects of life have evolved radically since that time. For dealing with modern society, planetary influences must be reinterpreted accordingly. Hindu astrologers today are looking at how modern inclinations and professions can be viewed through the chart.

 

\subsection{Misuse of Astrology}

Jyotisha is a sacred science of reading our karma, which makes it powerful and potentially intimidating. We would all like to improve our karma, promote the fulfillment of our desires and remove life’s difficulties. Most people go to astrologers primarily hoping for this, not necessarily seeking deeper spiritual and karmic guidance, which is what a good astrologer can best provide. Unfortunately, there are astrologers who, understanding this vulnerability, take advantage of people, charging large fees for consultations and recommendations.

 

One of the most controversial areas of Vedic astrology is remedial measures. Such measures are an integral part of the system, just as of medicine, but some can be expensive, such as certain gemstones and elaborate rituals. While these may be helpful, some astrologers intimidate the client into feeling they must have these expensive measures or their lives will be ruined. This is not unlike a doctor who recommends medical cures that are burdensome to his patient.

 

In India there are so-called tantric guides who utilize astrology and other occult and spiritual practices. Some are genuine and provide good advice. But there are charlatans as well, who advertise a kind of cure-all approach to human problems, including disease, infertility, lack of a proper marriage partner and career difficulties. Their promises extend even to fabulous wealth, fame or power—all for a certain price. Some do not actually charge for their readings, but offer a list of expensive remedial measures. Often the rituals they recommend are done at a distance, without the person being there, which is usually recommended for successful rituals. Astrologers who are improperly or inadequately trained may simply give bad advice, which can have a negative impact on the lives of their clients, much like a wrong diagnosis and treatment in medicine. Some, particularly new astrologers, may put too much confidence on mechanical techniques of chart readings and make dire predictions based upon these without any real track record in the field.

 

Vedic astrology is a genuine profession to follow, but only if applied with continual deep study and as a spiritual practice. It cannot be approached merely as a job and should not be taken up as a lucrative, influential or powerful career. We must be very careful of how we influence the karma of others and the karmas we create for ourselves by how we read charts, predict the future, and the remedial measures we offer for clients and the expectations and results involved.

 

Yet, we cannot always blame the astrologer. If we approach an astrologer seeking to avoid karmic responsibility in life, which is the opposite of what astrology is meant to teach us, then we can easily fall prey to misleading schemes.

 

Astrology should be part of a spiritual path of controlling the mind and reducing desire, a way of self-knowledge, not a means of ego enhancement for either the astrologer or the client. Then it can work magic—the magic of higher consciousness, not the magic of quick worldly benefits.

 

\subsection{Astrology for You}

THERE ARE FIVE PRIMARY USES OF JYOTISHA, which relate to the main goals of human life: 

\begin{enumerate}
\item[] 1) Kama: family and relationship issues such as marriage compatibility, timing of children and domestic happiness; 
\item[] 2) Artha: help with finances, business and investments; 
\item[] 3) Dharma: determination of career and vocation, and life purpose; 
\item[] 4) Moksha: guidance in spiritual life and for cosmic and self-knowledge; and 5) arogya: physical and mental health, which is the foundation of the first four.
\end{enumerate}

 

In addition, there are four main applications: 
\begin{enumerate}
\item[] 1) Hora or Jataka examines individual birth charts. This is the main approach that we consider for personal potentials and well-being. 

\item[] 2) Mundane astrology examines the charts of nations and political leaders to predict social and political events, including elections. It it also used to predict weather and earthquakes. 

\item[] 3) Prashna (“question”) astrology addresses specific questions—at both individual and collective levels. 

\item[] 4) Muhurta (“moment”) chooses favorable times for all types of action, mundane and spiritual, individual or collective. Hindu holy days, for example, are determined by calculations based on muhurta as recorded in the Hindu calendar (Panchangam).
\end{enumerate}

 

\subsubsection{How Might I Benefit from Jyotisha?}
 

Astrology can be of tremendous benefit. It clarifies our nature, destiny and karma, revealing our svadharma (“own” or “unique path”), so that we know how to pursue our life’s highest purpose. It helps us deal with the limitations of destiny that are present in every life. It shows us how to optimize our hidden potentials. It gives us the key to right timing of actions. And it helps us understand the fundamental laws and patterns of the universe.

 

\subsubsection{How Accurate Is It?}
 

Jyotisha deals with probability, as the factors that determine karma are very complex, both individual and collective, of present and past lives. In this respect it presents a forecast, something like a weather forecast, which contains variables, with some things quite likely and others only possibilities. The planets provide indications and energies that we can become aware of and use in a more positive manner. The stars themselves do not compel us to act, but reflect the subtle forces through which our actions must proceed. We are not controlled by the stars. Rather, they are a reflection of ourselves and our place in the cosmos. To be really accurate, an astrologer requires an extensive analysis of various factors. This can extend into many hours and multiple readings. For this reason, most astrology aims only at macro-managing the chart, looking at long term general trends. Micro-managing can only be done with charts that are given considerable time and effort, requiring hours of study of primary and secondary charts and influences.

 

\subsubsection{What Is the Nature of a Jyotish Reading?}
 

Most people go to astrologers for an examination of their birth chart. This can be looked at for a general life examination; or specific domains of life, like career or health, can be examined within it. Along with the birth chart, the Vedic astrologer will explore various divisional (amsha) charts, particularly the navamsha, nakshatra positions, and planetary periods (dashas and bhuktis), and perhaps annual charts or solar returns.

 

Hindu astrology is as much concerned with helping us improve our karma as with telling us what our destiny is likely to be. It is a kind of “karmic management” program to help us optimize our karma. It is not a “karmic fatalism” under which we are consigned to passively accept bad circumstances in life. To use it in a deterministic manner is to misuse it. By doing so, we fail to benefit from its real power, which is to help us gain mastery over our lives and not be the victims of fluctuating outer events. Astrology is the ultimate science of time management, an aid in dealing with life’s many choices.

 

\subsubsection{What Information Should I Expect to Acquire?}
 

A reading of your natal chart should yield an understanding of trends and periods of your life, with favorable times for action. It should provide a clarification of your karma in all the main fields of life. It may include remedial measures to follow, such as gems, mantras, yajnas and pujas. A good astrologer can easily see important trends and can sometimes predict specific events, but even the best will only be 80 percent correct in predictions, and may go wrong completely if the birth time is incorrect. Knowing that given birth times are not necessarily accurate, he will ask questions of the client to see if the events in the person’s life agree with their chart as calculated by the given date. Sometimes a change or “rectification” of a few minutes in the birth time will yield a much more accurate chart. Follow-up consultations should include a review of previous readings, their indications and predictions and any remedial measures suggested, along with appropriate adjustments. Follow-up readings may address changes in planetary periods, transits or annual chart indications, along with the client’s questions and concern

\subsubsection{Astrology for Health}
 

Medical astrology aims at assessing our health potential, our likely diseases, their possible cure and our lifespan, as well as potential emotional and mental problems. This system is intimately connected with ayurveda, the Vedic medicine. All of us eventually get sick and die, so every chart has negative health potentials—a disturbing fact when dealing with those close to us. Proper analysis can show us when a person is likely to get sick and their potential for recovery. By providing early warning of impending negative planetary periods for our health, astrology gives us time to take precautions and offers methods to minimize the negative effects.

 

\subsection{What Can I Do to Get Started with Astrology?}
 
\begin{enumerate}
\item[] 1) First, find a suitable astrologer and have your birth chart read. He or she will help you learn about your chart so you can understand its various elements, including your ascendant, Moon sign, Sun sign, important yogas, and the ruling planets.

\item[] 2) Some find it helpful to learn the birth charts of their family members as well.

\item[] 3) It is informative to be aware of your Nakshatra, its name, Deity, ruling planet and indications.

\item[] 4) Learn and celebrate your tithi pravesh, or Vedic lunar birthday, the same day and month of the lunar calendar at your birth, which can be as much as two weeks different from the solar birthday.

\item[] 5) Learn about remedial measures, particularly mantras to the planets and the place of planets in temple worship.

\item[] 6) You may wish to incorporate jyotisha mantra japa along with your regular japa.
\end{enumerate}
 

\subsubsection{Once I Have My Interpreted Chart, How Do I Use It?}
 
 
\begin{enumerate}
\item[] 1) Most importantly, you can use this knowledge to understand and mold your character, as you work with your emotional and intellectual inclinations, strengths and weaknesses.

\item[] 2) Through the years, you can observe and anticipate the ebbs and changes as you go through your planetary periods.

\item[] 3) You may find it helpful to consult your shastri when planning major events, changes or facing important life issues. Knowing when influences will prevail, you can plan accordingly in working through your karmas.

\item[] 4) Use the information you have gained when making long- and short-term plans and decisions.
\end{enumerate}
 

\subsubsection{How Is the Panchangam Best Used?}
 

\begin{enumerate}
\item[] 1) Acquire a panchangam (Vedic astrological calendar) for your area and observe the auspicious days and times it indicates. I recommend the detailed Panchangam by Himalayan Academy, produced annually for any time zone. It has a good introduction explaining its use. Yet Vedic software usually has an option indicating the Panchangam information for the time and place of the chart considered.

\item[] 2) Use the panchangam to choose auspicious days and times to begin activities and projects, such as weddings, new ventures or entering a new home. Many festival days like Ganesh chaturthi, Ram Navami, Diwali or Navaratri are ideal for special events.
\end{enumerate}
 

\subsubsection{What Other Ways Can I Use Jyotisha?}
 

\begin{enumerate}
\item[] 1) Those who have a good astrology to consult (or are well versed in the science themselves), may use jyotisha to help in selecting employees, associates, business partners, etc.

\item[] 2) Baby names are often chosen according to astrological factors, mainly Nakshatras.

\item[] 3) One of the main uses is for marriage. Traditional families will always consult a shastri to check compatibility between potential spouses, and between their families.

\item[] 4) Jyotisha can, in many ways, grant a deeper, more appreciative, understanding of other people and thus improve relationships.
\end{enumerate}
 

\subsubsection{How Can I Use this Wisdom to Guide My Children?}
 

\begin{enumerate}
\item[] 1) The knowledge revealed in the child’s natal chart will help you understand and confidently work with his or her nature and development.

\item[] 2) It will enable you to competently guide the child through the various periods indicated in the chart.

\item[] 3) Applied at a deeper level, jyotisha can help you cognize how your nature, as a parent, impacts the child. All this gives patience and stability. Satguru Sivaya Subramuniyaswami observed: “For raising offspring, a forecast can be of the utmost help. A baby predicted to have a fiery temper should be raised to always be kind and considerate of others’ feelings, taught to never argue with others. Of course, good examples must be set early on by parents. This will soften the inclination toward temper. Fighting the child’s impulse will just amplify it. A child of an independent nature should be taught early on to care for himself in all respects so that the life ahead will benefit society and bring honor to the family. ”
\end{enumerate}
 

\subsection{In a Nutshell}

Indeed, jyotisha is an intricate, complicated system of knowledge, requiring a good grasp of astronomy, astrology and human nature. People can and do spend lifetimes exploring its vastness. But here is a super-simple summary.

 

Vedic astrology is based on math­e­matical divisions of the zodiac and defined relationships between planetary locations. The zodiac is a narrow band across the sky through which the sun, moon and planets travel, expressing various influences, both physical and subtle. The main zodiac division used is that of twelve signs, or rashis, of 30 degrees each, but other divisional charts are used as well.

 

The Earth rotates at about one sign every two hours, causing the signs and planets in them to rise in the east and set in the west. The point of the sign rising in the east forms the cusp of the first house (bhava). This is the ascendant, rising sign or lagna, which determines the orientation of the chart as a whole. The sign ahead of the rising sign becomes the second house, with the rest of the houses following in sequence.

 

Planets: There are three levels of planetary Deities. The Devata represents the planet itself as a Divine power. The Adhidevata represents the over-ruling cosmic power beyond the planet. The Pratyadhi-Devata represents the aspect of Ishvara behind the planet.


\subsection{Are “Bad Times” Really Bad Times?}
 

\begin{enumerate}
	\item[] 1) There are astrologically bad times in life, just as we have difficulties in various spheres of life, or bad weather days. These may be related to health, work, finances or relationships. But bad times also can aid in spiritual growth and are good for sadhana.

	\item[] 2) There are ways to deal with bad astrological times, just as with adverse weather conditions. Astrology should never cause us to lose our sense of well-being.

	\item[] 3) Better than having good karma in the chart is having the strength to overcome adversity, which is always there in life to some degree. Satguru Sivaya Subramuniyaswami pointed out, “Difficulties need not be bad news if they are approached as our chance to grow in facing them.”
\end{enumerate}
 

\subsection{How Much Time and Emphasis on Jyotisha Is Healthy?}
 

\begin{enumerate}
\item[] 1) Using it as a personal meditative tool and timing aid is helpful.

\item[] 2) Daily examination of the panchangam is informative, like an astrological weather report.

\item[] 3) Looking at the birth chart around the times of one’s birthday or at the changing of planetary periods or important transits (two or three times a year) is wise.

\item[] 4) Professionals may find it useful to consult a shastri of jyotisha and vastu on a regular basis. Kings and politicians often had a full-time retinue of astrologers.

\item[] 5) Regular astrologically based worship is good, such as mantras to planets, and circumambulating the planetary altar in temples.

\item[] 6) Dependence on astrology can be taken to extremes. It should be a guide to action, not a substitute for it.
\end{enumerate}