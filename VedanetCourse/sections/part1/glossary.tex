\section{GLOSSARIES: SANSKRIT AND ENGLISH}

\subsection{I. SANSKRIT GLOSSARY}
The spelling of these terms may different slightly in various books.   Abda—year Abdadhipati—ruler of the year Abhijit—Nakshatra counted between the twenty-second and twenty-third, the star Vega Agni—fire, the Sun, the deity of Krittika Nakshatra Agnihotra—daily sunrise and sunset fire offerings Akshavedamsha—divisional forty-fifth Amatyakara—Signficator of friend or confidant Amavasya—the day of the new Moon Amsha—division, relative to divisional charts Angaraka—another name for Mars Antar Dasha—subminor planetary period Anuradha—seventeenth Nakshatra Apachaya—houses 1, 2, 4, 7, 8 Apoklimas—cadent houses Ardra—sixth Nakshatra Artha—wealth or material goals Ashtakavarga—system of point calculation or bindus for planets and signs Ashtami—ninth Tithi of the moon Ashwini—first Nakshatra in zodiac Aslesha/ Ashlesha—ninth Nakshatra Asuras—demons Atman—Divine Self or Soul Atmakaraka—significator of inner Self or Atma Ayanamsha—difference between sidereal and tropical zodiacs Ayana bala—strength of planets relative to solstice points Ayurdaya—calculation of longevity Ayurveda—Vedic medicine, medicinal approach used with Vedic Astrology  

Bhadra Yoga—Maha Purusha Yoga of Mercury Bhamsha—divisional twenty-seventh chart Bharani—second Nakshatra in zodiac Bhava—house Bhava chakra—house chart Bhava madhya—midpoint of house Bhava sandhi—transitional point between houses Bhava bala—strength of houses Bhinna Ashtakavarga—Ashtakavarga points of each planet by sign Bhrigu—famous Vedic Astrology family Bhrigu Samhita—collection of palm leaf astrological data by special Vedic astrologers Bhukti Dasha—minor planetary period Bija mantra—seed syllables Bindus—points in Ashtakavarga system Brahma—cosmic creative force Brahmins—spiritual class of people Brihaspati—Jupiter Buddhi—intelligence, reason Budha—Mercury (not Buddha)  

Chakra—horoscope wheel; force centers of the subtle body Chandra—the Moon Chandra Lagna—Moon as Ascendant Chandra Rashi—Moon Sign Chara Rashis—cardinal signs Chaturthi—fourth Tithi of the moon Chaturdashi—fourteenth Tithi of the moon Chaturtamsa—divisional fourth chart Chaturvimshamsha—divisional twenty-fourth Chesta bala—motional strength Chitra—fourteenth Nakshatra  

Dasha—major planetary period Dashami—tenth Tithi of the moon Dashamsha or Dashamsa—divisional tenth Devas—Gods Dhanus—Sagittarius Dharma—career, honor or status Dig bala—directional strength of planets Drig bala—aspectual strength of planets Drishti—planetary aspect Durga—the Goddess as the demon-slayer; related to Rahu Dushtanas—difficult houses, 6, 8 and 12 Dvadashamsha—divisional twelfth Dvadashi—twelth Tithi of the moon Dvisvabhava Rashis—dual natured or mutable signs Dvitiya—second Tithi of the moon  

Ekadashi—eleventh Tithi of the moon Gaja Keshari Yoga—Yoga of Jupiter in an angle from Ascendant or Moon Ganapati—same as Ganesh Ganesh—the elephant faced God, related to Jupiter Graha—planet, also demon Guru—Jupiter, spiritual guide   Hamsa Yoga—Maha Purusha Yoga of Jupiter Hasta—thirteenth Nakshatra Havana—same as homa Homa—Vedic and Hindu fire rituals Hora—Planetary hours, 1/2 division of sign, hour in general  

Jaimini—author of another system of Hindu astrology Janma Lagna—birth Ascendant sign Janma Rashi—birth sign, meaning Moon sign in birthchart Janma Nakshatra—birth Nakshatra of Moon Jyeshta—eighteenth Nakshatra Jyotish—Vedic or Hindu astrology, science of light  

Kala bala—temporal strength of planets Kali—dark form of the Goddess; related to Saturn Kali Yuga—dark or iron age Kama—desire Kanya—Virgo Kapha—biological water humor Karaka—significator Karana—twofold division of Tithi Karma—law of cause and effect Kataka—Cancer Kendra—angular house or quadrant Ketu—south node of the Moon or dragons tail Khavedamsha—divisional fortieth Krittika—third Nakshatra in zodiac Krishna—great Hindu avatar Krishna Paksha—waning or dark half of moon Kuja—Sanskrit for Mars Kuja Dosha—difficult placements of Mars for marriage Kumbha—Aquarius Kuta—point system, used in marriage compatibility

Lakshmi—Goddess of fortune and beauty; related to Venus Lagna—Ascendant   Magha—tenth Nakshatra Maha Dasha—major planetary period Maha Purusha Yogas—planetary combinations that give strong personalities Malavya Yoga—Maha Purusha Yoga of Venus Manas—mind or general feeling potential Mangala—another name for Mars Mantras—sacred or empowered sounds Masa—month Masadhipati—ruler of the month Matri Karaka—Significator of mother Mesha—Aries Mihira, Varaha—great Vedic Astrology of around 500 AD Mina—Pisces Mithuna—Gemini Moksha—liberation Mrigashiras or Mrigashirsha—fifth Nakshatra Muhurta—a thirtyfold division of the day (forty-eight minutes); also refers to electional astrology Mula—nineteenth Nakshatra Mulatrikona—root trine, specially favorable sign positions for planets, nearly as good as exaltation

Naisargika bala—natural strength Nakshatras—27 lunar constellations or asterisms Navami—ninth Tithi of the moon Navamsha—divisional ninth chart  

Pada—quarter, particularly of Nakshatra (03 20) Paksha bala—strength of planets relative to phases of the moon Pañchanga—Astrological forecasting in Vedic Astrology, based on the five factors of day, Nakshatra, Tithi, Karana and Yoga, name of Sidereal yearly almanac Panaparas—succedent houses Pañchami—fifth Tithi of the moon Panchanga—Hindu almanac Parashara—father of Vedic Astrology, author of main system used Pitri Karaka—Significator of father Pitta—biological fire humor Prashna—question, refers to horary astrology Prastara Ashtakavarga—Ashtavarga spread sheet for each planet Pratipat—day after the moon is full or new, day after full moon is called Krishna Pratipat, after new moon is called Shukla Pratipat Puja—Hindu rituals Punarvasu—seventh Nakshatra Purvashadha—twentieth Nakshatra Purva Bhadrapada—twenty-fifth Nakshatra Purva Phalguni—eleventh Nakshatra Pushya or Pushyami—eighth Nakshatra Putra Karaka—Significator of son or children  

Rahu—north node of Moon or dragons head Raja Yoga—combination of planetary influences or planet which gives great power Rajasic—agitated in quality Rama—seventh avatar of Vishnu, Divine warrior; related to the Sun Rashi chakra—basic sign chart Ravi—the Sun Revati—twenty-seventh Nakshatra Rohini—fourth Nakshatra Ruchaka Yoga—Mahapurusha Yoga of Mars Rudra—fierce form of Shiva; related to Ketu  

Sadesati—the seven year period centered on Saturns transit of the natal moon Sambhanda—full relationship between planets Saptamsha—divisional seventh Sapta varga—the seven vargas or divisional charts Sarvashtakavarga—total Ashtakavarga points of signs Sattvic—spiritual in effect Satya Yuga—age of truth or golden age Shadbala—system of determining planetary strengths and weakness Shadvargas—six main divisional charts Shani or Shanaishcharya—Saturn Shaha Yoga—Maha Purusha Yoga of Saturn Shastyamsha—divisional sixtieth Shatabhishak—twenty-fourth Nakshatra Shiva—God of the Hindu trinity who destroys the creation and takes us back to the transcendant Shodashamsha—divisional sixteenth chart Shravana—twenty-second Nakshatra Shravishta—twenty-third Nakshatra Shukra—Venus Siddhamsha—divisional twenth-fourth, same as Chaturvimshamsha Simha—Leo Skanda—the war God; related to Mars Soma—the Moon Sthira Rashis—fixed signs Stri Karaka—Significator of wive or marriage partner Surya—the Sun Swati or Svati—fifteenth Nakshatra  

Tajika—annual chart or solar return and system of its interpretation Tamasic—dark in quality Thula—Libra Tithi—thirty-fold division of lunar month Trayodashi—thirteenth Tithi of the moon Treta Yuga—third or silver age Trikona—trine houses Trimshamsha—divisional thirtieth Tritiya—third Tithi of the moon  

Upachaya—houses 3, 6, 10, 11 Uttarashada—twenty-first Nakshatra Uttara Bhadrapada—twenty-sixth Nakshatra Uttara Phalguni—twelfth Nakshatra, also called Uttara  

Vakra—retrograde Vara—day Varadhipati—ruler of the day Vata—biological air humor Varga—divisional or divisional charts Vargottama—in the same sign in both the birthchart and Navamsha Vedanta—Vedic philosophy of Self-realization Vedas—Vedic scriptures Vimshamsha—divisional twentieth Vimshopak—varga calculation of planetary strength and weakness Vishakha—sixteenth Nakshatra Vishnu—God of the Vedic trinity who preserves and maintains the creation and the cosmic order Vrishchika—Scorpio Vrishabha—Taurus  

Yajña—rituals to propitiate the planets, often using sacred fires, pronounced Yagya Yantras—mystic diagrams, used to harmonize planetary influences Yoga—combination of planetary influences; Sun-moon relationships in Panchanga or Astrological forecasting; spiritual practice Yuddha—planetary war Yugas—world-ages    

\subsection{II. ENGLISH GLOSSARY}
  Angular—houses 1, 4, 7 and 10 Ascendant—first house or rising sign Aspects—relationship between planets according to zodiacal angle between their positions Astral plane—subtle or dream plane Astrocartography—Western system of aligning the birthchart with latitude and longitude to show favorability of place locations Benefics—planets with facilitating or strengthening effect   Cadent—houses 3, 6, 9, 12 Cardinal signs—Aries, Cancer, Libra, Capricorn Causal plane—plane of cosmic law and cosmic intelligence Combust—close conjunction of planets with the Sun Conjunction—location of planets in close proximity to each other Composite charts—charts made by averaging the planetary positions of two charts Cusp—central point of a house   Debility—most difficult sign placement for a planet Decanate—threefold division of signs Declination—position of a planet north or south of the equator Descendant—point opposite the Ascendant,cusp of seventh house Dispositorship—rulership of planets over other planets located in its signs   Electional astrology—astrological science of determining favorable times for actions of various sorts Electional chart—a chart determined to be a favorable time for a certain action Ephemeris—book containing planetary positions by day Exaltation—best sign placement for a planet Fall—same as debility, most difficult sign placement for a planet Fixed signs—Taurus, Leo, Scorpio, Aquarius   Divisional charts—subdivisions of the birthchart Horary astrology—astrology directed towards specific issues Horary chart—chart based on a question Houses—twelvefold division of zodiac according to degree rising on eastern horizon at moment of chart House significators—planets generally controlling the affairs of specific houses Jupiter return—return of Jupiter to its place in the birthchart, happens every twelve years Lunar returns—return of the Moon to its place in the birthchart   Malefics—planets of difficult or damaging effect Midheaven—highest point in the chart, cusp of the tenth house Mutable signs—Gemini, Virgo, Sagittarius, Pisces Mutual receptivity—exchange of signs between planets   Natal—relative to the birthchart, as natal moon Natal chart—birthchart   Planetary war—condition caused by conjunction of planets to within one degree; the planet with the lower number of minutes considered the winner Progressions—planetary positions progressed from birthchart usually by the day per year basis Retrograde—backward movement of planets in zodiac   Saturn return—return of Saturn to its place in the birthchart, happens every twenty-nine and a half years Sidereal zodiac—zodiac of fixed stars Solar returns—returns of the Sun to its place in the birthchart; differs sidereally from tropically Succedent—houses 2, 5, 8, 11   Transits—current planetary positions relative to birth positions Trine houses—houses 1, 5 and 9 Tropical zodiac—zodiac defined by the equinoctical points