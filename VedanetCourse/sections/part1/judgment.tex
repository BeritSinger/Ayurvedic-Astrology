\section{FUNDAMENTALS OF HOROSCOPE JUDGEMENT}
 

In this section we will go into detail on how to judge a chart, starting with the prime factors of interpretation, how to gauge the planets and the fields in which they operate, as well as some of the basic groupings of planets that can occur in the chart.

 

Along with this Lesson, please make sure to study the Workbook Lesson 2 on Chart Judgement, as it adds more practical information to what is presented here. Note the charts examined in the Workbook and in the Example Charts at the end of the Astrology of the Seers.

 



\subsection{PLANETS AND SIGNS – FORCES AND FIELDS}
 

The signs represent the field of activity. The planets represent the forces in operation within that field. It is important to remember the following rule:

 

The signs maintain their integrity of meaning regardless of the planets within them or aspecting them.
Similarly, the planets maintain their integrity of meaning, regardless of the field or sign in which they are acting.
 

The field of activity is not determined by the nature of the planet influencing it. Nor is the nature of the planet determined by the field of activity. This is just like a person in a house. The person who happens to be in it does not determine the nature of the house, nor is the person determined by the nature of the house in which he resides. Both factors influence each other. The person may alter the house and the house may limit the activity of the person. Hence, it is a question of compatibility or appropriateness of place and function.

 

The house that the sign rules in the chart specifies the field of activity represented by the signs. The sign shows the character of the field. The house shows its affect upon the outer world.
 

The first house represents the field of activity for the life manifestation in general, which is specifically the physical body. The sign marking it will show the character field through which we project our energy in the outer world. For example, a Leo Ascendant will project Leo energy, a dramatic, expressive, willful solar field that will mark the field of general life action. If, however, Leo marks the seventh house, that expressive Leo field will mark the sphere of relationship and these traits will more likely occur in the partner.

 

The planets aspecting the sign show the quality of the force working behind it. For example, Saturn in a Leo Ascendant will show the Leo field of self-expression inhibited by a Saturnian influence. The person may sick or dominated by other people. Saturn in Leo in the seventh house will show that Leo field of relationship inhibited. Marriage could be denied or marital happiness may be limited.

 

A person may have many strong planets in Aries in their birthchart, like the Sun, Mars and Jupiter, but if they are posited in a weak house, like the twelfth, it will not be possible for them to manifest that strength in the outer world. This will give a strong character field for the twelfth house and its issues. It may cause them to seek liberation or spirituality, or simply make them work behind the scenes. It may even give them a strong character for self-indulgence, or work in a foreign country.

 

A person with Taurus rising will have a Taurean field of activity or manifestation in the material world because Taurus is the sign of fixed earth. They will have a strong sense of matter, beauty and order. They will have a lot of work to do on a practical, material or physical level with the accumulation, development and refinement of form. Their lives will be oriented towards practical manifestation in the material world. Even when highly evolved spiritually, they will still have to manifest that spirituality through a mastery of the world of form. They will seek order, structure, beauty or love.

 

On the other hand, someone with a Libra Ascend­ant will be strongly oriented to the social world and toward the projection of ideas to change the world because Libra is cardinal air. They will be more socially, politically, group or friendship oriented and will like to be leaders in social or intellectual contexts. Their ideas and ideals will be more important to them than practical matters and they will be willing to sacrifice worldly goals and possessions for them.

 

The planets represent the kind of energy the individual will manifest within the field of activity represented by the sign. Taurus Ascendant strongly aspected by Saturn will give the individual concern with property, possessions, banking and a conservative trend of character or at least an extreme practicality in life and need for control and domination in the material world. Saturns contracting nature will combine with the earthly qualities of Taurus.

 

Libra Ascendant strongly aspected by the Sun will make for a strong political, military, social or intellectual leader. The Libra field of activity will be dominated by the quality of the Sun. Such are the discriminations that the astrologer must make in combining sign, house and planetary influences.

 

The signs show the nature of the field;
The planets show the quality of energy operating within that field.
 

Generally, the ruler of the sign will have the strongest energy to influence it. If the ruler of a sign aspects its own sign or is strongly placed, the field and the force operating within the sign will be of the same character and strength. For example, if Aries is the Ascendant and Mars is in Leo, another fire sign, the properties of the Aries Ascendant will be increased. However, if Mars is in Virgo, a difficult sign for it, many of the properties of the Aries Ascendant may be weakened or inhibited. The individual may have ill health or difficulty in self-expression.

 

If the nature of the sign and that of planets aspecting it is similar to the ruler of the sign, the sign gains more strength.
 

For example, Aries ascendant aspected by the Sun and Mars will be strong because both of these are fiery planets and friends. However, if it is aspected by natural and temporal malefics like Saturn and Mercury, which represent air and ether, it will be weakened.

 

Yet this general rule has exceptions. As the ruler of the sign is only one planet of nine, other planets commonly overcome its influence. For this reason, individuals may have a quality of energy or body type different than the predominant element of the Ascendant.

 

For example, if Aries is the Ascendant and Mars is in Capricorn, well placed, this may not produce a strong Aries type if the Moon is located with Saturn and Rahu in Virgo. These Virgo planets will stifle the Aries energy. Weak health may affect the career gains indicated by an exalted Mars in the tenth as the Ascendant lord.

 

For this reason, it is best to balance the nature of a sign by the planetary influences upon it. The latter do have the power to overcome or modify the former. This does not mean that we take the character of the signs lightly, but that their capacity to project energy does not have the same unchangeablity as their determination as to what field the energy manifests in does.

 

This rule has some very basic practical applications. For example, the physical constitution tends to follow the planets most strongly aspecting the Ascendant or most strongly placed relative to it. Libra rising may not be an air‑type but a fire‑type if Mars and the Sun are strong, say in Aries in the seventh. Yet Libra will still determine the field of material manifestation and give social, group or intellectual contacts typical of an air sign. So whatever planets aspect­ a sign, it still does not lose its nature as determining a specific field of manifestation.

 

Generally, the Ascendant determines the field of material manifestation, the field of the physical body, but not necessarily the quality of its energy.
The Moon sign determines the field of social manifestation, the field of the astral body (mind and emotion, basically memory). Yet the influences on the Moon and the Moon sign should also be considered to determine mental and emotional tempera­ment.
The Sun sign determines the field of individual manifestation, the private or inner level, the field of the causal body (will and intelligence, ego and self). Yet the influences on the Sun and the Sun sign will have to be considered also to determi­ne more the level of develop­ment of the soul.
 

The indications of the other planets should be considered in the same light. Therefore, we must always discriminate between fields and forces. Planets are usually stronger than signs that are their projections. Yet the fields the signs represent are not canceled by the planets aspecting them.

 

 
 

\subsection{STRONG AND WEAK ASCENDANT TYPES}

 

We must discriminate between strong and weak Ascendant types. A strong type Aries Ascendant would be willful, forceful, outgoing, outspoken, headstrong with much drive and initiative. A weak Aries type, however, would be weak, impulsive, emotional, reactive, defensive, may get headaches, ulcers or other diseases. The weak type reflects the field of the sign but not the quality of its force.

 

A strong Leo type will have a strong character, will power, drama and power about them. A weak Leo type would be self-defeated, have excessive imaginations about themselves, have too high standards and suffer from a feeling of lack of accomplishment.

 

Often when an Ascendant is weak, other people in the life of the native may express its qualities. A weak Leo Ascendant may not make for a strong personality. The strong Leo traits may appear in relationship or in the circumstances of the life. The individual may work for or under or be related to strong Leo types. Leo would show their field of human communication and self-projection, but not the energy that they are able to develop themselves.

 

\subsection{HOW TO JUDGE THE CHART/ INITIAL CONSIDERATIONS}
 

Now let us go over the broad and general factors used in judging charts. Try to understand these thoroughly before going into specific and detailed interpretations. Many available books on Vedic Astrology tell us what a certain planet in a certain sign or house means. In this course, we have aimed at explaining the meaning of the planets and signs so that you can determine this for yourself.

 

You will often find that the meaning of planets in signs and houses is portrayed rigidly in astrological cookbooks. Saturn in the fifth may be said to cause sterility, lack of intelligence and poverty, negating all possible indications of the fifth house. Then you will see a chart with Saturn in the fifth in which the person has children or is highly intelligent. Such delineations can be confusing unless we remember that they are only general guidelines, not the last word on the matter.

 

Similar books list yogas or planetary combinations that produce certain results. Again, these yogas may not always work and should not be taken rigidly. It is only if they are very prominent in the chart and are not canceled by other factors that they give their specific results. Hence, it is better to know what the planets, signs and houses mean, rather than merely to look up stereotypic interpretations. Such interpretations can be very helpful as a reference but should not constitute our primary means of interpretation. With the information in this course, you should be able to examine such factors objectively.

 

When examining a chart, many issues come up and all the different factors can easily confuse us. The same chart may have major strengths and weaknesses, for example. Therefore, it is important to proceed step by step, from the primary to the secondary. The basic rule is fairly simple:

 

First, thoroughly acquaint yourself with the general factors, the Ascendant, Moon and Sun, for example, and see what is obvious in them.
Then go into greater detail. But always remember that these major factors will not be altered by minor factors but only adjusted by them. In this way, the mass of planetary data will not confuse you.
 

For this reason, it is helpful to first examine charts only in terms of Ascendant, Moon or Sun. Once you have classified a number of charts by Ascendant, for example, then you will have a sense of what this factor means and can proceed to other factors with a feeling of confidence.

 

\subsection{RELATIVE IMPORTANCE OF DIFFERENT FACTORS}
Here we will look at the chart relative to the prime factors of chart analysis.

\subsubsection{ASCENDANT VERSUS MOON SIGN}

 

In judging the chart, first determine the relative strengths of the Ascendant and the Moon sign. If, for example, the chart is Gemini rising and Taurus Moon with no planets aspecting the Ascendant and with Jupiter, Venus and Ketu in Scorpio aspecting the Moon, then the Moon sign will be stronger than the Ascendant, and houses can be counted mainly from the Moon. Such a person would probably be watery type, owing to the strength of the Moon sign and the aspect of watery planets upon it.

 

So the first major choice to make is the relative weights to give the Ascendant and Moon signs. Usually the ascendant has more weight but many exceptions exist.

Generally speaking, the Ascendant counts for two-thirds and the Moon sign for one-third on any issue.
 

If the same condition exists from the Moon as from the ascendant, then its results are more likely to manifest. If the seventh from the Moon, as well as the seventh from the ascendant is afflicted by separative planets, then divorce is more likely.

 

\subsubsection{HOUSES FROM THE SUN SIGN}

 

Some astrologers consider houses from the Sun, as well as those from the Ascendant and Moon. In this system, the ascendant counts for one‑half, the Moon sign for one‑third, the Sun sign for one‑sixth. This system was called “Sudarshan” or perfect vision, as it provides a more complete view. Yet for most purposes, consideration of the houses from the Moon and Ascendant is enough. Still we have to consider the Sun sign as very important, particularly for issues of the self and soul for which the Sun is the significator.

 

\subsubsection{ASPECTS}

 

The effect of planetary aspects is usually stronger than that of planetary conjunctions.
 

If the Sun is with Mars but aspected by Jupiter then Jupiter will generally have a stronger effect upon the Sun or Mars than either conjunct planet. Conjunctions are not aspects in Vedic Astrology but associations. Aspects are stronger determining factors than conjunctions. An aspect can nullify the effect of a conjunction. For example, Jupiter‑Venus conjunct aspected by Saturn may be largely neutralized, particularly if Saturn is more prominently placed, such as in the tenth house.

 

Of course, if three or more planets are conjunct, this creates a powerful concentration of energy, particularly if the Sun and the Moon are involved, that can be more significant than any aspect. It becomes another Ascendant, often more important than the rising sign.

 

All planets in angles to the Ascendant influence it, almost like an aspect. In this regard, planets in the fourth house will be stronger than planets in the first house, those in the seventh will be stronger than in the fourth, and those in the tenth the strongest. Planets will be stronger to the extent they are in their own signs or those of their own element or combined with like planets. For example, the Sun with Mars will be stronger than with Mercury. The Sun in Sagittarius (a fiery sign) in the tenth house will be stronger than in Pisces (a watery sign).

The closer planets are in degree to exact aspects and conjunctions, the stronger their action.
 

Saturn on the degree of the Ascendant, for example, will be more debilitating than ten degrees away in the same sign. So the exact degree of positions should always be considered.

 

\subsubsection{HOUSE AND SIGN POSITIONS}

 

Generally speaking, the house position or location is more important than sign position. For example, it is better if Jupiter is debilitated in Capricorn in a good house, like the ninth, than exalted in Cancer in a bad house like the eighth. But each factor has its weight as well. Good sign position will give some good results. Bad house position will give some bad results, and vice versa.

 

\subsubsection{BIRTHCHART, NAVAMSHA AND OTHER CHARTS}

 

The regular birthchart, the sign chart or rashi chakra, is only a general chart. It remains the same for people born within the same hour or two. Sometimes it is repeated for people born the day before or after at the same hour. This is because the Moon stays in the same sign for two and a half days. For this reason, we cannot rely upon the rashi chart alone for great accuracy. Those who read specific events from the rashi chart alone are relying upon intuition because the rashi chart is shared by hundreds of people.

 

For this reason, the rashi chart is not enough to allow for exact determination of physical constitution. It shows major tendencies but is at most 80% accurate.
 

We must, therefore, examine other factors than the basic birthchart before making primary determinations. The birthchart shows us the general field of energies in operation but not their specific manifestations.

 

We should always use the navamsha along with the basic birthchart.
 

The navamsha Ascendant is also an important factor for determining the physical constitution (though with the generally approximate birth times we often have we cannot always be sure of it).

 

It is helpful to examine the degree of planets in a chart, the closeness of aspects, and the bhava or house chart along with the positions of the midheaven and nadir, which require specific degree positions.
 

Aspects that are nearly exact have a special power. Aspects that occur in the navamsha chart as well as in the birthchart also possess special power. Generally, the closer the aspects, the more likely they will occur in different divisional charts as well.  This is particularly true for conjunctions.

 

I also like to examine the Drekkana chart as well as the Rashi and Navamsha charts for all general indications. Two people with the same Rashi but different Drekkanas have very different charts. One can generally be certain of the Drekkana even when the Navamsha itself may be in doubt because of an uncertain birthtime. The Drekkana remains the same for a period of about forty minutes, while the Navamsha changes about every thirteen minutes.
 

\subsubsection{STRENGTH OF PLANETS}

 

Strength of planets should be determined by sign, house and divisional positions, as well as by directional strength and by aspect. A planet can only give the results it indicates if it has the power to do so.

 

Elaborate calculations to determine exact planetary strength exist, as under Shadbala. Please examine the section relative to it. Such calculations, however, tend to obscure the obvious. If, for example, the Moon is aspected by Mars, Saturn and Rahu and is weak in brightness, it will not likely give good results whatever its Shadbala or other strengths. So it is important to learn the obvious first and not to consider bewildering calculations which are only a matter of fine‑tuning.

 

Generally speaking, aspect strength overrides positional strength.
 

Mercury in Virgo will not be good if aspected by several malefics. In addition, house strength usually overrides sign strength. For example, Jupiter in a bad house, like the twelfth even in its own or exalted sign, will still create some problems. On the other hand, Jupiter debilitated in a good house like the ninth or tenth will still give benefits.

 

\subsection{SYNTHESIS: YOGAS}

 

Readings always require synthesizing various factors. These are the basis of the various Yogas or planetary combinations looked for in a chart. We have already mentioned these briefly. Note audio that explains the way of thinking behind these Yogas, which reflect the prime patterns that are possible in chart interpretation. This integrative way of chart examination, revealing background energies is what is meant by Yogas, not just any formal set of such combinations that have been listed in various books.



\subsection{\textbf{Audio Overview of Yogas in Vedic Astrology with Dr. Frawley}}
 

\subsection{GENERAL RULES OF CHART EXAMINATION.}

 

First read the sign position of planets. This relates more to the soul or inner dimension. Note own sign rulership, exaltation and debility in particular.
Then examine their house positions. This shows their outer dimension or potential to manifest in the material world. Most important is benefics in angles and trines (1, 4, 5, 7, 9, 12) and malefics in upachaya (3, 6, 11) for giving strength to the chart.
Above all, note the groupings of planets and their indications as house lords.
Try to ascertain the main pattern of planetary influences that exists in the chart. Determine the predominating or most characteristic planet, its allies and its opponents.
 

There are no absolute rules of chart interpretation. All depends on our capacity to read the pattern of the chart according to the logic of how the main indicators function. Learn the language and the logic of astrology but do not accept any rules as final. As interpretation depends upon the synthesis of all factors, no rule can be applied rigidly. Vedic Astrology should not be approached with a sense of awe but one of experiment and creativity. It is a system with much to offer but also with much inertia to be removed. The combinations that produced certain results in medieval India cannot be expected to do the same thing today in America, though a certain type of energy should remain the same. Accept only what of this system really works.

 

\subsection{HOUSE CONFIGURATIONS}
 

Here we examine the broad configurations of planets in their orientation relative to the houses. These are involved in many special Vedic astrology yogas or combinations.

 

Planets vary whether they predominate in angular (1, 4, 7, 10), succedent (2, 5, 8, 11) or cadent houses (3, 6, 9, 12). Angular planets give action, change, energy and motivation but can be impulsive and dominating. Succedent planets are good for communication, commerce, financial gains and expression, but may lack in leadership or energetic qualities. Planets in cadent houses are better for the mind, creative expression and spirituality but may not be successful in the outer world and may stay behind the scences.
In upachaya and apachaya houses. A prime principle that we must always reiterate. Natural malefics do best in upachaya or increasing houses, 3, 6, 11 and sometimes 10. Benefics in apachaya or decreasing houses lose their strength over time (1, 2, 4, 5, 7, 8).
Planets predominating in houses of dharma (1, 5, 9), artha (2, 6, 10), kama (3,  7 , 11)and moksha (4, 8, 12), strengthen these aspects of life.
 

The combinations listed below are among the common or general yogas of Vedic Astrology. Many Western astrologers consider these groupings as well.

\subsubsection{ALL OR MOST PLANETS ABOVE THE HORIZON (HOUSES 7-1)}

 

This creates an outgoing, worldly or political nature because the planets are in the visible half of the chart. The individual will be social or work oriented, communicative but possibly superficial. They will easily reveal themselves and make their lives a public affair. It will be difficult for them to keep secrets or to be uninfluenced by their social environments.

They may be evolved spiritually but will express their spirituality through work or communication. They will be drawn into public activity, even though their basic nature (as revealed by the signs) may be more private.

 

\subsubsection{ALL OR MOST PLANETS BELOW THE HORIZON (HOUSES 1-7)}

 

This creates an introverted or inward nature because the planets are in the invisible half of the chart. The individual will be indrawn, unexpressive or not revealing of their true feelings or motivations. They will hide things, keep things close to themselves and can be hard to understand. More evolved types will be drawn towards meditation.

They can be successful in the outer world but they will work behind the scenes or through other people.

 

\subsubsection{ALL OR MOST PLANETS IN THE EASTERN HALF OF THE CHART}

 

The individual will be more self-motivated, individualistic and self-expressive because planets are in the eastern or personal side of the chart. They may be selfish, egoistic or simply individualistic. They may talk much about themselves and be impulsive in their behavior. Relationships may prove difficult for them.

 

\subsubsection{ALL OR MOST PLANETS IN THE WESTERN HALF OF THE CHART}

 

The predomination of planets in the relationship side of the chart will make the person relationship oriented. They may define themselves by the people with whom they are in partnership. Self-identity and self-worth may be difficult to establish. They will find it difficult to stand alone. However, they can be aggressive in relationship.

 

\subsubsection{OPPOSITE HOUSES}
 

Duality is the essence of life. These prime dualities of life are outlined in the relationship between the houses, between each house and its opposite. Planets in opposite houses reflect these prime issues. The Rahu-Ketu axis, which is always in opposite houses, does this as well.

 

\subsubsubsection{HOUSES 1 AND 7}

 

These are the houses of self and other. Planets here show issues in relationship and self-identity. When the planets involved are of harmonious nature they show harmony of self and partner. When the planets are of conflicting nature they show conflict between the self and other, between personal drives and relationship needs.

 

\subsubsubsection{HOUSES 2 AND 8}

 

These are the houses of personal and collective resources. They show material and financial issues. Badly disposed, they show the loss of personal resources to groups, organizations or to society at large. Well disposed, they show both personal and collective gains.

These are also houses of personal and collective expression. They can show conflict of expression with the partner. They can also show the ability to communicate on both personal and collective levels. Afflicted, they can give health problems because the eighth is the house of death and the second a death-dealing (maraka) house.

 

\subsubsubsection{HOUSES 3 AND 9}

 

These are houses of dharma, vocation and motivation. The third shows our basic energy, drive and curiosity, what we like to do of our own accord when not seeking any collective goal. The ninth shows the principles that we stand for in the world. It shows what we think is our duty, what we feel a responsibility or calling to do for the benefit of the world. The third is what we like to do and represents following our own personal motivations.

Issues between these houses center on the conflict between the personal and social usage of our energies, personal interests and enthusiasms versus collective and spiritual responsibilities. Harmony between these two houses shows a harmony of desire and duty, of personal will and spiritual purpose.

 

\subsubsubsection{HOUSES 4 AND 10}

 

These are houses of the private life versus the public life, of personal identity versus social roles. Here the issues of individual needs versus social responsibility come out. Some of these are similar issues to the third and ninth houses, but for those houses it is a question of values and the usage of energy; for these houses it is a question of home life and work life. Conflicts between the domestic and the work sphere are shown here. Also shown is the relationship between our hearts desire (fourth house) and its public realization (tenth house). It shows the relationship between desire (fourth house) and karma (tenth house).

 

\subsubsubsection{HOUSES 5 AND 11}

 

These are both houses of gain, expansion and creativity, the former in the personal sphere, the latter in the collective sphere. Issues here are those of creativity and self-expression.

The fifth house shows our intelligence and creativity. The eleventh shows how we communicate and share it on a group or collective level. A conflict between these two houses shows a conflict between personal and public aspirations for the person.

 

\subsubsubsection{HOUSES 6 AND 12}

 

These are houses of negation and sorrow. The sixth house shows factors that negate our personal well-being, the twelfth house those which negate our social well-being. Issues between these houses relate to disease, enmity, sorrow and loss. On a higher level, these are the issues of service and surrender. This is a very sensitive axis and malefic influences on this axis can cause much harm, though malefics in the sixth house itself can be beneficial.

 

\subsection{SIGN CONFIGURATIONS}
 

Here we will examine the orientation of planets in various types of signs. These constitute some primary sign based planetary yogas.

 

\subsubsection{MANY PLANETS IN ONE SIGN}

This shows a very concentrated and sometimes unbalanced person. If malefics are involved, they become stronger in such conditions. Too close proximity of planets muddies the planetary rays. This combination can cause renunciation or loss regarding the domain of life represented by the house/sign involved.

 

\subsubsection{MANY PLANETS IN EVEN SIGNS}

This gives more feminine, receptive, nurturing and passive qualities.

 

\subsubsection{MANY PLANETS IN ODD SIGNS}

This gives more masculine, active, assertive and aggressive qualities.

 

\subsubsection{MOST PLANETS IN THE FIRST HALF OF THE ZODIAC}

This keeps us in a movement of personal projection, initiation and manifestation as the first half of the zodiac is the personal sphere. The direction of energy is from within to without.

 

\subsubsection{MOST PLANETS IN THE SECOND HALF OF THE ZODIAC}

This keeps us in a movement of completion, universalization and communication as the second half of the zodiac is the collective sphere. The direction of activity is from without back to within.

 

\subsubsection{PLANETS AND QUARTERS OF THE ZODIAC}

\subsubsubsection{MOST PLANETS IN THE FIRST QUARTER OF THE ZODIAC (SIGNS 1-3)}

This gives a self-motivated but often self-centered person, expressive and dynamic but perhaps lacking the capacity to hesitate or observe himself. The first quarter of the zodiac relates to our personal drives and urges.

 

\subsubsubsection{MOST PLANETS IN THE SECOND QUARTER OF THE ZODIAC (SIGNS 4-6)}

This gives a strong sense of self, work and identity. The individual will be prominent in their work and create an issue of their personality. The second quarter of the zodiac relates to our mind and character.

 

\subsubsubsection{MOST PLANETS IN THE THIRD QUARTER OF THE ZODIAC (SIGNS 7-9)}

This gives a strong sense of relationship, emotion and principle. The individual will project themselves very strongly into relationship and social goals as the third quarter of the zodiac relates to relationship and emotion.

 

\subsubsubsection{MOST PLANETS IN THE FOURTH QUARTER OF THE ZODIAC (SIGNS 10-12)}

This gives a more social or collective orientation but a tendency towards diffusion and loss of self-control. The individual is moving from the collective field toward immersion in himself. The fourth quarter of the zodiac is the collective realm.

 

\subsubsection{PLANETS AND QUALTIES OF SIGNS}

\subsubsection{MOST PLANETS IN MOVEABLE OR CARDINAL SIGNS}

This makes a person move and change easily both inwardly and outwardly, while maintaining a certain focus and direction, will and motivation. There is usually a high seeking of achievement.

 

\subsubsubsection{MOST PLANETS IN FIXED SIGNS}

This makes a person fixed both inwardly and outwardly, slow to change but persistent in what they do. On the positive side this gives endurance and stability, on the negative side it can cause stagnation and inertia.

 

\subsubsubsection{MOST PLANETS IN MUTABLE OR DUAL SIGNS}

This makes a person shifting, indecisive or flexible. They go back and forth and cannot hold to a fixed position or one consistent direction of movement. This is good for adaptability of mind but can create some ambivalence and weakness in the character. While the mind is strong, the will can be weak.

(These three qualities are described more in the Astrology of the Seers (89-94, 109–112), as are the four elements below. Here we are summarizing the main points.)

 

\subsubsection{PLANETS AND ELEMENTS OF SIGNS}

\subsubsubsection{MANY PLANETS IN EARTH SIGNS}

This keeps one in the field of earth manifestation, the body, making things for the practical world, outer achievements, or working with nature and the Earth element, which can extend to the physical body. Doshic factors are less evident than with the other elements.

 

\subsubsubsection{MANY PLANETS IN WATER SIGNS}

This keeps one in a sphere of water, which equates to emotional and intuitive manifestations, a strong feeling nature and regard for others. It can also promote Kapha dosha in body and mind.

 

\subsubsubsection{MANY PLANETS IN FIRE SIGNS}

This keeps one in the field of fire either outwardly as fiery occupations or inwardly as a perceptive or willful personality. It can also promote Pitta dosha in body and mind.

 

\subsubsubsection{MANY PLANETS IN AIR SIGNS}

This keeps one in the field of air, generally communication, intellectual, or social activities. It can also promote Vata dosha in body and mind.

 