\subsection{6. DIVISIONAL OR HARMONIC CHARTS} (Astrology of the Seers, Chapter 10, Pages 161-172)
 

Normally western astrology only divides the zodiac into twelve parts based upon the twelve signs of the zodiac. In addition, Vedic astrology divides the zodiac into twenty-seven parts based upon the twenty seven lunar Nakshatras.

Yet there are also numerous different divisional (amsha) charts in Vedic astrology, calculated by how we can further divide up the signs. Each divisional chart, also called harmonic chart, has its specific meaning and application. There is the saptavarga or the division of seven important divisional charts.  There is also the list of sixteen divisional charts used.

Common divisions are twofold (hora), threefold (drekkana), fourfold (chaturtamsha), sevenfold (saptamsha), ninefold (navamsha), tenfold (dashamsha) and twelvefold (dwadashamsha).

 

Note that an accurate birth time is necessary to be certain of positions for divisional charts, particularly for their ascendants, though the positions of other planets will not so likely change. Even for the Navamsha, we may not be able to trust the Navamsha Ascendant, which changes every fifteen minutes or so, unless we are certain of the birth chart time to within a few minutes. 

 

Divisional charts vary in importance for in fine-tuning in chart interpretation. For some examples, the Drekkana or harmonic third is important for brothers and sisters, friends, prana and vitality, much like the third house in the birth chart, and has some general value in examining the chart as a whole, particularly the drekkana of the ascendent. The Dashamsha or tenth harmonic chart chart is used along with and much like the tenth house in the birth chart for determining career and success in life, but does not have broader connections.

The Navamsha or divisional ninth chart is most important divisioal chart for all general indications, for the future, for relationship, and for spiritual or ninth house indications. It has a special connection with the Nakshatras, with each Navamsha sign reflecting one-quarter or pada of the Nakshatra.
If the birth time is approximate or the ascendant is within a degree of the shift between one Navamsha and another, read the Navamsha chart with the Rashi Ascendant sign as its Ascendant. This reflects the old system of chart formation in which the positions of the Navamsha were included in the corner of the Rashi chart signs. In other words, you can benefit from looking at the Navamsha even when the birth time may not be exact.
 

Always look at the Navamsha chart at least along with the basic sign (Rashi) chart for primary chart examination. Some Vedic astrologers use the Drekkana or third divisional chart in the same general manner. The other divisional charts have their specific indications, like the Divisional Second or Hora as wealth, the Divisional Third or Drekkana as brothers and sisters. The Divisional Fourth or Chaturtamsha as home. The Divisonal Seventh or Saptamsha as children, the Divisional Tenth or Dashamsha as career. The Divisional Twelfth or Dwadashamsha as parents and past history. We are just noting these divisional charts here and will examine them in greater detail later in the course.

 

Divisional charts also reflect subtler planetary aspects that are not visible from the basic sign chart. Aspects in divisional charts have their place and should be examined carefully, though not all Vedic astrologers do this. New Vedic astrology software can calculate the degree and minute of the planets within the Navamsha. This is interesting but was not part of traditional Vedic astrology and is more a secondary concern.

 

\subsection{7. PLANETARY PERIODS OR DASHAS AND FACTORS OF ASTROLOGICAL TIMING} (Astrology of the Seers, Chapter 11, Pages 173-182)
 

Vedic astrology has its own special set of planetary periods called dashas that is absent in western astrology. There are many types of these planetary periods based upon the Moon’s location in the birth chart, various signs or other special calculations, including Jaimini Chara Dasha and Yogini Dasha.

 

Yet most important of the dashas is the Vimshottari (120 year cycle) dasha system. It is based upon the Nakshatra where the Moon is located in the birth chart, and the planetary ruler of that Nakshatra. In this system, each planet rules over a certain number of years in the 120 year cycle from the Sun (six years), Moon (ten years), Mars (seven years), Rahu (eighteen years), Jupiter (sixteen years), Saturn (nineteen years), Mercury (seventeen years), Ketu (seven years), Venus (twenty years). This cycle works very well, though the specific rationale behind it is not clearly known.

 

In fact, the key to the success of Vedic astrology in timing and prediction rests largely on the incredible accuracy of Vimshottari dasha. Dashas are very important in all chart readings as they reflect the current concerns of the client, which are usually the main reasons they are seeking a consultation. Without relevance to the Dashas other indications in the chart may not be able to manifest.

 

Major planetary periods (Mahadashas)are further divided into minor planetary periods (Bhuktis, Antardashas) based upon the same ratio of the periods ruled by each planet. Vedic astrology software makes these calculations easy and automatic. Minor planetary periods can similarly be broken down into further shorter tertiary periods, though theese depend upon a very accurate birth time, to be certain of their timing. Generally we seldom go beyond the minor period or bhukti, and beyond that point look more to transits.

 

The calculation of different Ayanamshas can change the calculation of planetary periods by some months. Note the Ayanamsha used if you find such variations in different charts. Some Vedic astrologers calculate Dashas by 360 day years instead of the normal years. This also brings in some slight variations.

 

There are certain principles for determining favorable planetary periods that we will discuss later in the course. House rulership of a given planet in the chart is most important in determining whether planetary periods are favorable or unfavorable.

 

Periods of benefic house rulers like the ruler of the ninth house or the fifth house are generally good, for example. Periods of the eighth, sixth or twelfth lords, malefic house rulers, on the other hand, can be difficult. But this is a complex matter requiring examining the chart as a whole, and planets may rule more than one house.

 

The period of the Ascendant Lord is crucial for the overall manifestation of the chart. A strong Ascendant and Ascendant Lord will bring many good results during its period, particularly for career and purpose. A weak Ascendant and Ascendant Lord can be bring disease or difficulties.

 

\subsubsection{GOCHARA OR TRANSITS}

 

Transits, called Gochara in Sanskrit, are current movements of the planets. These can be examined in their own right, like the difficulties possible at times of malefic Mars/Saturn conjunctions. More importantly, they can be applied against positions given in the birth or natal chart, like Saturn transits to the natal Moon (Sade Sathi) which can be very stressful.

 

Transits are very important at in terms of Muhurta, mundane, collective or political astrology, where their influence is prominent in world affairs, national and yearly charts. This is particularly true of transits of the distant planets, Jupiter and Saturn, and the lunar nodes Rahu and Ketu that are involved in eclipses of the Sun and Moon.

 

Relative  to the individual chart, transits modify what may happen during a given planetary period for a person, altering to some degree the results of the planetary period. While certain result may be indicated during a planetary period or subperiod, a special transit may either trigger it or obstruct it.

 

\subsubsection{PRASHNA}

 

Prashna, horary or question based astrology consists of erecting a chart for the specific moment and place that a question is asked. The Prashna chart is interpreted by rules similar to the birth chart, but in some ways different. It may be used when the birth time is not known or if a very specific question is to be examined. We will study it more in Part III of the course, but Prashna will not be a prime or direct part of this course as it forms a subject in its own right that can be quite complicated.

 

\subsubsection{MUHURTA, ASTROLOGICAL FORECASTING AND MUNDANE ASTROLOGY}

 

Muhurta, astrological forecasting and mundane astrology consists of looking for favorable times to perform various actions. Many Vedic astrologers do a lot of such Muhurta work, examining favorable times for marriage, moves, travel, starting a business, mantra initiation and so on.

 

Muhurta is an important and integral part of any Vedic astrology practice. Muhurta is connected to the Panchanga or Vedic soli-lunar calendar, in which the position of the Moon in various Nakshatras is most important, along with its phases (Tithis). Dashas and transits are only secondary.

 

Mundane astrology is related to Muhurta at a collective level and consists of examining charts of countries or their leaders to see the effect of national and world affairs, notably elections. We will examine Muhurta and mundane astrology in detail in Part III of the course, both in terms of Panchanga and Ashtakavarga.

 

\subsection{8. ASTROLOGY AND AYURVEDA, REMEDIAL MEASURES} (Astrology of the Seers, Chapter 12, Pages 183-196)


Ayurvedic Astrology is the more specific theme of this course and is dealt with in detail in the second section of the course, where it is the subject matter of a number of lessons. Here we are only introducing the basics. It is addressed in our book Ayurvedic Astrology that will be referred to later.

For Ayurvedic Astrology, we need to first learn the basic qualities of the three Doshas or biological humors of Vata, Pitta and Kapha and their planetary connections. The doshas are similar to the elements in their qualities and each has a primary elemental connection. Each dosha has planetary connections to various degrees.
Vata (Air) – Saturn, Mercury, Rahu\\
Pitta (Fire) – Sun, Mars, Ketu\\
Kapha (Water) – Moon, Venus, Jupiter\\
Yet planets may reflect more than one dosha, with Jupiter as Kapha and Pitta, Venus as Kapha and Vata, and the Moon varying by dosha as to whether it is waxing and waning and in which sign it is located in

You can supplement your study of Ayurveda by examining various books on the subject. These include the author’s books Ayurvedic Healing, A Comprehensive Guide and The Yoga of Herbs. 

 

The three Doshas also relate to the signs of the zodiac and their respective elements. Earth and Water signs relate more to Kapha dosha. Fire signs relate more to Pitta dosha. Air signs relate more to Vata dosha. But planetary factors relative to the doshas are more important than these sign connections, and various admixtures of influences occur, including relative to the planets exalted in these signs.

Generally the doshic influences of the first six signs from Aries to Virgo are more clear, while those from Libra to Pisces are more mixed as we will discuss in Part III of the course.

Fiery Mars ruled fire-based Aries is Pitta, Venus ruled earth-based Taurus is Kapha, airy Mercury ruled air-based Gemini is Vata, and watery Moon-ruled water sign Cancer is Kapha.
Fiery Sun ruled fire-based Leo is Pitta, airy Mercury ruled earth-based Virgo is Vata, Venus ruled air-based Libra is Vata, fiery Mars ruled water-based Scorpio is Pitta.
Jupiter ruled fire-based Sagittarius is Pitta, Saturn ruled earth-based Capricorn is Vata, Saturn ruled air-based Aquarius is Vata, and Jupiter ruled water-based Pisces is Kapha.
 

The determination of physical constitution can be difficult or complicated from the astrological standpoint as it includes a number of factors, mainly relative to the ascendant that represents the physical body and the sixth house of disease, particularly the dominant planet in the chart or the planetary type. That will be taken up later as well.

 

Certain planets and houses are particularly strong to cause disease, notably malefic planets Saturn, Mars, Rahu and Ketu. The sixth house and sixth sign (Virgo) important in terms of health indications as well as other difficult houses like the eighth and the twelfth. Diseases generally manifest in periods of these disease-causing planets or their transits of significant positions in the chart, or during periods of the ascendant lord if it is weak in the chart.

 

\subsubsection{ASTROLOGY AND PSYCHOLOGY} (Astrology of the Seers, Chapter 13, Pages 189-196)

 

Overall we relate Astrology and Psychology to both Ayurvedic Psychology and Yoga Psychology as well as to Vedantic philosophy, which are interrelated. (Note the authors book Ayurveda and the Mind for understanding Ayurvedic psychology).

The sequence of planets affects  the unfoldment of the karmas of the reincarnating soul of Jivatman. Each planet has psychological implications, as do signs and houses. But this is according to the yogic or Vedic view of psychology, which is very different from that of modern psychology, and extends beyond the physical to the subtle, causal and transcendent realms.

 

By natural status, the Sun is the self, soul or ego, Moon is mind and emotions, Mercury is speech and intellect, Venus is desire, love and affection, Mars is will, motivation and aggression, Jupiter is wisdom, dharma and creativity, Saturn is isolation, withdrawal or negation, Rahu is openness to psychic influences and expansion of collective karma, Ketu is psychic insight, doubt, contraction and withdrawal.

Relative to the houses and their significators, the fourth house  and the Moon is emotional mind, the fifth house and Jupiter is intelligence or buddhi, the first house and Sun is self or ego, and the ninth house and Jupiter is the soul or Jivatman.

 

Vedic astrology is an important psychological tool, which is one of its main and most significant usages. This topic is specially  examined in the Workbook at an advanced level of study, as well as in the lesson on Planetary Types. Here just introduce yourself to the topic.

 

\subsubsection{REMEDIAL MEASURES/ TREATMENT METHODS }(Astrology of the Seers, Chapter 14, Pages 199-233)

 

This topic also relates more to the second part of the course. We will only go through it briefly here. Don’t worry about the details at this point, Vedic astrological is not only predictive and interpretative but provides practical methods called upayas to help us promote positive planetary influence and reduct those that are negative.

 

There are various methods through which weak planets can be strengthened, using colors gems, herbs, aromas, and foods, as well as rituals (flower offerings and fire offerings), mantras and meditation. It can extend to actions of a charitable nature, service, pilgrimage or visiting temples.

The use of gems is central in this regard, with primary gems like ruby for the Sun, pearl for the Moon, red coral for Mars, emerald for Mercury, yellow sapphire for Jupiter, diamond for Venus. blue sapphire for Saturn, hessonite garnet for Rahu and cat’s eye for Ketu .

 

Generally, we use gems to strengthen the Ascendant lord or other benefic planets, like the ruler of the ninth house or the fifth house, when afflicted in the chart. We are hesitant in strengthening natural or temporal malefics through the use of gemstones, like Saturn. We present the rules for determining this in the second section of the course, which can be complicated.

Some people think that a Vedic astrological reading requires a gem recommendation. Actually, yogic and spiritual methods like mantras, yantras and ritual can be more helpful and less expensive than gems, though these do require more effort and time to sustain. While some charts may have weak benefic planets like the ascendent lord that can clearly benefit by a gemstone, other charts have mixed planetary influences for which gemstones for planets may not be of such clear or lasting value.

 

\subsubsection{WORSHIP AND MEDITATION ON THE PLANETS} (Astrology of the Seers, Chapter 15, Pages 227-233)

 

We will only introduce this topic here as it is a main topic of Part II of the course.Yogic and spiritual methods are very important in strengthening the planets and can be better than gems, but require regular practice and devotion to empower.

 

Mantra is most important in this regard, with Vedic astrology having several sets of of name and bija mantras for each planet. Each planet has its ruling deities. It also has its symbols and yantras relate to the planets. This is also covered the book Ayurvedic Astrology and will be detailed later in the course, particularly Part II that has several lessons devoted to this and an outlining of mantras for planets, signs and Nakshatras.

 

\subsubsection{THE ASTROLOGY OF THE SEERS, EXAMPLE CHARTS} (Astrology of the Seers, Chapter 16, 237-268)


This section of the book consists of the examination of charts of various famous people, whose life details and timings are known to us. You should begin examining such charts as you move through the course.progress with the course.  Try to study one chart a week. We will not examine the specific charts in this lesson as they are explained in the book, along with their necessary calculations. Please read that material at your convenience.

Note that we will examine additional instances of chart examination in the Course Workbook that supplements the later lessons of Part I of the course, as well as Lesson 7 on Horoscope Judgement. 

\subsubsection{COURSE WORKBOOOK READINGS}
Make sure to study the Course Workbook 1. Reading North and South Indian Charts, so you know the principles behind reading charts and complete that lesson before looking at any charts!
Make sure you know the mechanics of reading both North and South India charts and how to determine planets, signs, houses, aspects and planetary periods from them.
In addition, begin collecting astrological charts and casting them in the Vedic manner. Start with your own chart and those of your friends and family whose details are known to you. You should get a Vedic astrology software program for this purpose. You can examine the charts of famous or well known people as well. You should continue to look at such charts along with your study of the course overall.

 

First note the mechanics of a chart with the positions of planets, signs and houses, as well as the aspects involved. Note how a person’s planetary periods is reflected in the development of their lives and careers. See if you can determine their dominant planets. See how astrology mirrors their karmas.

 

You can also find books on Vedic astrology that contain charts of famous people, or you can find this information on-line. Or you can get a book of famous charts from western astrology, and transpose them into the Vedic system by subtracting the appropriate Ayanamsha, and then examine the basic or Rashi chart according to major positions. Use your own Vedic astrology computer program to look at and collect birth charts.

 

Learn to look at the daily charts as well, noting the current planetary positions and their relationship to different natal charts or to social and political events in the outer world.  Be aware of the position of the Moon and Sun in the zodiac, as well as outer planets like Mars, Saturn and Jupiter that move more slowly. Note the positions of eclipses and the lunar noes of Rahu and Ketu. Most importantly look at your own birth chart regularly as you learn more about Vedic astrology in the course, and see how Vedic astrology principles are mirrored in your life and character. Try to form a relationship with the planets through their names, mantras and deities.

 

Go out and look at the stars and try to make a connection with them. Note the path of the zodiac, with its twelve signs, and of the Milky Way that dominates the night sky. As an amateur astronomer of many years, I can identify quickly over two hundred deep sky objects with a small telescope that remain a source of great inspiration. Besides planets are wonderful star clusters, nebula, globular clusters and double stars. Remember that the universe dwells within you and the distant stars are part of your own higher mind that can guide. Never forget the role of intuition and meditative insight in astrology.

 

Remember to spend as much time looking at charts as examining the course material or reading books on Vedic astrology. In summary, it is important to read the book Astrology of the Seers, but also remember that much of what is presented in the book will make more sense as you move through the course.

 

\subsubsection{Table of the Planets}
Tabulated from Various Classical Sources (variations do exist!)
Summarizes the main points about the planets

Sun
\begin{center}
\begin{tabular}{ l l l l}
Name	&Surya	&Guna	&Sattva \\
Nature	&Malefic	&Element	&Fire\\
Dosha	&Pitta	&Color	&Red\\
Caste	&Kshatriya	&Sex	&Male\\
Taste	&Pungent	&Tissue	&Bones\\
Abode	&Temple	&Time Period	&Half-year\\
Direction	&East	&Relationship	&Father\\
Psychological	&Ego, Soul	&Physical	&Heart\\
Role	&King	&Deity	&Agni/ Shiva\\
\end{tabular}
\end{center}

Moon
\begin{center}
\begin{tabular}{ l l l l}
Name	&Chandra	&Guna	&Sattva\\
Nature	&Benefic	&Element	&Water\\
Dosha	&Kapha	&Color	&White\\
Caste	&Vaishya	&Sex	&Female\\
Taste	&Salty	&Tissue	&Blood\\
Abode	&Watery Areas	&Time Period	&48 Minutes\\
Direction	&Northwest	&Relationship	&Mother\\
Psychological	&Mind, Emotions	&Physical	&Stomach\\
Role	&Queen	&Deity	&Soma/ Devi\\
 \end{tabular}
\end{center}

Mars
\begin{center}
\begin{tabular}{ l l l l}
Name	&Mangala, Kuja	&Guna	&Tamas\\
Nature	&Malefic	&Element	&Fire\\
Dosha	&Pitta	&Color	&Red\\
Caste	&Kshatriya	&Sex	&Male\\
Taste	&Bitter	&Tissue	Marrow\\
Abode	&Weapons Room	&Time Period	&Day\\
Direction	&South	&Relationship	&Brothers, Siblings\\
Psychological	&Will, Vitality	&Physical	&Small Intestine\\
Role	&Army Chief	&Deity	&Skanda\\
 \end{tabular}
\end{center}

Mercury
\begin{center}
\begin{tabular}{ l l l l}
Name	&Budha	&Guna	&Rajas\\
Nature	&Benefic	&Element	&Earth\\
Dosha	&Vata	&Color	&Green\\
Caste	&Vaishya	&Sex	&Child, neutral\\
Taste	&Mixed	&Tissue	&Skin\\
Abode	&Sports Ground	&Time Period	&Season\\
Direction	&North	&Relationship	&Friends\\
Psychological	&Intellect, Speech	&Physical	Brain\\
Role	&Prince	&Deity	&Vishnu\\
 \end{tabular}
\end{center}

Jupiter
\begin{center}
\begin{tabular}{ l l l l}
Name	&Brihaspati	&Guna	&Sattva\\
Nature	&Benefic	&Element	&Ether\\
Dosha	&Kapha	&Color	&Yellow\\
Caste	&Brahmin	&Sex	&Male\\
Taste	&Sweet	&Tissue	&Fat\\
Abode	&Treasure House	&Time Period	&Month\\
Direction	&Northeast	&Relationship	&Husband, Guru\\
Psychological	&Higher Mind, Conscience	&Physical	&Liver\\
Role	&Minister	&Deity	&Indra\\
 \end{tabular}
\end{center}

Venus
\begin{center}
\begin{tabular}{ l l l l}
Name	&Shukra	&Guna	&Rajas\\
Nature	&Benefic	&Element	&Water\\
Dosha	&Kapha	&Color	&Variegated\\
Caste	&Brahmin	&Sex	&Female\\
Taste	&Sour	&Tissue	&Reproductive\\
Abode	&Bedroom	&Time Period	&Fortnight\\
Direction	&Southeast	&Relationship	&Wife\\
Psychological	&Desire, Love	&Physical	&Sex Organs\\
Role	&Minister	&Deity	&Indrani\\
 \end{tabular}
\end{center}

Saturn
\begin{center}
\begin{tabular}{ l l l l}
Name	&Shani	&Guna	&Tamas\\
Nature	&Malefic	&Element	&Air\\
Dosha	&Vata	&Color	&Blue, Black\\
Caste	&Shudra	&Sex	&Neutral\\
Taste	&Astringent	&Tissue	&Muscles\\
Abode	&Unclean Places	&Time Period	&Year\\
Direction	&West	&Relationship	&Grandfather\\
Psychological	&Suffering, Ego	&Physical	&Colon\\
Role	&Servant	&Deity	&Brahma\\
 \end{tabular}
\end{center}

Rahu
\begin{center}
\begin{tabular}{ l l l l}
Name	&Rahu	&Guna	&Tamas\\
Nature	&Malefic	&Element	&Air\\
Dosha	&Vata	&Color	&Smoky\\
Caste	&Outcaste	&Sex	&Female\\
Taste	&Poisons	&Tissue	\\
Abode	&Wandering	&Time Period	&Eclipses\\
Direction	&Southwest	&Relationship	&Distant Relatives, Foreigners\\
Psychological	&Illusion (Maya)	&Physical	&\\
Role	&Army	&Deity	&Durga\\
 \end{tabular}
\end{center}

Ketu
\begin{center}
\begin{tabular}{ l l l l}
Name	&Ketu	&Guna	&Tamas\\
Nature	&Malefic	&Element	&Fire\\
Dosha	&Pitta	&Color	&Smoky\\
Caste	&Mixed Caste	&Sex	&Male\\
Taste	&Poisons	&Tissue	\\
Abode	&Hiding	&Time Period	&Eclipses\\
Direction	&Southwest	&Relationship	&Ancestors\\
Psychological	&Secret Knowledge	&Physical	&\\
Role	&Army	  &Deity	&Ganesha\\
\end{tabular}
\end{center}