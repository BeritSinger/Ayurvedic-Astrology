\section{SHADBALA: THE SIX MEANS OF DETERMINING PLANETARY STRENGTHS AND WEAKNESSES}
 

In this lesson, we will examine the most complex system found in Vedic, or for that matter, any system of astrology for determining the power of the planets in the chart. This information is largely for reference and need not be understood in detail, particularly in regard to calculation which is offered by Vedic astrology software and can be quite complex.

 

Most astrologers like to calculate. Many would like a system of calculations that would guarantee the rightness of all their predictions. All that we would have to do would be to run a program and we would know the strength and weakness of planets and exactly what they will do in each chart. However, life is very complicated. The influence of the planets can come on several different levels and we have some freedom as to how we use these influences. Hence, all such mechanical systems should be applied with discretion. They are a good place to start but cannot be relied on entirely. The information below is mainly for reference value.

 



 

Shadbala is an elaborate system of computations to aid in determining planetary strengths and weaknesses. It is perhaps the most sophisticated and detailed of all such astrological systems. It requires much astronomical knowledge to compute accurately. While Shadbala is not necessary to give an accurate astrological reading, it is helpful information to have. As the calculations are very complex and would require much skill and time to do oneself, various computer programs are preferable.

\subsection{\textbf{Audio Overview by Dr. David Frawley on Shadbala}}

Once computed, all the different types of Shadbala are averaged out to determine the general planetary strength or weakness. Generally, it is considered that if a planet has a Shadbala of 1.0 or more it has the power to give the effects of its position in the chart. If its Shadbala is less than 1, it is considered weak and may cause difficulty. Shadbalas occur commonly up to 1.3 and uncommonly as high as 1.8. They occur commonly as low as .8 and uncommonly as low as .6. They average around 1.1 or 1.2.

 

Some astrologers, however, consider only the Rupa total on the Shadbala. In this way, they like to discover what is the strongest and what is the weakest planet in the chart in terms of Shadbala prior to any special divisions of the amount.
Other astrologers look more at the particular factors in Shadbala, like exaltation strength (uccha bala) and are not so concerned about the total figures for the planet.
 

I am more of the second opinion. Shadbala is not absolute. If a planet has a high Shadbala, it does not necessarily mean it will always give good results or if it has a low one, it does not mean that it will give bad results. Other factors have to be taken into consideration (see Ishta and Kashta Phala, Good and Bad Results of Planets). Similarly, if a planet has only average Shadbala but is strong in the chart in terms of position and aspect, it can still act in a very potent manner. Shadbala shows the basic strength and weakness of planets but we have to look back to the chart for what they are empowered to do.

 

Shadbala does not appear to adequately consider planetary aspects. It does read aspectual strength into its factors but this does not count for much. It seldom makes a difference of more than 5%. My experience has been that aspects are probably more important than any Shadbala factors.

 

I have given additional criticism to Shadbala as its factors sometimes cancel each other out. While its calculations are each useful to consider in themselves, it is not always useful to average them all out. Astrology requires qualitative judgments and these cannot always be reduced to mere quantitative calculations. There is still no substitute for insight and experience.

 

While all the factors of Shadbala are important and should be considered to some degree (like whether a planet is strong by location, position or aspect), the present system of Shadbala for doing this may not be entirely accurate. I would like to see Shadbala researched and revamped into a more workable system. Hence, I have presented this material primarily for reference, regarding the individual factors as more important than their overall average. These calculations provide us much information about how to view the planets. Otherwise a simple examination of the chart relative to usual factors of sign, house, aspect and yoga is much more important.

 

\subsection{MAIN FACTORS OF SHADBALA}

 

The six factors of Shadbala are:

\begin{enumerate}
\item[*] Positional Strength (Sthana Bala)
\item[*] Directional Strength (Dig Bala)
\item[*] Temporal Strength (Kala Bala)
\item[*] Motional Strength (Chesta Bala)
\item[*] Natural Strength (Naisargika Bala)
\item[*] Aspectual Strength (Drik Bala)
\end{enumerate} 

 


\subsection{1. POSITIONAL STRENGTH}
This consists of five factors:

 

\begin{enumerate}
\item[*] Exaltation Strength (Ucha Bala)
\item[*] Divisional Strength (Saptavargaja Bala)
\item[*] Odd-Even Sign Strength (Ojayugmarasyamsa Bala)
\item[*] Angular Strength (Kendra Bala)
\item[*] Decanate Strength (Drekkana Bala)
\end{enumerate} 

\subsubsection{A. EXALTATION STRENGTH}


 

Exaltation Strength is always important and should be considered even when Shadbala is not calculated. Planets are always stronger at exaltation and weakest at their fall. This is similar to the Moon being stronger full than when new. The determination of Exaltation Strength is simple.

A planet is given 60 points of value when at the degree of exaltation and 0 points of value when at the degree of fall.
For every three degrees away from exaltation, it loses one point and for every three degrees away from fall it increases one point.
 

Exaltation Strength does have its limitations. First, it does not consider how much the fall of a planet may be canceled (and if the fall can be canceled, the exaltation should be canceled by opposite factors). Second, it does not consider the sign location of planets. Planets do not lose their strength in a uniform manner between exaltation and fall, as they gain strength in the signs they rule.

 

For example, the Sun in Leo has a low exaltation strength as it is nearer to its fall in Libra than its exaltation in Aries, but it is still strong being in its own sign. In my consideration of planets I like to consider both exaltation and sign residency strength together. If we consider them separately, as Shadbala does, they may cancel each other out. For example, the Sun though strong by residency in Leo is weak by exaltation strength, hence neutral. However, I think it should still be generally considered as strong as its position in its own house cancels the weakness in exaltation strength.

 

\subsubsection{B. DIVISIONAL STRENGTH}

 

This is calculated relative to the seven divisional charts of the divisional first, second, third, seventh, ninth, twelfth and thirtieth charts. It follows the same rules of friendship and enmity as that of the birthchart, giving each a certain point total.

 

Divisional Strength of Planets

\begin{center}
\begin{tabular}{ l l l l}
Own sign	& 30 points.                 \\
Great friend	 &22.5 points                \\
Friend	 &15 points                \\
Neutral sign	 &7.5 points                \\
Inimical sign	 &3.75 points                \\
Great enemy	 &1.875 points                \\
\end{tabular}
\end{center}
 

It is necessary to determine the planetary friendships and enmities for all seven divisional charts to do this. There is the special consideration that a planet in its Mulatrikona division in the basic birth or divisional first (rashi) chart would get 45 points. Otherwise, Mulatrikona is not considered.

 

This is an important consideration generally speaking and is always worthy of noting by itself.

 

\subsubsection{C. ODD OR EVEN SIGN STRENGTH}

 

Planets gain strength whether in odd or even signs in the Rashi (divisional first) and Navamsha (divisional ninth).

Sun, Mars, Jupiter, Mercury and Saturn do better in odd-numbered signs
Moon and Venus do better in even-numbered signs
 

Hence, most planets gain strength in odd signs and lose it in even signs.

 

Planets get 15 points for being in their appropriate odd or even sign in both the Rashi and Navamsha, giving them a maximum of 30 points in this regard. This is a minor point and it is not given a lot of weight. However, it can be canceled by other factors. Mercury, for example, has no odd and even strength by sign in Virgo, though exalted.

 

\subsubsection{D. ANGULAR STRENGTH}

 
\begin{center}
\begin{tabular}{ l l l l}
Planets in angular houses	 &60 points               \\
Planets in succedent houses	& 30 points               \\
Planets those in cadent houses	 &15 points               \\
 \end{tabular}
\end{center}

This is an important factor because planets are usually stronger in angles. House qualities are important factors in all delineations and must always be considered even if we are not figuring Shadbala.

 

Yet this naturally depends upon the house system we use. If we use a midheaven system like the Placidian or Indian Sripati system, we will get different results than using the equal house systems. But even though some computer programs for Vedic Astrology allow for different house systems, they may calculate the Shadbala only by the house-sign system (the system of house formation that judges houses by sign only).

 

This type of strength has two limitations. First, it does not consider the meaning of the houses adequately. For example, the ninth, though a cadent house, is still good by its being the best trine, and is an excellent position for most planets, particularly benefics. We could not consider Jupiter to be weak here though it would suffer by angular strength according to Shadbala. Second, it may be canceled somewhat by directional strength. For example, the Moon though in an angle in the tenth house, suffers from being in its weakest place (the south) in terms of direction.

 

Another problem is that malefics in angles can be strong to do evil and may harm the person. Saturn in the Ascendant where it has angular strength may harm the person or render them them weak.

 

\subsubsection{E. DECANATE STRENGTH}

Planets are divided into masculine, feminine and neuter.

 
\begin{center}
\begin{tabular}{ l l l l}
Masculine planets	 &Sun, Mars and Jupiter               \\            
Feminine planets	& Moon and Venus               \\
Neuter planets	 &Mercury and Saturn               \\
  \end{tabular}
\end{center}

They gain strength if located in the appropriate decanate or ten degree division of a sign. The rule of determining Decanate Strength is as follows:

\begin{center}
\begin{tabular}{ l l l l}
Masculine planets gain &15 points if they are located in the first decanate of a sign &(00 00—09 60).               \\
Neutral planets gain &15 points if located in the second decanate of a sign    &(10 00—19 60).               \\
Feminine planets &gain strength if located in the third decanate of a sign       &(20 00—29 60).               \\
  \end{tabular}
\end{center}

This is a minor consideration that does not count for more than 15 points. Yet it is part of a greater consideration of Decanate position in the chart. I always look at the Decanate of the Ascendant for determining its results. For example, the Gemini decanate or final ten degrees of a Libra Ascendant will have better intellectual skills as it is in a Decanate of Mercury.

 

\subsubsection{SUMMARY OF POSITIONAL STRENGTH}

 

Divisional Strength is  important and Shadbala is a significant means of determining it. I believe it is the most important part of Shadbala and that it carries most weight in the system. But it remains a general determination. It does not outweigh regular positional issues of planets by sign, house and aspect.

 

\subsection{2. DIRECTIONAL STRENGTH}
 

Directional strength (Dig Bala) is always important and should be considered even if one is not looking at the overall Shadbala.

The  idea behind Directional Strength is similar to that of Exaltation Strength. Just as planets have one sign position in which they are exalted, they have one house position in which they gain Directional Strength. The calculation is the same.

 

A planet gets 60 points of strength at the place of full Directional Strength and 0 points at the place opposite it. The intermediate positions are also divided by three. Each three degrees away from the place of Directional Strength causes a loss of one point of strength.

Planets have Directional Strength in different directions.

Directional Strength

\begin{center}
\begin{tabular}{ l l l l}
Sun and Mars	  &South (tenth house)            \\
Saturn	  &West (seventh house)            \\
Moon and Venus	 & North (fourth house)            \\
Jupiter and Mercury	  &East (first house)            \\
  \end{tabular}
\end{center}

This calculation has the same limitation as that of Angular Strength. It depends upon the house system used. Midheaven house systems will give different results than regular sign-based house systems. I prefer that.

 


\subsubsection{3. TEMPORAL STRENGTH}
 

This is a combination of nine factors based upon the time of birth in hours, days, months, years, etc.   These are:

 

\begin{enumerate}
\item[*] Day-Night Strength (Nathonnatha Bala)
\item[*] Monthly Strength (Paksha Bala)
\item[*] Four Hour Strength (Tribanga Bala)
\item[*] Lord of the Year Strength (Abdadhipati Bala)
\item[*] Lord of the Month Strength (Masadhipati Bala)
\item[*] Lord the Day Strength (Varadhipati Bala)
\item[*] Lord of the Hour Strength (Hora Bala)
\item[*] Declinational Strength (Ayana Bala)
\item[*] Planetary War Strength (Yuddha Bala)
 \end{enumerate}

This is an important factor seldom considered in Western Astrology. After Positional Strength, it is the most significant factor in Shadbala and counts for a great deal in its computation. It is a helpful general consideration, expanding the Panchanga or daily calendar considerations that we look at in Part III of the course.

 

\subsubsubsection{A) DAY-NIGHT STRENGTH}

 

There are several rules in this computation. Planets are powerful at different times of the day:

 
\begin{center}
\begin{tabular}{ l l l l}
Moon, Mars and Saturn	 & Midnight           \\
Sun, Jupiter and Venus	  &Noon           \\
Mercury is always strong	  &Always gets 60 points           \\
   \end{tabular}
\end{center}

At these times of strength, each planet gets 60 points. The amount of time that has elapsed from these times of strength is divided into sixty portions or about 24 minutes. Hence, every 24 minutes away from these times a planet loses one point in strength.

 

This is a factor of some significance but perhaps of not as much as it is given.

 

\subsubsubsection{B) MONTHLY STRENGTH}

 

This computation similarly has its rules. Planets are stronger during certain times of the lunar month:

 

\begin{enumerate}
\item[*] Benefics – bright lunar fortnight or waxing Moon
\item[*] Malefics – dark lunar fortnight or waning Moon
 \end{enumerate}

Benefics are Jupiter, Venus, the Moon when bright and an unafflicted Mercury. Malefics are the Sun, Mars, Saturn, the Moon when not bright and an afflicted Mercury.

 

The factors of calculation are as follows:

\begin{enumerate}
\item[*] The distance of the Moon from its position when new is divided by three. This amount is added to benefic planets.
\item[*] The distance the Moon from its position when full is divided by three and this is added to malefics. Sixty is thus the maximum.
  \end{enumerate}

On the other hand, if the Moon is near new or considered malefic, the calculation is reversed.

\begin{enumerate}
\item[*] The distance the Moon from its position when new is divided by three. This amount is added to malefic planets.
\item[*] The distance the Moon from its position when full is divided by three and this is added to benefics.
  \end{enumerate}

The amount ascribed to the Moon is sometimes doubled.

 

\subsubsubsection{C) FOUR HOUR STRENGTH}

 

Here we divide both day and night into three equal portions or about four hours each. It will vary by the season as the Hindu day is counted from sunrise. It is a secondary consideration.

The planet that rules this period, approximately four-hours, a third portion of day or night, gets 60 points of strength.

 

Mercury	  first third of the day	Sun	  middle third of the day
Saturn	  final third at the end	Moon	  first third of the night
Venus	  middle of the night	Mars	  final third of night
Jupiter	  always gets 60 points		always strong
 

This factor is similar to Day-Night Strength and can cancel it out. For example, Mars is most powerful at midnight in Day-Night Strength but in terms of Four Hour Strength, it is most powerful a couple of hours before sunrise.

 

\subsubsubsection{D-G) LORD OF THE YEAR, MONTH, DAY AND HOUR STRENGTH}

 

The rules here are very simple:

 

Lord of the year	 15 points	Lord of the month	 30 points
Lord of the day	 45 points	Lord of the hour	 60 points
 

These calculations are more complex than they appear. The year is considered to be of 360 days and it is counted off from the theoretical beginning of creation some more than 714,000,000,000 years ago! The Lord of the Month is the planet that rules the weekday of the month during which the birth occurs. This is by a 30-day month from the same theoretical creation eons ago.

 

The Lord of the Day is the planet ruling the day in the normal sequence Sun-Sunday, Moon-Monday, Mars-Tuesday etc., with the normal Hindu day counted from sunrise to sunrise. (Though this turns out the same if it is calculated from the mythic beginning of creation). The Lord of the Hour is according to the planets ruling the hour starting daily from sunrise (see Planetary Hours).

 

Generally a person will do better when under the influence of the hour, day, month and year lords as found at birth. When these repeat themselves during the course of a persons life great changes occur.

 

\subsubsubsection{H) DECLINATIONAL STRENGTH}

 

This strength considers the declinations of planets (how far north or south of the zodiacal equator that they are). We could also refer to it as “Equinoctial or Seasonal Strength”. It is similar to and has the same weight as Angular and Directional Strength.

 

First we have to convert from the Sidereal to the Tropical Zodiac to arrive at the directional points of the solstices and equinoxes. This we can do by adding our Ayanamsha to the position of the planets. On this some other complex calculations are added. The rule is: A planet with the best Directional Strength gets 60 points, down to 0 at the worst, losing one point every three degrees away from their maximum point of directional strength.

 

We must then consider the specific points of Declinational Strength for each planet, which varies relative to (but not exactly the same as) their position via the solstice and equinoctial points:

 

Sun, Mars, Jupiter and Venus	 place of summer solstice
Moon and Saturn	 place of winter solstice
Mercury	 place of the equinoxes
 

Sun, Mars, Jupiter and Venus do best when their declination is at the furthest point north, while the Moon and Saturn do better when it is at the furthest point south. Mercury does best when it is at the equatorial or neutral point, neither north nor south.

 

Hence we can ascertain Declinational Strength generally by noting the declinations of the planets, or by seeing how close to the solstice points they may be.

 

\subsubsubsection{I) PLANETARY WAR STRENGTH}

 

This is a special factor that only comes into play when planets (not including the Sun and Moon) are located within one degree of each other. As per the rules of planetary war, the planet with the lower position in degrees and minutes wins. By other accounts, the planet with the higher declination (further north relative to the ecliptic) wins. Points are added to the victor and subtracted from the loser according to the amount of strength of the planet generally and the relative size of its disc.

 

Planets can gain or lose significantly by planetary war, but the nature of the planets also comes into play here. Venus would lose more in a Planetary War with Mars than with Mercury, as it is generally an enemy of Mars but a friend of Mercury.

 

Generally, I don’t rate planetary war very highly by itself but look more at the nature of the conjunction by the natural and temporal statuses of the planets involved.

 

\subsubsubsection{SUMMARY OF TEMPORAL STRENGTH}

 

While important, it still has limitations. It is good to consider if a planet has some factor of Temporal Strength to give it power. Such factors are whether birth has taken place at a favorable time during the day, month or year for the planet or during a day, month or year ruled by the planet.

 

\subsubsection{4. MOTIONAL STRENGTH}
 

By one system, this is said to consider the distance of a planet from the Sun. Planets are strongest when furthest away from the Sun and weakest when in conjunction with it. They get 60 points when at the furthest point away and 0 when in conjunction with it. As Mercury and Venus never get far from the Sun, their position is not as easy to determine.

 

Yet the calculations to do this are more complicated than that. Motional strength for the Sun and Moon is figured as well, but this is related to Ishta and Kashta Phala, Good and Bad Results of Planets and I have not explained it here.

 


\subsubsection{5. NATURAL STRENGTH}
 

This is the same in all charts and has its own intrinsic value that we must look at in all charts under all circumstances. It mainly reflects the brightness of planets as seen from the Earth. The planets in order of strength are: The Sun, Moon, Venus, Jupiter, Mercury, Mars, and Saturn. This is in terms of apparent brightness. The Sun gets 60 points. This amount is reduced by one-seventh for the other planets in order of their brightness.

 

Sun	60 points	Moon	51 points
Venus	43 points	Jupiter	34 points
Mercury	26 points	Mars	17 points
Saturn	9 points		
 

\subsubsection{6. ASPECTUAL STRENGTH}
 

This is complicated as it considers major and minor aspects according to the exact arc and degree of aspect. Aspects of benefics are counted as positive, those of malefics as negative. The special aspects of planets are given additional weight.

 

Yet this actor seldom has little numerical value, counting from 30 points positively or negatively! It becomes one of the least important of the Shadbalas, on par with Decanate Strength.

 

This can be different than the usual strength of planetary aspects given in general chart readings in Vedic Astrology. Hence, when considering aspectual strength as part of Shadbala, we must still consider aspects in the chart itself as having their own weight apart from Shadbala. The power of aspects in the chart is not limited to it.

 

 

\subsubsection{TOTAL SHADBALA}

 

All the factors of Shadbala are added together. This total is called “Virupa”. This is divided by 60 and becomes the “Rupa” total. I like to examine this and note particularly which planet is strongest and which planet is weakest. It is this Shadbala calculation that is given the most weight.

 

This amount is divided differently relative to each planet:

Division of Rupa Total by Planet

Sun                by 5 or 6

Moon              by 6

Mars               by 5

Mercury          by 7

Jupiter            by 6.5

Venus            by 5.5

Saturn            by 5

 

This gives the Shadbala Power Ratio. A planet requires a power ratio of at least 1.0 to have sufficient Shadbala strength. Most planets get this rather easily. Saturn is most likely to be low. Planets with higher Shadbalas gain greater power and over those with lower. This is usually the case.

 

However, I have seen charts with all planets high in Shadbala and the person has had a miserable life and accomplished nothing. This can be traced to common afflictions like bad aspects or difficult house positions. These things cannot simply be erased by positive Shadbala indications.

 

I have also seen people with charts that have no planets high in Shadbala doing quite well. I have seen genius Mercury types with an average Mercury Shadbala. On the other hand, I have seen several instances of weak hearts with a low Shadbala for the Sun. Yet, generally if a Shadbala is either very low or very high, it will indicate something very weak or very strong about the planet. Average Shadbalas may give good or bad results depending upon other factors.

 

Therefore, though of general value, the details of Shadbala may require a specific analysis for us to really benefit from them. 

 

\subsubsection{GOOD AND BAD RESULTS OF PLANETS (Ishta and Kashta Phala)}
 

Called Ishta and Kashta Phala in Sanskrit, this refers to the ability of planets to give good or bad results during their periods.

 

A planets power to good during its period is determined by multiplying its Exaltation Strength by its Motional Strength and taking the square root of it. Its power to do ill is determined by subtracting the Exaltation Strength by sixty and the Motional Strength by 60, multiplying them and taking the square root of that. These two totals are compared. If the power to do good is higher, then the planet will give good results during its periods. If the power to do ill is higher, it will tend to cause difficulties.

 

Such elaborate calculations seldom counter the obvious indications in a chart and so mainly relate to more detailed research or fine-tuning, not the first line of interpretation.

 

A good rule is never to weigh the effects of mechanical interpretations like Shadbala more than one-third, but at the same time, not to ignore them. Their specific factors like directional strength can be very important even apart from Shadbala as a whole. The main thing is that if a planet is unusually high or low in Shadbala, we should carefully examine its place in the chart.

 