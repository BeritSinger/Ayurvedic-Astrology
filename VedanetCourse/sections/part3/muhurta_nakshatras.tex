\section{ASTROLOGICAL FORECASTING AND MUHURTA 1
NAKSHATRAS}
 

While this course is primarily concerned with Natal astrology or examining the birth chart, astrological forecasting, or determining the right times for various actions, is also important. We want to examine this issue further at this point and give the student a foundation for understanding the

 

Much of Vedic Astrology is concerned with determining favorable times for action, both from the standpoint of mundane actions and spiritual practices. This is part of a reverence for the forces of the universe, the foremost of which is time itself, and a need to link our action with these forces in order for our actions to be successful.

 

The background for this study will be the sections in the course on Prashna and the Nakshatras, which should be thoroughly understood to benefit from this material. As astrological forecasting is a complex issue, it may take some time for the student to grasp the ideas presented. This and the following lessons are more advanced and, to a certain extent, supplementary in nature. Hence, they will not have as many questions as the other sections. The student should approach these sections carefully. It may take repeated examination and application of them for them to make sense.



 

 

\subsection{ASTROLOGICAL FORECASTING}

 

Most Vedic astrologers in the West are not well versed in this system of astrological forecasting, though it is very commonly used in India and is perhaps the main job for which astrologers are approached for guidance. Western astrologers emphasize the birth chart and Vedic astrologers in the West usually follow this practice. However, as Vedic astrologers extend their knowledge, and as Vedic Astrology becomes more known in the West, its forecasting side will become more important, as it greatly extends the range of its practice.

 

\subsection{THE VEDIC SCIENCE OF THE MOON}

 

The Vedic system puts much emphasis on the Moon and its positions. This is true not only in the case of the birth chart but even more so in the case of astrological forecasting in Horary and Electional Astrology. Hence, for determining what activities may be favorable on particular days, we should understand the Vedic science of the positions of the Moon. This lunar science is much more developed in Vedic than in Western astrology, which contains a number of new ways of examining the effect of the Moon that have no counterpart in the Western system.

 

Vedic science gives us a key to understanding the influence of the Moon on human experience. The Moon governs the rhythms of the earth, being its satellite, thus impacting vital energy in all its forms. It is well known that the fluctuations of the Moon influence human, plant, animal and ocean life. The Moons connection with human moods and mental states is part of literature and mythology all over the world.

 

\subsection{PANCHANGA}

 

Generally, five factors are considered relative to Vedic astrological forecasting and the timing of actions. This five-part system, called “Panchanga” (having five parts), is comprised of:

 
\begin{enumerate}
\item[*] Tithi or lunar day
\item[*] Vara or solar day
\item[*] Nakshatra or lunar constellation
\item[*] Yoga or combinations relative to the Sun and Moon
\item[*] Karana or half of the Tithi
\end{enumerate}

With the exception of the solar day, they depend upon the position of the Moon.

 

Various yearly almanacs or Panchangas can be purchased giving these calculations. The Vedic sidereal almanac presents all these factors on a daily basis, which would be very complex and time consuming to calculate on ones own. Such SIDEREAL ASTROLOGICAL ALMANACS are published in the United States and India and are available through JDR Ventures in Fairfield, Iowa. Every Vedic astrologer should possess and use a Panchanga and note the positions in effect each day. In addition, most Vedic astrological computer programs provide this information.

 

Otherwise one can use Vedic software, which always includes a Panchanga option.

 

\subsection{PANCHANGA AND MUHURTA}

 

Panchanga is part of the Vedic System called Muhurta, which we introduced in the lesson on Horary Astrology. For background information, one can consult Muhurta or Electional Astrology by B.V. Raman. Such books, though cumbersome and containing many issues not very relevant for modern Western society, contain much that is useful.

 

Panchanga is a complex subject and we cannot deal with it fully here. However, we can introduce the main principles so that at least the student can learn how to ascertain a favorable day, Nakshatra and Tithi as these are most important factors in Muhurta. If we choose a favorable day, Nakshatra and Tithi, our action is much more likely to be successful. This is what we will aim at primarily in this chapter, though we will also outline the fundamentals of Karana and Yoga, as well as other related factors.

 

\subsection{THE ASCENDANT AND THE GENERAL CHART}

 

The Ascendant and its lord must be strong for any good actions to be successful. The indications of other planets and houses are also important, particularly those most relevant to the particular topic (like the seventh house and its lord for marriage issues). These factors can be examined relative to the ordinary rules of astrology and those we have indicated in our section on Prashna.

 

Hence, the information on the Moon and Panchanga should supplement our study of Muhurta and Prashna as a whole, and should not substitute for a complete examination of the chart.

 

We must also consider the birth Nakshatra (Janma Nakshatra) of the person and the birth sign of the Moon (Janma Rashi) for a more detailed analysis.

 

Such complete analysis, however, can be very complicated and time consuming and hence is only necessary for very important actions that can be planned in advance. For most ordinary purposes, a favorable day, Nakshatra and Tithi is enough. This can be merely looked up in the almanac and thus saves much time and effort. Or, once we find these factors, we can then look for a favorable Ascendant during the day in question to make the timing of events even more favorable.

 

Muhurta is used specifically for the various sacraments of the Hindu religion like marriage, birth of children, performance of important rituals, spiritual initiation, etc., of which there are sixteen in number. It is also used for matters like travelling, buying homes, and other major life decisions.

 

\subsection{I. NAKSHATRAS}


We will start with the Nakshatras as these are the most important factor in Muhurta. The Moon located in different Nakshatras is favorable for different activities. Some Nakshatras are good for many actions, others are good for very little. Some are better for spiritual practices than for ordinary actions, others vice versa. Hence, noting the Nakshatra occupied by the Moon, we can determine what may be good or bad for the day involved.

 

In the same way, people born under these Nakshatras will share similar propensities. Their capacity for action will have the same basic potentials as that of their Nakshatra (though it should not be judged by the Nakshatra alone but by the whole chart). Hence, this information is also useful for birth chart analysis.

 

We have previously discussed the general qualities of Nakshatras and, particularly, their spiritual indications in the section on the Nakshatras. Here their more mundane and temporal indications will be explored.

 

 

\subsubsection{BASIC QUALITIES OF NAKSHATRAS}
 

The Nakshatras can be divided into several groups according to general qualities or their nature. These qualities are very important and should be memorized by all Jyotish students.

 
\begin{center}
\begin{tabular}{ l l}
FIXED (Dhruva)	& Rohini, Uttara Phalguni, Uttarashadha, Uttara Bhadra \\
HARSH (Tikshna)	 &Ardra, Aslesha, Mula, Jyeshta.                                   \\
FIERCE (Ugra)	 &Bharani, Magha, Purva Phalguni, Purvashadha, Purva Bhadra.  \\
 \end{tabular}
\end{center}

\begin{center}
\begin{tabular}{ l l}
QUICK (Kshipra)	 &Ashwini, Pushya, Hasta, Abhijit                                   \\
SOFT (Mridu)	 &Mrgashiras, Chitra, Anuradha, Revati                                   \\
MIXED (Mridu-Tikshna)	 &Krittka, Vishakha                                   \\
MOVEABLE (Chara)	 &Punarvasu, Swati, Shravana, Dhanishta, Shatabhishak                                   \\
  \end{tabular}
\end{center}

 

\subsubsection{FIXED NAKSHATRAS}

Rohini, Uttara Phalguni, Uttarashadha, and Uttara Bhadra (4, 12, 21, 26), are “fixed” Nakshatras favorable for establishing or founding any project expected to be enduring in nature. Building houses, planting trees, marriage, and conception of children are included here. Since most important projects in life are of this type, this type of Nakshatra is very favorable for most major actions.

 

\subsubsubsection{MUTABLE NAKSHATRAS}

Punarvasu, Swati, Shravana, Dhanishta, and Shatabhishak (7, 15, 22, 23, 24), are mutable or moving Nakshatras. They are favorable for actions wherein change and movement are important. Education, communication and commerce, travel, acquiring new vehicles, recreational ventures and influencing the public are included here.

 

\subsubsubsection{SOFT NAKSHATRAS}

Mrigashira, Citra, Anuradha, and Revati (5, 14, 17, 27) are soft Nakshatras. Hence, they are generally favorable for issues requiring softness, pliability and sensitivity. They are good for putting on new clothes, for artistic ventures, for healing, for domestic, personal and emotional issues, and for sexual activity.

 

\subsubsubsection{LIGHT NAKSHATRAS}

Ashwini, Pushya, Hasta, and Abhijit (1, 8, 13) are light (as opposed to heavy) Nakshatras. They are generally favorable for new undertakings as they allow for growth, development and movement. They are good for putting on new clothes and gemstones, for taking medicines or medical therapies, for subtle endeavors such as gaining knowledge, for physical activities and sports, and for travel.

 

\subsubsubsection{SHARP NAKSHATRAS}

Ardra, Aslesha, Jyeshta, and Mula (6, 9, 18, 19) are sharp or harsh Nakshatras. Hence, they are generally unfavorable and most useful for negative, protective, or wrathful actions like countering enemies, warding off evil spirits, or dispensing punishment. They require that we act in a sharp and decisive manner. Arguments, competition, and struggle are more likely under their influence. Yet they do allow us to make new breakthroughs if we are open to their energy.

 

\subsubsubsection{CRUEL NAKSHATRAS}

Bharani, Magha, Purva Phalguni, Purvashadha, and Purva Bhadra (2, 10, 11, 20, 25) are cruel, fierce, or dreadful Nakshatras. Hence, they are generally the most unfavorable. They are best for negative actions, as are the sharp Nakshatras. They require that we take a wrathful or harsh mode in our actions. On their days, difficulties, obstructions, enmity, and calamities of various sorts are more likely to happen. However, they can enable us to overcome such difficulties and gain much more power in life.

 

\subsubsubsection{MIXED NAKSHATRAS}

Krittika and Vishakha (3, 16) are mixed in their nature (combining good and bad, harsh and gentle influences). Hence, for performing negative actions to overcome difficulties and gain power, they are suitable for ordinary activities but not for extraordinary activities or for commencing new ventures.

 

\subsubsubsection{NAKSHATRAS IN AQUARIUS AND PISCES}

As these Nakshatras are in the last portion of the zodiac, they are not considered to be so favorable for establishing new actions. This includes the last half of Dhanishta to the end of Revati. Their period is called Nakshatra Panchaka. When the Moon is here, one should avoid journeys (particularly to the southern direction), repairing or renovation activities, or acquiring new items. However, this is a secondary consideration.

 

\subsubsubsection{Favorable Nakshatras}
 
\begin{center}
\begin{tabular}{ l l l l}
1  Ashwini                  &F                  & 15  Swati                   & MF                                  \\

4  Rohini                    &VF                 &  17  Anuradha           &   F                                  \\

5  Mrigashiras            &F                 &  21  Uttrarashada         & F                                  \\

7  Punarvasu              &VF              &     22  Shravana              &  F                                  \\

8  Pushya                   &MF              &   23  Dhanishta             &F                                  \\

12  Uttara Phalguni      & F              &    24  Shatabhishak        & F                                  \\

13  Hasta                      &F                &  26  Uttara Bhadra         &F                                  \\

14  Chitra                      &F                &  27  Revati                     & F                                  \\

 &&&                                  \\

F  –  Favorable              & VF  –  Very Favorable           &  MF  –  Most Favorable                                  \\

   \end{tabular}
\end{center}

 

\subsubsubsection{Unfavorable Nakshatras}
 
\begin{center}
\begin{tabular}{ l l l l}
2  Bharani                  &VU                  16  Vishakha                   &X                                 \\

3  Krittika                    & X                  18  Jyeshta                   &U                                 \\

6  Ardra                       &U                  19  Mula                       & U                                 \\

9  Ashlesha                &U                  20  Purvashada           &VU                                 \\

10  Magha                    &VU                 &25  Purva Bhadra       &VU                                 \\
11  Purva Phalguni      &VU&&                                 \\

 & &  \\

  U  –  Unfavorable           &   VU  –  Very Unfavorable               & X  –  Mixed                                 \\
   \end{tabular}
\end{center}

 

\subsubsection{MUHURTA YOGAS}
 

Specific Nakshatras falling on particular days of the week give rise to certain combinations (Yogas), some of which are auspicious, others which are adverse. Some auspicious combinations of weekdays and Nakshatras are traditionally listed under Siddha Yoga and Sarvartha Siddhi Yoga, both being favourable for all pursuits, and Amrita Siddhi Yoga, ensuring accomplishments in ones endeavors.

               

\subsubsubsection{Siddha Yoga  }           

\begin{center}
\begin{tabular}{ l l l l}                                                                                   

1 Sun     –       & 19 Mula                     & 5 Thu   –          & 7 Punarvasu                                \\

2 Mon    –       & 22 Shravana              & 6 Fri     –        & 11 Purva Phalguni                                \\

3 Tue     –       & 26 Uttara Bhadra        &7 Sat    –         &15 Swati                                \\

4 Wed    –          &3 Krittika &&                                 \\
   \end{tabular}
\end{center}
 

\subsubsubsection{Sarvartha Siddhi Yoga   }                                                                                        

\begin{center}
\begin{tabular}{ l l l l}    
1 Sun     –        &1, 8, 12, 13, 19, 21, 26            &5 Thu   –         &1, 7, 8, 17, 27                                \\

2 Mon    –        &4, 5, 8, 17, 22                         & 6 Fri     –         &1, 7, 17, 22, 27                                \\

3 Tue     –        &1, 3, 9, 26                                &7 Sat    –        & 4, 15, 22                                \\

4 Wed    –        &3, 4, 5, 13, 17

    \end{tabular}
\end{center}

\subsubsubsection{Amrita Siddhi Yoga}                                                                           
\begin{center}
\begin{tabular}{ l l l l}    
1 Sun     –        &13 Hasta                     & 5 Thu   –          & 8 Pushya                               \\

2 Mon    –       & 22 Shravana                & 6 Fri     –        & 27 Revati                               \\

3 Tue     –          &1 Ashwini                   &7 Sat    –           &4 Rohini                               \\

4 Wed    –       & 17 Anuradha
   \end{tabular}
\end{center}
 

Specific Nakshatras falling on particular days of the week give rise to certain combinations (Yogas), some of which produce adverse results.  These inauspicious combinations of weekdays and Nakshatras, listed under Mrityu Yoga, should be avoided, when possible, in performing important activities, especially travelling.

               

\subsubsubsection{Mrityu Yoga }                                                                                             

\begin{center}
\begin{tabular}{ l l l l}  
1 Sun     –        &17 Anuradha               &5 Thu   –          & 5 Mrigashiras                              \\

2 Mon    –        &21 Uttarashadha         &6 Fri     –           &9 Ashlesha                              \\

3 Tue     –        &24 Shatabhishak         &7 Sat    –         &13 Hasta                              \\

4 Wed    –          &1 Ashwini  &&                              \\
   \end{tabular}
\end{center}
 

 

\subsubsection{SPECIFIC CHARACTERISTICS OF NAKSHATRAS FOR TIMING EVENTS}


 

Each Nakshatra has its specific indications, though for most purposes determining whether they are generally favorable or unfavorable is enough.

 

\subsubsubsection{1. ASHWINI, 00º 00—13º 20 Aries}

As the first Nakshatra in the zodiac, Ashwini is useful for all good beginnings and for creating higher goals and good intentions in our actions. The first quarter of the Nakshatra, however, is not so auspicious as it is a transitional period between the ending and beginning of the zodiac.

 

\paragraph{Favorable For:}

Healing and preparing or taking medicines and therapies (the Ashwins who rule this Nakshatra are renowned doctors and miracle workers)
Planting
Beginning of study
Putting on new clothes or gemstones
Taking a new name
Legal matters
Learning astrology and other spiritual and occult sciences
Giving initiations, secret teachings and empowerments
Installing sacred items (like gems, altars, statues, yantras or temples)
 

\subsubsubsection{2. BHARANI, 13º 20—26º 40 Aries}

Bharani is generally unfavorable as it is ruled by Yama, the God of death, and has an active, aggressive and sometimes impulsive energy that cannot always be trusted. Impulses and desires happening under its influence are likely to be impetuous or misleading.

 

\paragraph{Favorable For:}

Harsh or rash actions
Competitive ventures
Things relating to fire
Deep meditation
Self-discipline and yogic and ascetic practices including Hatha and Raja Yoga
Purification actions of various sorts, like fasting or observing silence
Driving out negative energies and emotions
 

\subsubsubsection{3. KRITTIKA, 26º 40 Aries—10º 00 Taurus }

Krittika is also generally unfavorable, as it is ruled by fire and its symbol is a razor.

However, it gives victory over enemies and obstacles and is therefore suitable for harsh actions. It is the Nakshatra of the wives of the seven sages or rishis. Hence, it is a Nakshatra of Shakti or power, and gives a martial and discriminating power to speech and perception. The full Moon here is particularly good for lighting sacred fires. Children are more likely to be sick or disturbed under its influence.

 

\paragraph{Favorable For:}

Harsh actions

Competition and debate
Working with metals or fire (including fire offerings)
Gaining control of the senses and unfolding our deeper spiritual will, aspiration and perception
Worship of Lord Skanda, the son of Shiva, who was born under this star. Hence, his devotees fast on the lunar day governed by it.
 

\paragraph{Unfavorable For:}

Lending or borrowing money
Initiations
Travel
Sexual activity
 

\subsubsubsection{4.  ROHINI, 10º 00—23º 20 Taurus}

Rohini is a very creative and favorable Nakshatra that allows us to manifest our desires in material form. It is the Nakshatra most loved by the Moon. It gives much enjoyment but can create much karma if we dont know how to restrain its positive outgoing influences. As it is ruled by the Creator of the universe, Lord Brahma or Prajapati, it is favorable for his worship. As Krishna was born under this star, it is good for his worship and has a similar energy of devotion and joy.

 

\paragraph{Favorable For:}

Putting on new clothes or gemstones
Taking a new name
Healing, taking medicines or administering therapies (such as massage)
Practices aimed at longevity and rejuvenation
Planting or gardening
Marriage
Sexual activity
Performing religious rituals, initiations, devotional practices and chanting Establishing altars or empowering sacred objects
Learning languages
Building houses or temples, or entering a new house
Legal matters
Financial gain
Acquiring fixed items
 

\subsubsubsection{5. MRIGASHIRAS, 23º 20 Taurus—06º 40 Gemini  }

Generally favorable but can tend toward self-indulgence.

 

\paragraph{Favorable For:}

Travel
Marriage
Sexual activity
Building temples, consecrating religious items or altars
Spiritual initiations
Taking a new name
Healing and treatment of disease (particularly for convalescence)
Rejuvenation practices (Soma, the ruler of this Nakshatra is the heavenly nectar of immortality.)
Artistic work (particularly sculpture)
Buying, building or entering a new house
Commencing educational ventures
Learning languages
Legal matters
Excellent for moving or change of residence
Spiritually good for developing Divine joy or bliss, but requires a degree of self- control, renunciation and sacrifice to allow this to occur
 

\subsubsubsection{6. ARDRA, 06º 40—20º 00 Gemini }

Ardra is generally unfavorable as it relates to wrathful Rudra and to the storm. It is the abode of Shiva or the Divine masculine force and is thus useful for propitiating wrathful forms of the Divine, like Rudra (Shiva), and for gaining control of the mind and senses. Under its influence meditations for peace and for dispelling fear are helpful. However, it is an unstable Nakshatra generally and cannot be relied upon for establishing long term projects.

 

\paragraph{Favorable For:}

Fighting
Administering punishment
Warding off negative influences (exorcism)
Clearing out old or difficult karma
Atonement
Psychological cleansing
Harsh actions such as surgery
Commencing educational ventures which require strictness and discipline
 

\paragraph{Unfavorable For:}

Initiations (except those of wrathful Deities)
Travel
Sexual activity
 

\subsubsubsection{7. PUNARVASU, 20º 00 Gemini—03º 20 Cancer}

This Nakshatra is regarded as very favorable, particularly the last quarter wherein the Moon is located in its own sign. It is the abode of Sakti or the Divine feminine force.

 

\paragraph{Favorable For:}

Commencing educational ventures
Learning astrology
Learning languages
Taking medicines
Healing therapies
Putting on gems or new clothes
Taking a new name
Planting and gardening
Buying houses
Legal matters
Religious ceremonies, initiations, fasting
Building temples, altars or installing deities
Worship of the Divine Mother and return to the source
Reconciliation between people
Returns to old places or positions (Punarvasu means return of the light)
 

\paragraph{Unfavorable For:}

Borrowing or lending money
 

\subsubsubsection{8. PUSHYA, 03º 20—16º 40 Cancer}

Pushya is considered to be the most favorable Nakshatra as a whole for all auspicious actions (Jupiter is exalted here). It grants fearlessness and success. It resembles the God Ganesha in what it can do for us.

 

\paragraph{Favorable For:}

Commencing educational ventures
Learning astrology
Learning languages
Artistic ventures
Taking of medicines and therapies
Putting on gems or new clothes
Taking a new name
Planting and gardening
Building or entering houses
Legal matters
Religious activities
Installing altars or statues of deities
Performing rituals
Meditation
Spiritual Initiation
Excellent for learning spiritual teachings (Vedas) as it is ruled by Brihaspati who is the original and foremost of the rishis and sages.
 

\paragraph{Unfavorable For:}

Marriage
 

\subsubsubsection{9. ASLESHA, 16º 40—30º 00 Cancer }

Ashlesha is generally an unfavorable and unreliable Nakshatra. Deception is more likely to occur under its influence, as is disease, including contagious diseases like colds and flus. The last quarter is especially inauspicious as it marks the end of a major segment of the zodiac.

 

\paragraph{Favorable For:}

Competitive ventures
Propitiating negative forces (like serpent deities)
Occult knowledge including astrology
Harsh actions like surgery
Buying houses
Starting businesses
Travel
Legal ventures
Initiations
Sexual activity
 

\subsubsubsection{10. MAGHA, 00º 00—13º 20 Leo }

Magha is the Nakshatra of the Pitris or the fathers and is good for honoring ones ancestors.

 

\paragraph{Favorable For:}

Honoring ones father, guru, the rishis, and other ancestors
Promoting tradition or family
Meeting great personages
Assuming positions of authority
Marriage
Planting
Buying houses
Hazardous undertakings
Learning music and dancing
 

\paragraph{Unfavorable For:}

Initiations
Lending money
Marriage
Sexual activity
 

\subsubsubsection{11. PURVA PHALGUNI, 13º 20—26º 40 Leo }

Purva Phalguni is ruled by Bhaga or Bhagavan, the beloved form of the Lord. It is generally regarded as unfavorable, but perhaps unrightly so.

 

\paragraph{Favorable For:}

Buying houses
Hypnotising and gaining control over others
Gaining fame, charisma, or personal power
Artistic ventures, like music, painting or dance
Business matters
Commencing studies
Adoration of the Divine; devotional practices
 

\paragraph{Unfavorable For:}

Travel
Initiations
Sexual activity
 

\subsubsubsection{12. UTTARA PHALGUNI, 26º 40 Leo—10º 00 Virgo }

Uttara Phalguni is generally favorable and is the best Nakshatra for marriage.

 

\paragraph{Favorable For:}

Marriage
Performing rituals
Starting mantras
Taking spiritual initiation
Taking vows
Installing altars or deities
Healing and treating diseases
Giving advice and good counsel
Mediating between hostile parties
Legal matters
Gaining fame and recognition
Commencing studies
Learning philosophy
Building or entering houses
Putting on gems or new clothes
Planting
Sexual activity
Service ventures
 

\paragraph{Unfavorable For:}

Lending money
 

\subsubsubsection{13. HASTA, 10º 00—23º 20 Virgo }

Hasta is ruled by Savitar, the Divine solar Creator, and is generally a favorable Nakshatra. It is also the place where Mercury is exalted.

 

\paragraph{Favorable For:}

Travel
Commencing studies
Beginning work ventures
Marriage
Putting on gems or new clothes
Taking a new name
Treatment of disease
Planting
Building a house
Manual, artistic and creative ventures (Hasta means “hand”)
Legal matters
Learning astrology
Learning languages or philosophies
Commencing educational ventures
Mantras, spiritual initiations, chanting, yogic practices
Study of the Vedas
Change of residence
Marriage
 

\subsubsubsection{14. CHITRA, 23º 20 Virgo—06º 40 Libra}

Chitra is ruled by Tvashar, the Divine craftsman, and so good for production activities of all types. It is generally favorable.

 

\paragraph{Favorable For:}

Taking medicines, or giving and receiving healing therapies
Artistic work
Construction work such as building houses, temples, or making statues of deities Making or putting on of gemstones or new clothes
Entering a new house
Commencing educational ventures
Artistic activities such as music and dancing
Planting and gardening
Legal matters
Public, social and political issues
Relating to women or the opposite sex
Collecting herbs and preparing medicines
Spiritual practices such as visualization
 

\subsubsubsection{15. SWATI, 06º 40—20º 00 Libra}

Ruled by Vayu or the wind. Generally one of the most favorable Nakshatras, though can be unstable.

 

\paragraph{Favorable For:}

Putting on gems or new clothes
Taking a new name
Installing altars or deities
Treating diseases
Planting, particularly gathering herbs and preparing medicines
Commencing educational ventures
Learning astrology
Social and political activities
Marriage
 

\paragraph{Unfavorable For:}

Travel
Sexual activity
 

\subsubsubsection{16. VISHAKHA, 20º 00 Libra—03º 20 Scorpio }

Vishakha is good for worshipping Lord Krishna (as it is associated with Radha, his consort). Lord Buddha was born with the full Moon in this star, hence, it is good for his worship as well. It often requires decisive action on our part. Spiritually, it is good for reorienting the mind from ordinary emotions to devotion.

 

Good For:

Making gems, ornaments, arts and crafts
Sculpture
Architecture
Dancing
Mechanical work
Taking of medicines or giving of therapies
Putting on gems or new clothes
Buying houses
Overcoming enemies
 

\paragraph{Unfavorable For:}

Travel
Initiations
Sexual activity
Health of children
 

 

\subsubsubsection{17. ANURADHA, 03º 20—16º 40 Scorpio }

Generally, a favorable Nakshatra.

 

\paragraph{Favorable For:}

Artistic pursuits
Planting
Friendship
Marriage
Sexual activity
Treatment of diseases
Entering a new house
Spiritual initiation
Development of devotion and compassion
Legal matters
Travel or change of residence
 

\subsubsubsection{18. JYESHTA, 16º 40—30º 00 Scorpio}

Generally unfavorable. It requires a strong will and decisive action to cut through the obstacles it may bring us.

 

\paragraph{Favorable For:}

Competitive ventures
Harsh actions like surgery
Challenging and overcoming obstacles and obstructions
Artistic pursuits
 

\paragraph{Unfavorable For:}

Spiritual initiation
Sexual activity
Legal matters
 

NOTE: The point between Jyeshta and Mula is considered to be the most unfavorable position for the Moon. The Moon is between signs, between the end of the second third of the zodiac and the start of the last third, and between demonic (Rakshasa) Nakshatras. It is a particularly dangerous time for children to be born and can cause poor health or early death. This marks the last quarter of Jyeshta and the first two of Mula.

 

\subsubsubsection{19. MULA, 00º 00—13º 20 Sagittarius}

Hanuman, the monkey God and companion of Lord Rama, was born under this star and hence it is favorable for his worship.

 

\paragraph{Favorable For:}

Planting trees, agriculture and gardening (Mula means root)
Buying or constructing houses (Mula also means foundation)
Taking a new name
Marriage
Sexual activity
Learning astrology and philosophy
Harsh actions like surgery
Warding off negative influences and avoiding calamities through discretion and protective action
 

\paragraph{Unfavorable For:}

Spiritual initiations
Lending or borrowing money
 

\subsubsubsection{20. PURVASHADHA, 13º 20—26º 40 Sagittarius }

Generally unfavorable.

 

\paragraph{Favorable For:}

Agriculture
Preparing medicines
Making religious objects
Reciting mantras
Reconciliation, forgiveness and settling debts
Issues relating to water (it is ruled by the Waters)
Spiritual practices aimed at purification and elimination of sin and negative karma
 

\paragraph{Unfavorable For:}

Spiritual initiation
Sexual activity
 

\subsubsubsection{21. UTTARASHADHA, 26º 40 Sagittarius—10º 00 Capricorn}

 

\paragraph{Favorable For:}

Artistic ventures
Wearing new clothes or ornaments
Treatment of disease
Planting
Installing altars, temples or deities
Religious ceremonies and initiation
Beginning Vedic studies
Marriage
Sexual activity
Entering a new house
Improving ones home
Public and political action
Assuming political power or public office
Taking positions of authority
Gaining success and victory in our endeavors
 

\paragraph{Unfavorable For:}

Travel
 

\subsubsubsection{22. SHRAVANA, 10º 00—23º 20 Capricorn }

Generally auspicious for all favorable activities, including education and administrative actions and travel. It is important for the worship of Vishnu, the preserver of the universe, who rules this Nakshatra.

 

\paragraph{Favorable For:}

Purchasing new items
Taking a new name
Study and teaching
Philosophy
Learning music
Meditation
Taking herbs and medicines
Influencing the public and political matters
Learning spiritual teachings (Vedas)
Spiritual initiations and rituals
Gaining space, freedom and understanding
 

\subsubsubsection{23. DHANISHTA OR SHRAVISHTA, 23º 20 Capricorn—06º 40 Aquarius}

Ruled by the Vasus or powers of earthly light and abundance. Generally favorable.

 

\paragraph{Favorable For:}

Travel
Putting on gems or new clothes
Taking a new name
Treating disease
Artistic ventures
Financial gain and profit (Dhanishta means “the wealthiest”)
Commencing educational ventures
Learning languages
Handling weapons
Legal matters
Gaining fame and recognition
Excellent for learning spiritual teachings (Vedas)
Spiritual initiations
Learning medicine
Gaining of merit
 

\paragraph{Unfavorable For:}

Borrowing money
 

\subsubsubsection{24. SHATABHISHAK, 06º 40—20º 00 Aquarius}

Ruled by Varuna who governs debts, karmic retribution and purification. Mixed in results.

 

\paragraph{Favorable For:}

Medicines and therapies (Shatabhishak means “a hundred medicines”)
Vehicles and travels (particularly by sea)
Commencing educational ventures
Taking names
Learning philosophy
Artistic ventures
Starting a new business
Sexual activity
Spiritual practices aimed at purification of the mind and clearing out old karma Psychological therapies
Longevity and rejuvenation
 

\paragraph{Unfavorable For:}

Spiritual initiations
Legal issues
Visiting spiritual teachers
Lending money
 

\subsubsubsection{25. PURVA BHADRA, 20º 00 Aquarius—03º 20 Pisces }

Generally unfavorable.

 

\paragraph{Favorable For:}

Building and construction
Agriculture
Business purchases
Receiving mantras
Installation of religious objects
Working with water
 

\paragraph{Unfavorable For:}

Travel
Sexual activity
 

\subsubsubsection{26. UTTARA BHADRA, 03º 20—16º 40 Pisces }

Generally favorable.

 

\paragraph{Favorable For:}

Artistic ventures
Healing and treatment of disease
Planting and gardening
Sexual activity
Marriage
Putting on of gems
Installing altars, temples or deities
Religious ceremonies or visiting temples
Spiritual initiations
Entering a new house
Naming of children
Establishing deep and solid foundations in life
 

\paragraph{Unfavorable For:}

Travel
 

\subsubsubsection{27. REVATI, 16º 40—30º 00 Pisces}

The last quarter of this Nakshatra is generally inauspicious as it marks the end of the zodiac. Ruled by Pushan who is the solar power of perception, nourishment and transition to a new life.

 

\paragraph{Favorable For:}

Putting on gems or new clothes
Healing and treatment of disease
Marriage
Sexual activity
Planting and gardening
Legal matters
Spiritual initiations
Learning astrology, languages or philosophy
Artistic ventures (particularly music)
Entering a new house
Buying houses
Business ventures, exchanges, finances and income (Revati means “the most wealthy”)
Protection
 

\subsubsection{NOTES ON INTERPRETING NAKSHATRAS}

In the Nakshatra system we note some deviation from what we might expect relative to the Moons sign rulership and exaltation. Vishakha, the Nakshatra of the Moons debility, appears more favorable for most activities than Krittika, the sign of its exaltation. Ashlesha, part of Cancer, the Moons own sign, is generally unfavorable.

We also see some deviation in the Vedic Gods ruling these Nakshatras. For example, Purva Phalguni is not very auspicious, though it is ruled by Bhaga, the most auspicious Vedic deity.

In referring to the Nakshatras as outlined above, it is the timing of actions that is referred to, and often mainly more mundane actions, not necessarily the basic characteristics of a person. An exalted Moon may give better qualities to the character. However, the debilitated Moon may be a better place for starting new actions on a daily basis. Yet, as a rule, I dont like to prescribe a debilitated Moon for most actions.

I also think that this system may require some modification in light of our modern circumstances today. For example, the Nakshatras traditionally regarded as good for sexual activity are mainly those that are good for fertility, not necessarily those that are best for communion. Hence, one should not take this system rigidly.

One may also find some variations on these recommendations in different Vedic Astrology books. As the Nakshatra is only one factor, we need not find a perfect Nakshatra; one that is generally good for a matter is usually sufficient if the other factors, the day and the tithi are good.

 

\subsubsection{BIRTH NAKSHATRA AND DAILY NAKSHATRA}
 

To determine the favorability of the daily Nakshatra, we should examine its relationship with the birth Nakshatra (the Nakshatra of the Moon at birth).

For this, we count from the birth Nakshatra to the daily Nakshatra (counting the birth Nakshatra as one). We subtract multiples of nine from that amount and consider the remainder. If the amount is less than nine, we take that.

For example, if the birth Nakshatra is Uttara Phalguni (12) and the Nakshatra for the day is Shravana (22), the number is 11. Subtracting 9 we get 2.

 

\subsubsection{RELATIONSHIPS TO BIRTH NAKSHATRA}

 \begin{enumerate}

\item[] 1) “Janma” or birth	Indicates danger, particularly to the body; generally unfavorable
 

\item[] 2) “Sampat” or gain	Gains and accomplishments; generally favorable
 

\item[] 3) “Vipat” or loss	Losses, dangers and accidents; generally unfavorable
 

\item[] 4) “Kshema” or prosperity	Security and safety; generally favorable
 

\item[] 5) “Pratyak” or opposition	Difficulties and obstacles; generally unfavorable
 

\item[] 6) “Sadhana” or accomplishment	Success, achievement and the realization of our goals and desires; generally favorable
 

\item[] 7) “Naidhana” or danger	Generally unfavorable
 

\item[] 8) “Mitra” or friend	Help and alliances; generally favorable
 

 

\item[] 9) “Paramamitra” or supreme friend	Great help; very favorable
 
\end{enumerate}
 

The difficulty given by the third, fifth and seventh positions is said to be less if it occurs at the twelfth, fourteenth and sixteenth Nakshatras from the birth Nakshatra (the second division of nine or paryaya from the birth Nakshatra). It is only slight if it occurs at the twenty-first, twenty-third, and twenty-fifth from it (the third division of nine from the birth Nakshatra). Hence, it is better if the daily Nakshatra is behind the birth Nakshatra in the zodiac than in front of it. This system is the general rule and has its variations.

In addition the Nakshatra previous to ones birth Nakshatra, the twenty-seventh, is often not favorable. For example, if the birth Nakshatra is Shravana. Uttarashadha, the previous Nakshatra, is not good for important actions.

Hence, even if a Nakshatra is generally favorable or unfavorable, it may not be so if we look at its relationship with the birth star (Nakshatra) of a person. It is best to have a Nakshatra that is favorable in both ways.

 

\subsubsection{GENERAL FAVORABILITY OF THE BIRTH NAKSHATRA}

 

Days when the Moon is in ones own birth Nakshatra are generally not favorable for special actions, but are good for career gains, for planting and gardening, for buying and building, learning, rituals, and for meditation. Ones birth Nakshatra is not good for travel, for medical treatment, or for sexual activity. For women, however, it is excellent for marriage.

 

\subsubsection{MOON AND HOUSES}

When the Moon is in the eighth house from its natal sign, it is usually not auspicious, regardless of the Nakshatras involved.

 

\subsubsection{NAKSHATRA AND THE SUN}

For the Nakshatra to be favorable, it should not be in the same sign as the Sun at that time or in the sign before or after that of the Sun. Hence, if the Sun is in Virgo, the Moon should not be in Leo, Virgo, or Libra to be really favorable.

This factor is also considered under the Tithis and the general factors for creating a strong Moon. We should generally avoid a combust or new Moon, as is well known. This is regardless of whatever Nakshatra in which the Moon may be located.