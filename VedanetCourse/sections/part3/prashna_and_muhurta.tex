\section{HORARY ASTROLOGY AND ASTROLOGICAL TIMING}
\subsection{PRASHNA AND MUHURTA}
 

Astrological charts can be done not only for the birth of a person, they can be done for any important moment, event or decision in life. They can be done for marriages, initiation of business ventures, for moves, or for charting the course of any important activity. This branch of astrology is called “Electional Astrology” or “Muhurta” as it is concerned with choosing the right time to do things.

 

Muhurta means a “moment of time.” It refers to a favorable time for initiating a certain venture. As a branch of astrology, it refers to the timing of events.

 

A related branch of astrology is called “Prashna Tantra.” Prashna means a “question.” Tantra means a teaching or a science. This branch of astrology is used for answering questions, which can be of any type. They are usually concerned with whether a person will achieve success in a particular venture. The branch of astrology that deals with such subjects in general is called “Horary Astrology” or “Prashna” and the chart so done is called a “Horary Chart.”

 

In this lesson we will introduce these topics, particularly as they relate to the casting of charts like the birth chart. Additional information on them will be given in the sections on Astrological Forecasting that follow. In addition, factors like transits and planetary periods can be examined. Additional information on these occurs in the sections on Ashtakavarga.

 



 

\subsection{1. PRASHNA OR QUESTION BASED-CHARTS
CHARTS FOR CERTAIN MOMENTS AND EVENTS}
 

A Vedic Astrologer aids in attunement to cosmic law and helps human affairs to flourish in harmony with it.  They may be called upon to judge the course of events that began at a particular time. Or they may be called upon to choose an auspicious time for beginning a particular enterprise.

 

Astrologers can give answers to any specific question – even without knowing the birth data of a person. They simply judge the question relative to the astrological influences current at the time in which the question is asked. Some Vedic astrologers may primarily do this type of question and answer astrology, or Prashna. Many Vedic astrologers do more of this type of astrology than natal astrology because it is more specific. Any number of horary readings can be given for a client, while natal readings are usually done once.

\subsubsection{USE WITH NATAL ASTROLOGY}

 

Prashna can be combined with natal astrology. The natal chart shows the general influences throughout the persons life, with the planetary periods and transits showing those which are current. Prashna shows those influences specific to the question and the situation it is based on.

 

Muhurta or Electional astrology can also be used with natal astrology, and may require this to some degree. It is helpful if the Muhurta chart supports the birth chart and that the time in a persons life is generally favorable to begin the particular venture. For example, if a persons birth chart is not favorable for marriage, even the best Muhurta may not give happiness in marriage.

 

As a whole, the horary and electional side of astrological science is more developed in Vedic than in Western astrology, as the latter is more oriented to the natal chart. Vedic astrologers are frequently asked to time marriages, initiations, or travels or to comment on business, family or spiritual issues via the examination of Prashna and Muhurta charts.

 

\subsubsection{CHARTS FOR THE MOMENT OF THE READING}

 

Astrologers may do a chart for a client based upon the moment the client arrives, and use that chart for predicting various aspects of the clients life. Cross-referencing the birthchart of the client with a horary chart for the moment of their visit gives very important information as to their present needs.

 

If the client has a particular question, like how their health will proceed, how their relationships will go, or anything else for that matter, the moment they put the question to the astrologer becomes the basis for the chart. A chart can be drawn up for any moment, giving insight into the issues at hand.

 

In addition, by looking at the planetary positions in operation for each day, the astrologer can see the trend of the readings to be given that day, and the type of people they are likely to see.

 

For some days that have unfavorable planetary influences – whether generally or for the chart of the astrologer – astrologers may not give readings or they may restrict the number or type that they give. It is not always favorable to give astrological readings if the planetary influences do not support it.

\subsubsection{PRASHNA  FOR SPECIAL EVENTS}

 

Nothing in life is without meaning. Each event follows cosmic law and can be interpreted according to prevailing planetary positions. With computers to calculate charts, it becomes relatively easy to examine charts for a variety of conditions in life.

 

It is helpful in this way to do charts for important moments in our lives. This can help us understand our own greater destiny. Charts for times of marriage, initiation or other ventures can be very significant.

 

In judging marriages, for example, a Prashna chart for the time of marriage can be as important as the comparison of the birthcharts of the couple. Charts done for the moment of death of a person will usually give indications of their next life. Charts done for the time of accidents or when the person was hospitalized will be helpful in prognosis for recovery.

 

\subsubsection{INTERPRETATION OF PRASHNA CHARTS
NATURE OF THE QUESTION}
 

To do a Prashna reading, one must start with the question of the client. Generally, a simple question is best, such as: Will this matter (business, legal issue, travel, relationship, etc.) succeed for me? Avoid ambiguous questions or questions relating to more than one topic. Make sure that your client is quite clear about their question and that they state it clearly. Should the answer not be what the client wants to hear, do not draw up any additional charts according to the hope of the client to find something better.

 

Any resultant questions should also be examined from the Prashna chart. For example, if the question is in regard to success in a business venture and the answer is favorable, to determine the degree and timing of success, use the same Prashna chart. Do not draw up additional charts.

 

The chart is usually drawn up at the time the client presents the question to you. This may be the time they come to see you, if the consultation is in person, or the time they call you if the query is by phone. If it is through the mail, you should use the time you first examine the letter. When it is known, the time that the issue first became important to the client can be used.

 

\subsubsection{PRASHNA AND BIRTH CHARTS}

 

The main point to understand is that Prashna charts are interpreted similarly to birth charts, but there are some differences. The prime difference is that it is the issue, rather than the life of the person, that has to be interpreted. While this course focuses on the birth chart and its ramifications, the same principles can be applied to the Prashna chart. Even the planetary periods can be used to trace the development through time of a particular course of events. Hence, below we will only outline some of the primary factors of interpretation. Prashna requires a shift of orientation but not a fundamental change from the same basic principles of chart interpretation.

 

Three excellent books on the subject of Prashna and Muhurta are Prasna Tantra, Prasna Marga, and Muhurta or Electional Astrology by B.V. Raman.

 

\subsubsubsection{The Importance of the Ascendant}

 

Generally, the Ascendant represents the person posing the question, the client. However, the house we examine for the query depends upon the nature of the question.

 

We look to the first house for issues involving the questioner himself, his happiness, health, success, or whatever is most relevant to the person. The nature of the Ascendant and the influences on it will show the attitude and energy of the questioner and what and how things in general are going to affect him. In addition, the Moon and Navamsha Ascendant should be considered.

 

If these are under primarily benefic influences, the object of the particular question will be fulfilled or the questioner will be successful in achieving their aim. If the query is a simple yes or no, success or failure issue, we need only examine the Ascendant itself. For more complex issues we then examine the other relevant houses.

 

\subsubsection{EXAMINATION OF RELEVANT HOUSES}

 

We look to the relationship between the Ascendant and the houses that relate to the particular topic for issues concerning the questioner. However, if the questioner is asking something on behalf of another person, we read the chart from the house which pertains to the other person.

 

If the question is in regard to the wife, the seventh house should be looked on as the Ascendant. In regard to children, one should look at the fifth house as the Ascendant. In the case of issues involving ones mother, the fourth house should be examined while issues involving the father are examined via the ninth house. In regard to younger brothers, sisters or friends, one looks at the third house; for elder siblings, one looks at the eleventh house, and so on.

 

For example, if the issue is in regard to the younger brothers wife, we look at the seventh from the third or the ninth house. This will tell us the condition, health, happiness, etc. of the party in question.

 

\subsubsection{RELATIONSHIP BETWEEN THE ASCENDANT AND DIFFERENT HOUSES}

 

If the question involves the relationship between the questioner and the party involved, we examine the relationship between the Ascendant and the house in question. For issues of marriage or relationship, or for examining any issues relative to the partner, such as their fidelity, etc., we look at the relationship between the Ascendant and seventh house, much like in the regular birth chart. All the relevant houses can be examined in this manner.

 

In cases of enemies or litigation, the sixth house can be examined as showing the nature, energies and influences of the opponent. For example, if the sixth house is stronger than the Ascendant, the questioner will likely loose their case. The eighth house also has some bearing relative to the issue of opponents. If the eighth house is stronger than the Ascendant, loss of reputation is likely.

 

If one is looking for income, the eleventh house should be examined. It should be strong and in relationship with the Ascendant or its lord. For gaining fame or public success, the tenth house should be examined. The house and its lord should be strong and in connection with the Ascendant and its lord.

 

In other words, we follow the same range of house significations for Prashna as are standard in the birth chart.

 

The relationship between the Ascendant (along with its lord and significator) and the house in question (along with its lord and significator) shows fulfillment of the matter if the houses are positive, obstruction if they are negative.

 

For example, if the question is relative to public recognition and the Ascendant lord is in the tenth house and vice versa, naturally the results should be quite good, and recognition would be gained provided the aspects were positive. If the question is for a person fighting in a war, and the Ascendant lord is in the eighth house and vice versa, then defeat or death is indicated.

 

If the issue is getting married, the situation becomes favorable if the seventh house lord is in the Ascendant or if the Ascendant lord is in the seventh house. Also good is if the Ascendant and seventh house lords conjunct or mutually aspect each other from favorable houses like the first, fourth, fifth, seventh, ninth, tenth or eleventh.

 

If the issue is getting a house (property), if the Ascendant lord and the lord of the fourth house are in conjunction in a favorable house, say the ninth house, this would indicate gain of a house, perhaps with parental or government help.

 

If one is seeking income through property, a relationship between the fourth and eleventh house factors is necessary, with an additional connection with the Ascendant or its lord.

 

For health issues, the Ascendant represents the basic health of the person, the sixth house ones disease potential. The Ascendant should not be under malefic influences, nor should the Ascendant lord be influenced by the sixth house and its lord. If the sixth house and its lord are stronger than the Ascendant and its lord, then we would expect the health of the person to decline. If planets tenant the eighth house, like the Ascendant lord, death through the disease is possible.

 

These are only typical examples. Like interpretation of the birth chart, there are many possibilities. If the aspects and associations are good, then the results will be good.

 

Naturally, things will not always be black and white. Often ventures are mixed in their forecast. Success may take time or be limited in its results. Hence, these variations must be considered while examining the planets.

 

\subsubsubsection{Examining the Prashna and Birth Charts Together}

It can be helpful to compare the Prashna chart with the birth chart of the person. Favorable planetary aspects and exchanges between the two charts will aid in success. Negative ones will cause failure.

 

 

\subsection{2. FACTORS TO CONSIDER IN TIMING OF EVENTS
ASTROLOGICAL FORECASTING OR MUHURTA}
 

Here we will confine ourselves to factors based mainly on the signs, houses and days of the week. In the sections on Astrological Forecasting we will add the larger factors including Nakshatras and Tithis.

 

\subsubsection{MARRIAGE}

There should be no planets in the seventh house. Mars should not be in the eighth house or Venus in the sixth. The Ascendant should be free of malefic influences and not hemmed in by malefics. The Moon should not be associated with other planets. Jupiter, Mercury and Venus should be in the Ascendant or in angles. Malefics should be in Upachaya houses.

 

Gemini, Virgo or Libra Ascendants are best. Taurus, Cancer, Sagittarius and Pisces are average. Aries, Leo, Scorpio, Capricorn, and Aquarius are inferior.

 

Monday, Wednesday, Thursday and Friday are the best days. Tuesday should be avoided, particularly in the morning. Sunday and Saturday are not so favorable either but not as unfavorable as Tuesday. The Moon should preferably be waxing.

 

\subsubsection{CHILDREN}

For conception, the waxing Moon is preferable. The period from four days after the beginning of the womans menstruation to sixteen days after is good. Odd days after her period are better for female children, even days for male children.

 

\subsubsection{PREPARATION OF MEDICINES}

The Ascendant should be in a Cardinal or Mutable Sign. Virgo is generally good. Aries is good for medicines that require a hot potency. The sixth, seventh and eighth houses should be empty. The best days are Thursday, Monday, Wednesday, Friday and Sunday.

The Moon should be strong, particularly for nutritive type medicines.

For antibiotic or antifever medicines, it is good if the Sun, Mars and Saturn are in the Ascendant or in an angle.

 

\subsubsection{OPERATIONS}

The Moon should be waning, as the body fluids will then be low. It should not be in a sign that relates to the part of the body to be treated (like Aries for head operations). Tuesday and Saturday are good days, as a malefic influence is helpful for such harsh actions. Mars should be strong and the eighth house unoccupied by any planet (as it is the house of death). Above all, the Ascendant lord should not be in this house.

 

\subsubsection{TRAVEL}

BY AIR

The Ascendant should be an airy sign, but not afflicted. Mars should not be in the Ascendant, the seventh or eighth. The Moon should be waxing and not near Rahu. Rahu should not be strongly placed in the chart as it causes diseases during air travel or airplane accidents.

 

BY SEA

The Ascendant should be in a watery sign, but not afflicted. The other indications are similar to those under air. There should not be malefic planets in watery signs or in difficult houses like the eighth and the twelfth. Good watery planets like Jupiter and the Moon are helpful.

 

\subsubsection{SUPPLEMENTARY MATERIAL
EXAMINATION OF THE BREATH AND OTHER DIVINATORY METHODS}
 

An important factor used in Prashna – but may generally be used as well in all astrological ventures – is the examination of the breath in which the astrologer examines which nostril of theirs the breath is passing through predominantly.

 

The left nostril breath is favorable for the days of benefic or gentle planets – Monday, Wednesday, Thursday and Friday.
The right nostril breath is favorable for the days of malefic or harsh planets – Sunday, Tuesday and Saturday.
 

This is part of the science of Svara Shastra which we do not have the space to go into here in detail. It is commonly used by yogis. Other factors like directions (in which the questioner sits, for example), or an examination of signs and omens, are part of Horary Astrology. These are sciences in themselves and do not depend directly on the examination of any particular birth chart.

 

\subsubsubsection{Astro-Palmistry}

Vedic Astrology includes many Divinatory Methods. Most common is palmistry, in which the palm is read according to the planets. Western palmistry is based upon the Vedic, having been transmitted to Europe by the gypsies, who originated from India.

 

In this sytem, the planetary mounts on the hand are examined to see whether they are strong or weak. Often palmists prescribe gemstones for weak mounts. This method is used like the birthchart. Some palmists measure the size of the hand and fingers, the different mounts and the distances of the various lines in the hand to gain more precise information.

 

These divinatory methods, however, require a separate course. Here we will confine ourselves to the reading of astrological charts relative to these issues.

 

\subsubsubsection{Numerology}

A number of Vedic astrologers employ different systems of numerology as an interpretative device. Some even use numerology based upon the Western calendar. This is not strictly Vedic as the Western calendar was introduced to India only recently. The Western calendar follows no real logic or astronomical basis but such methods can give helpful results. We do not have the space to go into them here, but Vedic Astrology students should know of their existence.

 

Vedic numerology usually reflects the prime numbers for the planets as Sun-1, Moon-2, Mars-3, Mercury-4, Jupiter-5, Venus-6, Saturn-7, Rahu-8 , and Ketu-9. The numbers of the 12 signs and 27 Nakshatras are also often considered as well.