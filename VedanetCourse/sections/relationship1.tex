\section{RELATIONSHIP ASTROLOGY -
CHART COMPARISON: COMPOSITE READINGS}
 

One of the most important and common usages of Vedic astrology today is for compatibility. This extends from marriage and personal relationship to friendship and business issues. We will focus more on the relationship and marriage compatibility. Examining relationship potentials in charts is an important and complicated issue. Comparing two charts for their compatibility is one of the main things you will be asked to do as an astrologer as relationship is often the key issue in human life. We will examine these factors in this lesson.

 



\subsection{\textbf{We begin with an audio on Relationship Compatibility with Vamadeva that introduces the factors discussed in detail below}}

 

\subsection{GENERAL ISSUES OF RELATIONSHIP ANALYSIS}
 

\subsubsection{Hindu Versus Western Culture}

 

In traditional Vedic Astrology, charts of prospective marriage partners, generally young people whose marriages are being arranged, are judged for compatibility. This is the main emphasis of the astrological analysis of relationship in Vedic Astrology. In modern times, and in Western Astrology, charts are used to help work out issues in relationship, as well as to examine compatibility on different levels. With the ease of divorce and freer sexual relationships today, such relationship examinations are more frequent. Hence, this branch of astrology is different in the West today than it was in older India. It is used more broadly and astrologers will be asked to examine relationship potentials in most of their readings.

 

\subsubsection{Methods of Judging Compatibility}

 

There are a number of methods of comparing different charts for compatibility. Most of these focus on marriage. There are methods that compute compatibility factors, like favorable signs or Nakshatras (the kuta system that we will examine in the next lesson). They add up the various points and see if there are enough in common to guarantee a good match. Such calculations can be helpful but as they are based upon Indian marriage standards which are different from those in the West, we cannot expect them to work here without some modification. There are other methods that compare all relevant factors in both charts. While this second method is more complex, it is more detailed and more accurate so we will examine it first.

 

\subsubsection{Human Relationship Potentials}

 

Some of us have the idea that there is only one person for us in life. Most of the time, however, we have several relationship options on a number of levels. We come together for various reasons, some based on outer factors of physical or emotional attraction, others based on inner karmic or spiritual purposes. And inner and outer affinities can combine in different ways. Hence, we must strive to understand our general relationship potential rather than just fixate on one person we may happen to be involved with. Many people who are married more than once will be married to different types of people. We should see what planetary influences they are opening up to in each case.

 

Some individuals have powerful charts for relationship but may not find any one relationship that really works. Their strong relationship energy may draw them into different relationships. Other individuals have weak charts for relationship but may maintain a distant or detached marriage throughout life. Yet other types may have a life long relationship but out of inertia, not because there is real communication or compatibility.

 

In doing a comparative reading, a good deal of discretion is required. We should not just tell a client that they have a good or bad chart for relationship or that the chart of their potential partner is good or bad. As in all factors, we should not try to make decisions for the client. We should acquaint them with the positive and negative potentials available to them and allow them to make their own decisions. We must remember that relationship is an emotional issue and cannot always be approached along rational lines. We must approach our clients on this issue with the same tact with which we would expect someone to examine our own chart.

 

\subsubsection{Broader Issues of Compatibility}

 

It is not enough to judge compatibility emotionally or sexually. Issues of work, children and spirituality have to be brought into play, not to mention career which is so important today. Two people may love each other or have much in common but may not be able to live together. A certain harmony of style and daily action is required. We may also find a great deal of affinity between two charts. This, however, may not indicate marriage.  It may indicate friendship, a brother-sister relationship or a spiritual tie. In our culture we have a tendency to think that all male-female relationships should involve sexual activity. Hence, we may try to create a sexual relationship out of what is a relationship of another type. The astrologer should be careful not to turn what is an affinity, perhaps even profound, into something good for marriage or relationship, as this is not always the case.

 

Traditional Hindu astrology follows Hindu cultural values. These are of one life long marriage, no divorce, and having children as an essential part of marriage. Since traditional Hindu society is fixed and marriages are usually arranged, the rules for marriage from Hindu astrology are rather strict. We cannot follow that method in this country without modifications.

 

We must hold to the first principle of Vedic Astrology here as elsewhere – that the most important thing in human life is the development of consciousness. Relationship should be part of this but often it is against it. Hence, while we acknowledge the importance of relationship and the natural human urge towards it, we should give the other goals of life their due as well.

 

\subsubsection{Relationship and the Spiritual Life}

 

From the spiritual standpoint, marriage is not necessarily a good thing, nor is divorce necessarily bad, though it is usually very traumatic. If a relationship aids in the conscious growth of both people, it has its value. If it does not do so after a period of time, or if it obstructs inner growth, such a relationship may be better if it came to an end. However, many relationships can be successful even if only one partner is on the spiritual path. This requires openness on the part of the partner who is not on the path.

 

Relationship is always an individual matter that cannot be judged by general rules. We should aid our client in what furthers their aspiration in life and not just project social or religious standards upon them. Certainly marriage is a very powerful psychic tie that should not be entered into carelessly and which cannot be ended without considerable work and effort. Generally, we should only end a marriage relationship if it becomes a real block to spiritual growth, particularly if a pattern of unfaithfulness emerges.

 

If a marriage does not have children, if it is a second or third marriage or has lasted only a short time, it may be a weaker psychic link. We must examine its whole structure and not just place it under a category because of its name.

 

\subsection{COMPARATIVE READING OF CHARTS}
First we will begin by examining the relationship potential within the chart of each individual. On this basis we can better compare the two charts together, which is the second step.

 

\subsubsection{EXAMINATION OF THE INDIVIDUAL BIRTHCHART—}
 

It is best to examine both charts in a relationship issue, starting with each chart by itself. Often what may not be clear in one chart will be obvious in the other.

 

We can learn a great deal by examining the relationship potential and time factors within the individual chart itself. Even if we are doing some comparative analysis, we must start with this as a first step.

 

Determining the type of partner that is good for an individual comes in here. Through an individuals birthchart we see their general relationship potential, which may have several directions of application through different periods of time.

 

However, quite often we are attracted to the kind of partner who is not good for us, and we are not attracted to the partner who is. Our desire nature causes us to seek something new, sensational, or hard to achieve, which appeals to our illusions. We want a beautiful partner but not necessarily a kind or spiritual one. Hence, it is more a matter of living compatibility that we should seek, not necessarily how exciting the partner may be.

 

\subsubsection{ASTROLOGICAL FACTORS}

 

For the male, relationship potential involves examining the positions of Venus, the seventh house and its lord. For the female, it is the position of Jupiter, the seventh house and its lord. For both, the position of Mars should be examined, particularly with reference to Kuja Dosha.

 

Western Astrology considers Mars for the female. This is because western culture emphasizes marriage more as a sexual or emotional passion, as revealed through the Mars-Venus interchange. Vedic Astrology emphasizes dharma or principles of right living. This is seen by the Jupiter-Venus connection. However, in western culture we may have to give more emphasis to the role of Mars.

 

If a person has a highly afflicted seventh house and related factors, no matter how good the partner may be, harmony in marriage is going to be difficult.
So too, if a person has a very good seventh house and related factors, a successful marriage may be easy and may not require the perfect partner to succeed.
 

However, we have to examine the whole character of the chart, whether the person is introverted or extroverted, mental or emotional, spiritual or materialistic in orientation, and so on. Relationship must be based upon a harmony of spirit and values.

 

\subsubsection{PLANETS IN THE SEVENTH HOUSE}
Generally speaking, planets in the seventh house are not good for lasting relationship of a personal nature, though they may be good for public or social expansion, as well as for career.

 

THE SUN in the seventh causes separation and conflict, and while it orients the person towards relationship, the issue often becomes overweighed. The individual may be dependent, controlling or domineering, depending upon whether the Sun is strong or weak. They may use relationship to express their power potential rather than their own action in life. This position, however, is good for public and career work and can allow us to project a strong personality in these outer fields of life.

For women, the Sun in the seventh generally denies marriage. They seek relationships with married men or men beyond their status with whom it is not possible.

 

THE MOON is usually good in the seventh house, particularly for women, unless it is waning or poorly aspected. It gives the capacity for sharing, nurturing, caring, receptivity and sensitivity to others. It is also helpful for public and career issues as it gives the ability to influence people. It can, however, get a person caught in relationship or the social mind. Or it can make the person too emotional about relationship issues.

 

MARS is difficult in the seventh as it creates conflict of will, aggression and possibly even violence. Relationships are apt to be turbulent, dramatic and the individual may insist upon their own point of view. The individual may have a strong impact on others and wield power in society but is often not capable of lasting personal affection. Yet I have seen a number of charts with Mars in the seventh and a life-long marriage. This usually requires that both partners share some work in the world together, are both very achievement oriented, or that one surrenders to the will of the other. These factors serve to neutralize Mars.

 

MERCURY gives a good capacity for communication when it is located in the seventh house. They can relate easily to other people, particularly members of the opposite sex. Yet relationships are apt to be hasty, superficial or transient. Often an early unsuccessful marriage is indicated. The individual may enter into close relationships too quickly or easily. There may be much nervousness and volatility in relationship. Relationships with younger people are often indicated. This position is also good for career, business, communication or writing.

 

JUPITER is generally good in the seventh and gives the basis for a spiritual or dharmic relationship. It shows a loyal, ethical and honest approach to relationship. It gives friendship, happiness, compassion and a longing to share ones activity in life. It is also good for public and career work, which the partner may share. It shows a positive will in relationship and partnership. However, even Jupiter in the seventh can cause difficulties because it can make the person too expansive in relationship or apt to sacrifice their relationships to their principles.

 

VENUS is not always good in the seventh, as it can indicate an overly strong sensuality or sexuality. It does give a proclivity for love, passion and romance. For women, it renders them beautiful, attractive or sexy and usually keeps them in relationship. For men, it is not always good as it can make them feminine or passive but it may give them a beautiful wife. It also gives artistic capacities, like painting or dance. In its own sign or a sign of Mars, it gives a strong sexual nature.

 

SATURN is generally not helpful for relationship in the seventh. It causes separation, alienation and loneliness. The individual may be too introverted or self-involved to relate to others on an intimate level. Conflict, separation or rejection can occur. It also can make us prone to unusual sexual activity, as for example homosexuality. Well placed, however, it can indicate one long-term or life long relationship. Like Jupiter, it is also good here for public or career influence and additionally has its directional strength in this house. It also gives us relationships where there is an age difference. Usually this gives older partners for women and younger partners for men.

 

RAHU causes its typical difficulties and confusions here. Relationships may be superficial, based on a following of external influences or momentary whims. There may be much projection, unrealistic desires and some unwholesome fantasies. There may be strange psychic experiences in relationship. Much depends upon the planet that rules Rahu and some power of influence for career may come from this position. Rahu when strong here can be good for career, like its influence in the tenth house.

 

KETU causes a critical and contracted manner in relationship. The individual may be strongly introverted, eccentric and self-involved. Psychic conflicts may occur, with hidden power issues. Accidents or difficulties may occur in relationship. Again, the planet that rules Ketu should be consulted. Ketu can give a psychic or spiritual partner.

 

A combination of planets in the seventh should be judged by the nature of the planets involved and their house rulership, but generally the more planets are in this house the more difficult for relationship it will be. While a number of planets in the seventh can be good for social prestige, unless they are benefics, they prove difficult for relationship. Combinations of the Moon and Jupiter, with possibly Mercury or Venus are good. Combinations of Mars and Saturn cause many difficulties, if not perversion and violence. These are compounded if the lunar nodes are involved.

 

Malefics exalted or in their own signs in the seventh give much power in life, but still cause difficulties in relationship. For example, a woman with Mars in Aries in the seventh will have a strong will and achievement potential, often good business capacity, but will have difficulty in relationship, particularly on an intimate personal level.

 

\subsubsection{ASPECTS TO THE SEVENTH HOUSE AND ITS LORD}

 

Aspects of malefic planets (the Sun, Mars, Saturn, Rahu, and Ketu and the lords of the sixth, eighth and twelfth) to the seventh house or its ruler tend to cause separation or divorce. The Sun will make us too individualistic or dominating. Saturn will make us too contrary, solitary or fearful. The aspect of Mars can cause conflict or sorrow, and is a factor in Kuja Dosha. Rahu will give us unrealistic expectations and confusion of motivation. Ketu will make us defensive and contracted. The lord of the twelfth will cause us to negate relationship. The lord of the eighth or sixth houses can also be difficult here, not only for relationship but for health.

 

On the other hand, aspects of benefics boost up the seventh house and its effects. Best on the seventh or its lord is the aspect of a benefic Jupiter. It can guarantee happiness in relationship. It creates understanding of the partner and good will in relationship. It is particularly strong if Jupiter is also ruler of the seventh. The aspect of the Moon is generally good unless the Moon is waning and afflicted. The Moon gives a capacity to communicate and sensitivity of feeling, a general openness and sharing.

 

The aspect of Venus to the seventh house is helpful for giving affection and for making the individual attractive. The aspect of Mercury to the seventh house increases communication but does not guarantee lasting contacts.

 

The aspect of the lord of the seventh house to the seventh house that it ruels is usually helpful, unless it is a malefic. But even this is not always good if it comes from the first house, which can show the seventh lord limited to the sphere of the self (first house).

 

\subsubsection{DISPOSITION OF THE LORD OF THE SEVENTH HOUSE}

 

The lord of the seventh should be strong and well-aspected. Placed in the first house it can weaken relationship by keeping us focused on ourselves or on our own work in life. In this case, the indicator of relationship remains in the field of the self. The lord of the seventh in the first house is common in the charts of people who dont marry.

 

In the second house, the seventh lord can cause us to treat relationship as a function of work or livelihood. It also can give income or material advance through relationship. But it can show harm to the partner as it is the eighth house from the seventh.

 

Placed in the third house, the seventh lord can make us independent and impulsive in relationship. In the fourth, it is usually good for marital harmony and shows a receptive mind. It is particularly good for women. In the fifth, it gives a strong romantic nature and possible love marriage. In the sixth, it can indicate a sickly partner, difficulties in relationship, or some work with the partner.

 

In the seventh, it is good socially but not necessarily personally and should be judged by its nature. The lord of the seventh in the seventh is good for benefics like the Moon, Mercury, Venus and Jupiter. It is not good for malefics like the Sun, Mars and Saturn unless they combine with or are strongly aspected by benefics.

 

The lord of the seventh in the eighth can give health problems through relationship or loss through the partner, sometimes the loss of the partner. In the ninth, it is usually good and shows a spiritual or dharmic connection, or grace through the relationship.

 

In the tenth, it shows a prominent or powerful partner, with whom one may share ones career or possibly work for. In the eleventh, it shows more than one marriage and strong goals or gains through marriage. In the twelfth, it can indicate secret pleasure or secret sorrow in partnership.

 

\subsubsection{EFFECTS OF OTHER HOUSES OF RELATIONSHIP}

 

The seventh is not the only house of relationship. The fifth should be consulted as the house of love, romance and children. We should note the planets here and the aspects to this house and its lord like the seventh. An exchange between the lords of the fifth and the seventh indicates a love marriage. The twelfth is another house of pleasure. Venus here, for example, often gives a strong sensual orientation. Mars or Venus in any of these houses, particularly if they are signs of Mars and Venus, or if the two planets aspect each other, give a strong sexual nature.

 

\subsubsection{OTHER PLANETARY POSITIONS}
Besides the disposition of the seventh lord and seventh house there are other planetary positions that have important bearings on relationship.

\subsubsubsection{SUN AND MOON}



 

The Moon represents the general social and emotional capacity of a person. A good Moon is helpful for good relationships. Specifically, the Moon in a mans chart represents his general capacity to relate to women. This should be examined for long term and general marital and domestic happiness. With a weak or detached Moon the man may not have the patience for a long-term relationship.

 

The Sun similarly represents the woma’ns general capacity to relate to men. A weak or afflicted Sun may not give her the openness to male energy necessary for a happy relationship. A woman with a weak Sun will often choose weak men as partners, as will a man with a weak Moon.

 

\subsubsubsection{MARS AND VENUS}

 

For marital or sexual issues, it is important to note the relationship between Venus and Mars in the chart. If Venus and Mars are in close aspect or mutual reception, this gives a strong sexual drive. This is particularly true if they are in fixed signs or in houses of sex (the fifth, seventh, eighth or twelfth). A strong sexual drive, however, is often not conducive to marital happiness, though it projects us into relationship. This is because it causes us to seek additional fulfillment outside of marriage (it is promiscuous).

 

\subsubsubsection{KUJA DOSHA}

 

Mars in certain houses creates difficulties for harmony in marriage and relationship. Some Vedic astrologers take these very seriously, others do not. Generally north Indian astrologers weigh them more heavily than those in the south.

 

Kuja Dosha occurs if Mars is in houses 1, 4, 7, 8 & 12.
Exceptions occur in house 1 if the sign is Aries, house 4 if it is Scorpio, house 7 if it is Capricorn or Pisces, house 8 if it is Cancer and house 12 if it is Sagittarius.
 

Such a placement shows potential conflict in relationship or a negative impact on the life of the spouse. A person with such a planetary placement should generally only marry another person with a similar placement. A too strong Mars in the chart of a woman is particularly difficult for marriage from the Hindu point of view as it would make her dominate her husband and perhaps cause him ill health or even death in extreme cases. Such placements should not be interpreted too simplistically since over one-third of charts have them to at least some degree. I consider Kuja Dosha but dont weigh it too heavily. It is more an issue if Mars is in first, seventh or eighth. Then the other partner should have some form of Kuja Dosha to be on the safe side.

 

\subsubsection{PLANETARY PERIODS}

 

Marriage or close relationship, even if indicated in the chart, may not occur until a favorable planetary period or subperiod comes. If the chart is generally good for marriage but the planetary period if of an adverse planet, this may cause delay or difficulty. Different types of marriage or different marriages will be reflected in the different planetary periods in operation at the time.

 

\subsubsubsection{Results of Planetary Periods}

 

\begin{enumerate}
\item[*] Favorable periods for marriage are those of the lord of the seventh, planets in the seventh or which aspect the seventh (if they are not malefic), planets which are with the lord of the seventh or aspect it (again if they are not malefic).
\item[*] The period of VENUS is generally favorable for marriage, particularly in the minor period of the lord of the seventh. JUPITERS period is also usually good, particularly for women.
\item[*] Periods of SATURN are usually difficult unless it is the lord of the seventh. Saturn in the seventh may not give relationship during its period. Saturns period often gives partnerships with an age difference between the individuals involved.
\item[*] Periods of RAHU naturally bring marriage but it is not always or even usually successful unless well placed. It can bring marriage with a foreigner, with someone with whom there is much illusion and may not last beyond its period.
\item[*] KETUS period is generally not favorable for marriage but can bring marriage if it aspects the seventh. Such marriages seldom last beyond the period of Ketu. Again, we have to consult the rulers of these planets.
\item[*] The period of MARS is also usually not favorable for relationship, particularly for women, unless it is lord of the seventh. Mars major, Venus minor periods (or vice versa) are often very strong for relationship but usually bring romance or sexual encounters rather than marriage.
\item[*] The period of MERCURY is mixed in terms of relationship. It often gives early marriages as lord of the seventh, particularly if its period comes young. If it comes when we are older, it often gives relationships with younger people.
\item[*] The period of the SUN is more favorable for marriage for women, particularly when it is lord of the seventh or the fifth.
\item[*] The period of the MOON is generally favorable for marriage, particularly as lord of the fourth, fifth and seventh houses.
\item[*] The period of the seventh lord tends toward marriage. So do the periods of planets located in the seventh house or planets aspecting the seventh house or its lord both in the birthchart and in the navamsha.
 \end{enumerate}

\subsubsubsection{Varshaphal or Annual Chart}

The Varshaphal or annual chart indicates marriage by good planets in the seventh house or aspecting the seventh house or seventh lord with favorable aspects. The same is true of positions from the Moon. Strength to Jupiter and Venus also helps.

 

\subsubsubsection{Transits}

Transits of planets should also be considered, particularly the slower moving ones like Saturn and Jupiter. When Jupiter and Saturn both influence the seventh house or its lord, marriage is more likely. When a person is at the age for marriage, we should see when the most favorable planetary configurations are likely to occur.

 

\subsection{INDICATIONS OF INCOMPATIBILITY}
 

There are also factors in the individual chart that show difficulty in relationship or non-suitability for partnership.

 

\subsubsection{FACTORS OF DETACHMENT FROM RELATIONSHIP }

 

Other factors of detachment from relationship or emotion should be considered as negative factors for relationship. These include Saturns influence on the Moon or the Sun, particularly their conjunction that makes a person more prone to a life alone or to becoming alienated from others. Saturns influence on the second, fourth, seventh or twelfth houses and their lords promotes detachment as these are houses of personal happiness and relationship. The Sun or Moon with Rahu or Ketu are similar patterns of disruption in relationship or personal identity. Saturns influence on Venus is another consideration, though this may only delay relationship.

 

Such factors as Venus or the Moon in Saturnian signs may be noted as well. Yet these things may cause a detached marriage or an intellectual, spiritual, work or social relationship rather than a domestic or romantic connection.

 

\subsubsection{CHARTS DANGEROUS FOR THE PARTNER}

 

Some individual birthcharts may not only not be good for relationship, they may threaten the life, health success, or well-being of the partner. This was the basis of the Kuja Dosha or the difficult placement of Mars. Such factors include all the issues of Kuja Dosha. They include malefics in the eighth house, like Mars, Rahu, Ketu or Saturn or the lord of the seventh in the eighth house.

 

For example, the chart of a client who had contracted AIDS had Gemini Ascendant with Jupiter, the lord of the seventh, at its maximum degree of debility in the eighth house in Capricorn. This person lost many partners to the disease.

 

If a woman has a very strong Mars, it may prove difficult for the partner, unless his Mars or Sun is very strong. If a woman has a strong Sun, it does not threaten the life of the partner but may repel potential partners. Similarly, a man with a very weak Mars or Jupiter may not be helpful to a woman. A weak Mars will not give him enough courage in life. A weak Jupiter will give him misfortune. We must always remember that whoever we enter into partnership with, we tend to take on the karma in their birthchart.

 

\subsection{2. COMPARISON BETWEEN INDIVIDUAL CHARTS—}
 

After having thoroughly examined the individual chart for its basic relationship potential, we can begin our comparison of the individual charts involved. First, we should examine the chart of the partner the same way and see if the factors for relationship are similar, or if the chart of one or the other will compensate for or augment the difficulties in the charts of the other.

 

Generally, we are attracted to someone different than ourselves. We are not seeking to be married to ourselves. Hence, it is not always good if the charts are too much alike. On the other hand, the two charts must have enough in common to be able to communicate. There should be a balance of corresponding and complimentary factors, like the interchange of masculine and feminine planets.

 

Above all, there should be a sharing of primary goals in life. A person of spiritual bent should be careful in marrying a person of commercial disposition, for example. The long-term compatibility of values and daily activity is more important than short term infatuation. Issues of children, finance and work should be examined as well. Even strong emotional or spiritual ties may not be enough if these issues are not in harmony.

 



 

\subsubsection{EXCHANGES BETWEEN PLANETS}
 

Particularly important are the exchanges of male and female planets between the two charts, mainly the Sun and the Moon and Mars and Venus. The Sun in one chart may be conjunct the Moon in the other, for example.

 

\subsubsubsection{Sun and Moon}

 

Generally, the woman surrenders her Sun to the man and the man surrenders his Moon to her. We can see from a womans Sun how she relates to men generally. Its connection with the Sun of her partner shows us how she relates her partnership potential to him. Similarly, we can see from a mans Moon how he relates to women generally. Its connection with the Moon of his partner indicates his female relationship potential generally.

 

Exchanges of the mans Sun with the womans Moon show a deep harmony of will and emotion. Such exchanges of luminaries, however, do not guarantee a marriage relationship. They show the potential for deep friendship and sympathy and are also often found in the charts of brothers and sisters, fathers and daughters, mothers and sons or any close human tie.

 

An exchange of both luminaries, when the Sun in one chart is conjunct the Moon in the other and vice versa, gives a very strong affinity of characters.

 

A similar affinity may be created when the Sun or Moon in one chart conjuncts the ascendant in the other. It is also good if the luminaries are opposite each other in the chart.

 

\subsubsubsection{Venus and Mars}

 

Mars and Venus show how we project our sexual energy. Venus in a mans chart shows how he projects his sexual energy onto women. Mars in a woman’s chart shows how she projects her sexual energy onto men.

 

Exchanges of Mars and Venus, particularly the mans Mars with the woman’s Venus give a very strong sexual attraction. This, however, is not always good for marriage. It may give a sexual attraction that blinds the person. Where the rest of the chart is not compatible, it can cause many difficulties. Yet where there is compatibility in the chart it can be an important aid in making the relationship stronger. Without some Venus-Mars attraction relationships may be of a more friendship or familial nature.

 


\subsubsubsection{Venus, Sun and Moon}

Venus-Sun exchanges are generally good for male female relationships. The woman’s Venus on a mans Sun is very helpful. The woman”s Moon or Venus on the mans Venus or Moon is also good.

 


\subsubsubsection{Mars, Sun and Moon}

Mars exchanges or aspects on the Sun and Moon, however, are not favorable. A mans Mars on a woman’s Moon is particularly difficult and causes emotional disharmony and conflict.

 


\subsubsubsection{Jupiter Exchanges}

Jupiter-Venus or Jupiter-Mars exchanges between charts are also good. They show an alliance between Dharma (Jupiter) and Kama (Mars and Venus). A womans Jupiter on a mans Mars or a mans Jupiter on a womens Venus are excellent for compatibility.

 

 


\subsubsection{HOUSE INFLUENCES}

 

It is important to compare how the planets in one chart affect the houses in the other. This we can see by using the ascendant in one chart for reorienting the houses in the other.

 

Most important are planets that come into the seventh house in the partners chart or aspect it. Should Venus in the woman’s chart come into the seventh for the mans chart, attraction would be increased. Her Moon in his seventh house is similarly good but its influence is of a general nature.

 

For the woman, it is best if the mans Jupiter comes into her seventh house or at least aspects it. Its aspect on her Moon or the seventh from her Moon is also helpful.

 

Planets coming into the fifth house show love and affection. Planets coming into the ninth give spiritual affinities. Planets coming into difficult houses like the sixth, eighth and twelfth can show problems or conflict.

 

Benefic aspects or associations with the seventh lord are also important. A mans Jupiter on the seventh lord of the woman’s chart, or a woman’s Venus on the seventh lord in a mans chart are particularly good.

 

\subsubsection{THE LUNAR NODES – RAHU AND KETU}

 

The lunar nodes are strong karmic indicators. We often see a strong connection between them in charts of individuals who are very close. They can indicate past life relationships.

 

Often the lunar nodes in one chart will mark the Ascendant in the chart of the partner. Or we may see the nodes in one chart conjuncting the Sun and Moon in the other.

 

Nodal exchanges in relationship are very profound and show a deep psychic connection. They connect the astral bodies of the people involved. They are not always happy relationships, however. When difficult, they make it very hard for us to extract ourselves from the relationship.

 

\subsubsection{COMPARISON OF MAIN FACTORS}
 

It is important to check the harmony between the two charts in terms of the main factors of chart interpretation. Here we examine how the factors relate one by one, rather than by the exchange of factors as examined above.

 

\subsubsection{COMPARISON OF ASCENDANTS}

 

It is helpful to have a good relationship between ascendants. This affords communication and brings our daily actions and spontaneous impulses into harmony.

It is not best if the ascendants are the same as this can lack in balance. Ascendants of the same element are generally good and are in trine with one another. Ascendants ruled by the same planet can be helpful or those ruled by friendly planets.

 

\subsubsection{COMPARISON OF MOONS}

 

As the Moon represents our feelings, our social and personal side, it is important in all relationships. The two Moons should have a good relationship, such as in trines from each other. Moons in opposite signs are usually quite favorable. The planets ruling the Moons in both charts should be friendly to each other. Note also the second lesson on Relationship Compatibility as much of it is concerned with the relationship between Moons which is the basis of the Kuta system.

 

\subsubsection{COMPARISON OF SUNS}

 

As the Sun represents our self or ego, it shows how the private sides of our nature can relate. If there is no relationship between the Suns of two charts, or if they do not relate via other planetary influences, however close the two people may be on the deepest level of the self, they will remain separate or alone.

 

\subsubsection{COMPARISON OF PLANETS BETWEEN THE TWO CHARTS}

 

\paragraph{JUPITER}

As Jupiter indicates dharma or inner nature, it is important to have a good Jupiter relationship in the charts. For example, if Jupiter in one chart occupies good houses like the first, fifth, seventh and ninth, or if it aspects Jupiter in the other chart.

 

\paragraph{VENUS}

Good aspects or friendly relationships between Venuses gives attraction, love and affection.

 

\paragraph{MARS}

Aspects between Mars in different charts can cause difficulties and conflict. Most important is Mars in one chart conjunct the Moon in another. Mars conjunct Mars can also be difficult.

 

\paragraph{MERCURY}

A good relationship between Mercuries is helpful for providing long term good communication, essential for any relationship. But it does not necessarily give anything more than that.

 

\paragraph{SATURN}

Aspects between Saturns can also cause difficulties. More so, Saturn on sensitive points in the partners chart, like the Sun, Moon or Ascendant, can cause suffering or alienation. Saturn conjunct Mars or Venus can also cause problems.



\subsubsection{PLANETARY PERIODS}

 

Even if two charts have much compatibility, it is of little consequence if the planetary periods or timing is not good either in one chart or the other. In addition, the annual chart or Varshaphal and transits should be examined, particularly as to planets in or aspecting the seventh house and seventh lord.

 

\subsubsection{ADDITIONAL CONSIDERATIONS – CHANGE OF CULTURAL ISSUES IN RELATIONSHIP}

 

As our culture changes, the results that the planets give also changes. For example, vocational issues are very different today than in the time when Vedic astrology rules were formulated. Similarly, relationship has changed a lot. Now marriage is not as likely even if the chart might traditionally indicate it because our social situation for it has become more negative. Combinations that used to cause marriage may now only produce a relationship or friendship. Combinations before that would only cause stress in a marriage may more likely today cause divorce. Combinations that used to give children in marriage may today only cause the issue of children to be considered, but not any actual children produced. This means that we must remain flexible in our interpretation of charts, particularly on sensitive issues of relationship, given these variabilities.

 

Another consideration is same sex relationships, bisexuality or transgender. We do not have any set rules for these in the older texts to examine, though the idea of a third sex, something like transgender, is there. Yet observing charts of such individuals certain influences seem more common, though not necessarily determinative. Saturn in the seventh house or aspecting it is one of these. Certain signs, particularly dual signs like Gemini, Virgo and Pisces, as the ascendant or seventh house are another. Rahu and Ketu on the one-seven or first house and seventh house axis is another.

 

For women, fiery planets like the Sun or Mars in the seventh can create a strong character and career but may also draw them into a same sex relationship. Another factor is aspects of Saturn or Mars upon the Moon, which alters the emotional nature in various ways good or bad relative to career and psychology as well.

 

For men, stronger feminine planets in the chart like Venus and the Moon may come into play. Mercury in the seventh house is another factor that can extend to both men and women.

 

We must also consider timing here, as relationship attractions may change with planetary periods, transit or the age of a person.