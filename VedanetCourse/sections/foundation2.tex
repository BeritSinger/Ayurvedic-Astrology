\section{FOUNDATIONS OF VEDIC ASTROLOGY 2 
Houses, Nakshatras and Planetary Aspects}

In this lesson we continue with basic factors of Vedic Astrology. We summarize the main points in the lesson but have supplementary reading as well. This lesson’s prime topics will be Houses, Planetary Aspects and Nakshatras, but we will begin with additional factors of Planet and Sign Relationship.

 

Numbering of topics continues from the previous lesson (page numbers continue from the book Astrology of the Seers). Again we are introducing these topics, which will be explored in more detail as the course proceeds. Make sure here to get  basic familiarity with the terms and concepts involved. You might want to start with your own birthchart in this examination of chart factors.

 Again in this lesson there will be no lesson tests, study exercises or assignments as the lesson itself requires a lot of examination.

 

\subsection{FURTHER INDICATIONS OF THE PLANETS RELATIVE TO THE SIGNS
SIGNS OF EXALTATION AND DEBILITY }(Astrology of the Seers 103-104)

 

Planets do best when located in their signs of exaltation and suffer in their signs of debility. This is a very important consideration and the basis of certain calculations of planetary strength and weakness. Debility positions are opposite exaltation.

 

\begin{enumerate}
\item[*] Sun is exalted in Aries and debilitated in Libra, exact point 10 degrees.
\item[*] Moon is exalted in Taurus and debilitated in Scorpio, exact point 3 degrees.
\item[*] Mars is exalted in Capricorn and debilitated in Cancer, exact point 28 degrees.
\item[*] Mercury is exalted in Virgo and debilitated in Pisces, exact point 15 degrees.
\item[*] Jupiter is exalted in Cancer and debilitated in Capricorn, exact point 5 degrees.
\item[*] Venus is exalted in Pisces and debilitated in Virgo, exact point 27 degrees.
\item[*] Saturn is exalted in Libra and debilitated in Aries, exact point 20 degrees.
\item[*] Rahu and Ketu are not always given signs of exaltation and debility.
 \end{enumerate}

\subsubsection{Mulatrikona Signs for Planets}

 

Vedic astrology has the special concept of Mulatrikona or root trine, indicating other special places in which planets are powerful. These are mainly in the odd-numbered or positive sign that they rule. Mercury is exceptional in that Virgo is a sign that it rules, that it is exalted in, and that it has its Mulatrikona in. These locations give special power to the planet but not as much as exaltation.

 

\begin{enumerate}
\item[*] Sun – Leo, 4-20 degrees
\item[*] Moon – Taurus, 4-20 degrees
\item[*] Mars – Aries, 00-12 degrees
\item[*] Mercury – Virgo, 16-20 degrees
\item[*] Jupiter – Sagittarius, 00-10 degrees
\item[*] Venus – Libra 0-15 degrees
\item[*] Saturn – Aquarius, 00-20 degrees
  \end{enumerate}

Memorize the points of exaltation and debility and Mulatrikona positions for each planet, both in terms of signs and degrees. This is very important as planets gain strength towards their point of exaltation and lose it towards their point of debility. There are rules of cancellation of debility as well that we will examine later in the course.

 

\subsection{PLANETARY FRIENDSHIP AND ENMITY }(Astrology of the Seers 104-106)
 

Learn to determine planetary friendship and enmity along both natural status (unchanging) and temporal status (changing according to house location in a chart). Note that while Vedic Software will determine this for you, such basic information should be something you can calculate yourself.

 

\subsubsection{Natural Planetary Friendship and Enmity}

 

Be familiar with the two main natural groups of natural friends and enemies.

Sun, Mars, Moon, and Jupiter versus \\
Mercury, Venus and Saturn
 

Planets in the same camp will be natural friends, like Sun, Mars, Jupiter and Moon for one camp, or Mercury, Venus and Saturn for the other. Planets in different camps will be natural enemies, like Sun and Mercury, or Mars and Venus. This is a very important factor.

 

\subsubsection{Temporal or Temporary Planetary Friendship and Enmity}

 

The rule for temporal friends and enemies is also simple, but does require special calculation from the actual birth chart. It reflects the house positions of the individual chart, and is not an overall planetary rule like natural friendship and enmity. These house positions will be explained in greater detail below.

 

Planets located in houses 2,3, 4 or 10, 11, 12, from a given planet’s location will be temporal friends.\\
Planets located in the same house as a given planet or in houses 5-9 from it become temporal enemies.
 

Both natural and temporal friendship and enmity can be combined for a composite indication. Note that most Vedic software will make this calculation for you. Planets do better if located in friendly signs and suffer if located in unfriendly signs. This is something you should always examine in every chart at an initial phase of interpretation. We will examine this factor in detail later in this section of the course.

 

\subsection{3. THE TWELVE HOUSES} (Astrology of the Seers, Chapter 7, the Houses: Domains of Planetary Action, pages 111-144)


The houses (bhavas) are perhaps the most important factor in Vedic astrology and must be clearly understood in qualities. The signs reflect more the individual nature or character, while the houses reflect more our outer manifestation and external life affairs. Become capable of explaining the fields of life that relate to each house.

Lesson 6 will go into great detail on the houses and house rulership issues, which is a topic in ints own right. Note the basics here.

 

The signs and houses correlate in meaning at a general level, with each numbered sign and house having similar indications, (though this should not be taken too far as there are differences between sign and house meanings as well):

 

Aries and first house, Taurus and second house, Gemini and third house,
Cancer and fourth house, Leo and fifth house, Virgo and sixth house.
Libra and seventh house, Scorpio and eighth house, Sagittarius and ninth house.
Capricorn and tenth house, Aquarius and eleventh house, Pisces and twelfth house.
 

For example, the first house will reflect the fiery and assertive qualities of Aries, the second house will reflect the earthy and possessive aspects of Venus, the third house will address communication and expression issues like Gemini, the fourth house will cover mind and emotions like Cancer, the fifth house will cover creative intelligence like Leo, the sixth house will deal with health and disease like Virgo, the seventh house will deal with relationship like Libra, the eighth house will deal with secret and occult issues like Scorpio, the ninth house will deal with dharma like Sagittarius, the tenth house will indicate public work and impact like Capricorn, the eleventh house will address social issues like Aquarius, and the twelfth house will address spiritual issues like Pisces.

 

When the same number house and sign are influenced, the result will be stronger. For example, if a malefic planet like Mars afflicts the ninth sign Sagittarius and the ninth house, there is a greater danger of injury to the hips represented by these factors. If a benefic like Jupiter aspects the fifth sign Leo and the fifth house, it will indicate better past life karma. This is a common principle fo chart interpretation.

 

Western and Vedic astrology usually look at the houses with the same general interpretations, like the seventh house and relationship, but there are some variations. For example, the third house in Vedic is more a martial house indicated by Mars, whereas in western is more Mercurial, lie the third sign Gemini. The ascendent or first house is not a martial house like the first sign Aries but relates more to the Sun as the life energy of the person. These factors will also be examined more later in the course.

 

\subsubsection{Calculation of Houses}
 

There are different ways of calculating the houses (bhava), which can be by sign (rashi), by equal house system or by midheaven systems (Astrology of the Seers 113-115). Houses can be determined:

 

\begin{enumerate}
\item[] 1) By sign, or Equal Sign System – each sign starting with the Ascendant will refer to one house, regardless of the degree of the Ascendant or the planets within it.

\item[] 2) By 30 degree sections starting with the Ascendant or Equal House System. So if for example 10 degrees of Taurus is rising, the region 15 degrees around it will mark the first house, or from 25 Aries to 25 Taurus, with the other houses following in order, the second house as 25 Taurus to 25 Gemini and so on.

\item[] 3) By dividing up the area between the Ascendant and the Midheaven, or Midheaven systems like Sripati or Placidus. Midheaven is a special point that varies by the latitude of the place of birth. The Midheaven is not always 90 degrees from the Ascendant. So if we use the Midheaven as the cusp of the tenth house, we will need to divide up the difference into the houses in an unequal matter. This can be done automatically by current software programs by selecting such house system options.
\end{enumerate}
 

As a special note, Western and Vedic astrology interpret house cusps differently, Vedic astrology makes the cusp the middle of the house , while Western astrology makes it the the beginning of the house, though both regarding the cusp as the most powerful part of the house.

 

\subsubsection{Houses from the Moon}

 

Vedic astrology examines houses from the Moon as well as from the Ascendant. The Moon is a second ascendant or an ascendant in its own right. Houses from the Moon show more how the planets affect our feelings and personal happiness, extending to our social image, while positions relative to the Ascendant relate more to outer factors in the material world and how we are placed relative to the public. Generally we weigh the Ascendant as 2/3 in value and the Moon as Ascendant as 1/3 in value. But sometimes the weight of planetary placements and aspects will favor the Moon.

 

This means that if the same house from the Moon and the Ascendant is affected, the results will be stronger for the issues involved. For example, if the fifth house, which governs children, is afflicted from both the Ascendant and the Moon, say by aspects of Saturn and Mars, difficulties with children is more likely.

 

\subsubsection{Referred Houses}

 

We can turn any house into the Ascendant for that person or factor indicated by the specific house examined, what are called “referred houses” (bhavat-bhavm). For example, we can turn the seventh house the Ascendant and use it for reading the condition of the marriage partner from there. In Vedic astrology the house dial is a like a revolving wheel and can be placed in different parts of the chart relative to different considerations. The normal Ascendant or Rashi chart is the main dial but other secondary factors can also be read. We will examine this factor in greater detail in the Workbook.

 

\subsection{HOUSE QUALITIES} (Astrology of the Seers, pages 117-120)

 

There are three basic qualities of the houses, which are of great importance.\\
Angular or Kendra (1, 4, 7, 10)\\
Succedent  (2, 5, 8, 11)\\
Cadent (3, 6, 9, 12) 
 

Planets are usually stronger placed when kendra or angular positions (Houses 1,4, 7, 10). They are weak when placed in cadent houses (Houses 3, 6, 9, 12) They are of moderate strength when placed in  succedent houses (Houses 2,5,8, 11) These qualities roughly parallel the three qualities of the signs as cardinal, fixed and mutable (moveable, fixed and dual). It is a key consideration in looking at the chart for planetary strengths and weakness.

 

\subsubsection{Additional House Groupings}

Trine (trikona) house locations (1, 5, 9) are also auspicious and powerful, for all aspects of the life and character of a person.\\
Upachaya or increasing house locations (3, 6, 10, 11), as special distinction of Vedic astrology  are good for malefics and give powers of competition, endurance and resistance, gaining more power with age.\\
Apachaya or decreasing house locations (1, 2, 4, 7, 8) houses bring initial benefits but cause planets to lose their strength and value over time.\\
Bad or difficult house locations (6, 8, 12), Duhsthanas are the worst positions in the chart, particularly house 8. All planets weak are if located in these positions, particularly benefics.\\
The houses like the signs (following house-sign correspondence principles) also divided according to to the four elements of Earth 2,6,10), Water (4,8,12), Fire (1,5,9) and Air (3,7,11). For example, the first house like the first sign Aries has a fiery quality.\\
The houses area divided according to the four aims of life as dharma or vocation (houses 1,5,9), artha or prosperity (houses 2,6,10), kama or happiness (houses 3,7,11) and moksha or liberation (houses 4,8, 12).
 

Examine the Description of the Houses in the Astrology of the Seers  (121-128). Learn how to describe each house. Become intimate with its qualities, connections and associations. The houses are the key to chart interpretation. You should know the house qualities as clearly as you know those of the signs.

 

\subsubsection{The natural significators of each house:}

These are special planetary significators unique to Vedic astrology. Some houses have more than one depending upon the factors involved.
Sun and the first house\\
Jupiter and the second house\\
Mars and the third house\\
Moon and the fourth house\\
Jupiter and the fifth house\\
Saturn and Rahu and the sixth  house\\
Jupiter and Venus and the seventh house\\
Saturn and the eighth house\\
Jupiter and Sun and ninth house\\
Sun and Mercury and the tenth house, Mars is also strong there\\
Jupiter and the eleventh house\\
Saturn and Ketu and the twelfth house
 

Jupiter is the significator of several different houses (2, 5, 7, 9, 11). Saturn signifies several houses (6, 8, 12) or all the difficult houses or duhsthanas. Remember that if the natural significator of a house is weak, the house is also likely to be weak.

 

\subsection{PRINCIPLES OF HOUSE RULERSHIP} (Astrology of the Seers, pages 131-142)

 

House rulership, meaning the planet that rules a particular house in a given chart, is one of the most important principles of Vedic astrology and the basis of most interpretations and predictions. Many Vedic astrological combinations or yogas are defined by house rulership (like the ruler of the first house located in the sixth house and the ruler of the sixth house located in the first house, which is a combination that causes disease).

 

Examine the principles of house rulership and see how the meaning of planets changes relative to each ascendant according to the houses that they rule (also called the “temporal status” of a planet). But note that this is covered in a special lesson of its own for more detail.***

 

The complication is that the planets, except Sun and Moon, rule over two houses, one which may have good effects and one that may be difficult. The planet will give the results of both houses it rules, though at different times and different ways. For example, in the case of an Aries ascendant, Mars will rule both the first house (Aries) and the eighth house (Scorpio). Mars rulership of the eighth house will bring in some negative energies, while overall the ruler of the first is regarded as a positive planet.

 

When a planet rules both an angle (1, 4, 7, 10) and a trine (1, 5, 9), it  called a Raja Yoga planet . This gives great power, influence and success in life. Some Ascendants have one planet that does this, like Saturn ruling houses 4 and 5, for Libra Ascendant. Yet Raja Yogas can be created by two planets, if one rules and angle and the other rules a trine.

 

Note that planets function according to the nature of the houses that they rule as well as according to their natural qualities. For example, even when a natural malefic like Saturn rules good houses, it can still cause some difficulties, yet as a Yoga Karaka for Libra ruling houses 4 and 5, its overall effects will be quite good. Never forget to look at the houses they are connected to before judging their effects.

\subsubsection{4. THE TWENTY-SEVEN NAKSHATRAS} (Astrology of the Seers 108-109)
 

At this point we are just introducing the 27 Nakshatras by name and position.  Learn the 27 Nakshatras by name and their positions in the zodiac (the 13 degree 20 minute section that each rules). In the long-term try to memorize them. What you need to know at this point is only introductory. The third section of the course will examine the Nakshatras in great detail and several lessons.

THE 27 NAKSHATRAS\\
ASHWINI, 00 00—13 20 Aries:  “The horses head”. It originally represented the head of the sacrificed horse that symbolized the Sun, the year and the beginning of the cycle of time.Deity—the Ashwins, the twin horsemen (like Castor and Pollux of the Greeks).\\
BHARANI, 13 20—26 40 Aries:  “The bearers”. It is symbolized by the female reproductive organ. Deity—Yama, the God of death and immortality.\\
KRITTIKA, 26 40 Aries—10 00 Taurus:  “The razor”, which is also its symbol. It corresponds to the stars of the Pleiades, the small cluster of six stars in Taurus. Deity—Agni, the God of fire.
ROHINI, 10 00—23 20 Taurus:  “The red or ruddy female deer or antelope”, from its prime star, red Aldeberan or Alpha Taurus. Deity—Prajapati, the lord of creation.\\
MRIGASHIRAS, 23 20 Taurus—06 40 Gemini:  “The antelopes head”, or the head of the sacrificed wild animal or creator God. Now it is related to the head of the constellation Orion, but originally consisted of the three stars in his belt. Deity—Soma, the God of immortality.\\
ARDRA, 06 40—20 00 Gemini:  “The moist”, marked by the red star Betelgeuse, Beta Orion. Deity—Rudra (Shiva), the God of the storm and the bowman or hunter.\\
PUNARVASU, 20 00 Gemini—03 20 Cancer:  “Return of the light”, marked by Castor and Pollux, Alpha and Beta Gemini. Deity—Aditi, the Mother of the Gods who is sometimes identified with the earth but mainly represents the sky.\\
PUSHYA, 03 20—16 40 Cancer:  “The nourisher”, Deity—Brihaspati, the teacher of the Gods. Deity of the planet Jupiter.\\
ASHLESHA, 16 40—30 00 Cancer:  “The serpent”, Deity—Sarpa, the Serpent in all forms.\\
MAGHA, 00 00—13 20 Leo:  “the beneficent”, marked by Regulus, Alpha Leo, Deity—the Fathers or Ancestors.\\
PURVA PHALGUNI, 13 20—26 40 Leo:  “The earlier fig tree”, Deity—Bhaga, the Sun of bliss.\\
UTTARA PHALGUNI, 26 40 Leo—10 00 Virgo:   “The later fig tree”, marked mainly by Denebola, Beta Leo, Deity—Aryaman, the Sun as the beloved, the friend or the helper.\\
HASTA, 10 00—23 20 Virgo:  “The hand”, corresponding to the constellation Corvus. Deity—Savitar, the Sun of inspiration (Apollo).\\
CHITRA, 23 20 Virgo—06 40 Libra:  “The brilliant”, marked by the star Spica, Alpha Virgo. Deity—Twashtar, the Divine craftsman and demiurge.\\
SWATI, 06 40—20 00 Libra:  “The sword”, marked by the star Arcturus, Alpha Bootes. Deity—Vayu, the God of Air or the Wind, also Prana or the life-force.\\
VISHAKHA, 20 00 Libra—03 20 Scorpio:  “The two branches”, marked mainly by Alpha Libra. Deity—Indragni, the dual Gods of Fire and Thunder.\\
ANURADHA, 03 20—16 40 Scorpio:  “Subsequent success, following or devotion”, Deity—Mitra, the Divine friend and lord of compassion.\\
JYESHTA, 16 40—30 00 Scorpio:  “The eldest”, marked mainly by Antares, Alpha Scorpio. Deity—Indra, God of lightning and perception.\\
MULA, 00 00—13 20 Sagittarius:  “The root”, marked by the stars in the tail of Scorpio. Deity—Nirriti, the Goddess of disaster and negation.\\
PURVASHADHA, 13 20—26 40 Sagittarius:  “The earlier victory”. Deity—Apas, the Goddess of the Waters or cosmic sea.\\
UTTARASHADHA, 26 40 Sagittarius—10 00 Capricorn:  “The later victory”. Deity—Vishwedevas, the Universal Gods.\\
SHRAVANA, 10 00—23 20 Capricorn:  “The famous or renowned”, marked mainly by the star Altair, Alpha Aquila, Deity—Vishnu, the Pervador\\
DHANISHTA OR SHRAVISHTA, 23 20 Capricorn—06 40 Aquarius:  “The most wealthy or most famous”, Deity—the Vasus, the Gods of light.\\
SHATABHISHAK, 06 40—20 00 Aquarius:  “What has a hundred medicines”. Deity—Varuna, the God of the cosmic ocean or heavenly waters.\\
PURVA BHADRA, 20 00 Aquarius—03 20 Pisces:  “The earlier auspicious one”, marked mainly by Alpha Pegasus. Deity—Aja Ekapat, the one-horned goat or unicorn.\\
UTTARA BHADRA, 03 20—16 40 Pisces:  “The later auspicious one”, Deity—Ahir Budhnya, the Dragon of the depths.\\
REVATI, 16 40—30 00 Pisces:  “The rich or splendorous”, Deity—Pushan, the Sun in his protective, nourishing or fostering role, particularly relative to the Earth.\\
 

Nakshatras are a subtler division than the signs and help us understand how different parts of signs work. The main practical usage of Nakshtras is for determining planetary periods (dashas) through the Vimshottari dasha system. Yet we do consider Nakshatras as personality types, much like the Sun signs of western astrology. In addition we can judge the effects of planets relative to the Nakshatras in which they are located. Sometimes the Nakshatras are called Lunar Mansions or Lunar Asterisms reflecting their connection with the Moon, but they do have other connections as well and we do not like to use these terms.

 

 



 

\subsection{5. PLANETARY ASPECTS AND ASSOCIATIONS} (Astrology of the Seers, Chapter 9, pages 145-160)
 

Aspects are key factors in astrological interpretation and are highly emphasized in western astrology, where they are calculated with much precision. Planetary aspects or drishtis (meaning views or sights) are also important in Vedic astrology but calculated different than in western astrology, and in a more general manner. The Vedic view of aspects are largely sign based, from that of Western Astrology is degree based. Remember these differences between Western and Vedic usage and determination of planetary aspects. In addition Vedic astrology has other tools like friendship and enmity to judge planetary relations even when specific aspects may not exist. It is not as aspect centered as is western astrology.

 

Learn the major aspects of the planets as they occur by sign.

 

Each planet aspects the sign seventh from it.
For Sun, Moon, Mercury and Venus, these are the only aspects that the planets have.
For example, if the Moon is in Taurus it will aspect Scorpio as the seventh sign from it.
Each planet is also considered to have a direct relationship like an aspect with planets that it shares the same sign, which is called a conjunction.
 

\subsubsection{SPECIAL ASPECTS OF MARS, JUPITER AND SATURN}

 

Mars has additional special aspects on the fourth and eighth signs from its location.
Jupiter has additional special aspects on the fifth and ninth signs from its location.
Saturn has additional special aspects on the third and tenth signs from its location.
 

These special aspects give special power to these three planets, which relate to fire and Pitta (Mars), water and Kapha (Jupiter) and air and Vata (Saturn).

 

\subsubsection{Rahu and Ketu}

Sometimes Rahu and Ketu are given trinal aspects like Jupiter, but this is a secondary opinion. The Rahu-Ketu axis in the chart, including the opposite signs in which it occurs, has its importance in astrological interpretation as well. It forms important Yogas like Kala Sarpa.

 

\subsubsection{Aspects in Divisional Charts}

Note that the same aspects can be used in divisional charts like the Navamsha, though some Vedic astrologers do not use them. We find them to be very useful and always examine them.

 

\subsubsection{Determination of Aspects by Sight}

You should be able to determine these major aspects in the Vedic chart by sight alone, by merely looking at the chart (this is the basis of memorizing charts). That is why there is often  no table of aspects given on the Vedic chart as there is in most western charts. Such simple by sign aspects are easier to see than the degree aspects of western astrology. Yet we should note that when aspects are close by degrees, particularly conjunction and opposition (same sign or opposite sign), they may be stronger in their effects.

 

\subsubsection{Sambandha or Full Relationship between Planets}

Note the principles for determining Sambandha, or complete association between planets, which consists of either conjunction (same sign), mutual full aspect, or exchange of signs like Mars in Gemini ruled by Mercury and Mercury in Scorpio ruled by Mars.. This brings planets into a very close relationship.

 

\subsubsection{Combust Planets}

Planets become combust (burned up by being too close to the Sun). Generally any planet in the same sign with the Sun may likely be combust, generally within fifteen degrees of the Sun.  Generally combust for planets close to the Sun as Mercury and Venus, is not as severe, as they are always not far from the Sun. Of these two combust Venus is the worst. Meanwhile, combust for the outer planets Mars, Jupiter and Saturn is more severe. Combust Saturn is usually the most difficult.

Combustion also affects house ruler. For example, a combust Ascendant Lord can have problems for a person, particularly in terms of health. A combust fourth lord may not bode well for the mother, by way of another example.

 

\subsubsection{Planetary War}

Planetary war (graha yuddha) is generally said to occur if planets are in a conjunction of less than one degree. Generally the planet with the lower number of degrees and minutes will win the war, but some variant opinions are out there. The planet that loses the war becomes much weaker.

 

\subsubsection{Separative Planets}

The Sun, Saturn, Rahu and the lord of the twelfth house (and sometimes Mars) are considered to be separative planets and negate or remove us from the qualities of the house or planet they influence. On the other hand, Moon, Venus and Jupiter tend to strengthen the positive qualities of the house or planet they influence. These rules are extensions of the natural benefic or malefic statys of planets.

Planets or houses with natural malefic planets on either side suffer, called being “hemmed in between malefics” or Papakartari Yoga.\\
Planets or houses with natural benefic planets on either side do well, called being “hemmed in between benefics” or Shubhakartari Yoga.\\
Always look to the disposition of planets around any given planet, mainly in adjacent signs, which can be as important as aspects. Do not examine the planet in a single sign only. Consider not only aspects but planetary proximity, friendships and enmities.

 

\subsubsection{Weight of Planetary Aspects}

Note the weight of aspects in a chart, particularly relative to natural benefic and malefic planets, including which planets have the most influence on the chart.

See which planets or houses have the most aspects upon them in a given chart. As Mars, Jupiter, and Saturn have special aspects they become particularly powerful and generally indicate the fire (Pitta), water (Kapha) and air (Vata) factors in the chart.

 

\subsection{PLANETARY YOGAS}
 

Yogas are general names for combined planetary influences. They are defined in various ways relative to house or sign location, house or sign rulership, and aspects, or even special patterns of planetary influence. The concept of yogas is broader and diverse and can be more important than aspects in Vedic astrology but may include them. Yoga are the culminating and most important aspect of chart interpretation and are the subject of extensive analysis in advanced studies in Vedic astrology. Their are entire books on this subject.

 

Very important and a good place to start are Mahapurusha Yogas, formed by planets being located in an angle from the Ascendant or Moon as well as in their own sign or exalted, does not include Sun or Moon. These are important for determining planetary types of people which are often created by these planetary Yogas.

 

Yogas affecting the Moon are important as the Moon is like another Ascendant. The Moon does best with other planets, particularly benefics, with or around it. It suffers from isolation, which can lead to debility and depression. Yet many other Yogas exist. We will discuss these in greater detail later in the course as they are very important overall.